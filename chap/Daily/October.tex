\chapterimage{chap32.jpg}
\chapter{October}
\section{Week \Rmnum{1}}
\hl{\textbf{\textit{October 1}}}

1. 求累次积分$\int_{0}^{\frac{\pi}{4}}d\theta\int_{0}^{2\cos\theta}f(\rho\cos\theta,\rho\sin\theta)\rho d\rho$的等价形式
\myspace{1}
\begin{solution}
	
\end{solution}
\myspace{1}

\hl{\textbf{\textit{October 2}}}

1. 求$\int_{0}^{\frac{\pi}{4}}d\theta\int_{0}^{\frac{1}{\cos\theta}}\rho^2d\rho+\int_{1}^{\sqrt{2}}dx\int_{0}^{\sqrt{2-x^2}}\sqrt{x^2+y^2}dy$
\myspace{1}
\begin{solution}
	
\end{solution}
\myspace{1}

\hl{\textbf{\textit{October 3}}}

1. 求$\int_{0}^{1}dy\int_{y}^{1}\sqrt{2xy-y^2}dx$
\myspace{1}
\begin{solution}
	
\end{solution}
\myspace{1}

\hl{\textbf{\textit{October 4}}}

1.求$\lim\limits_{n\rightarrow +\infty}\dfrac{\sqrt[n]{2}-1}{\sqrt[n]{2n+1}}\left[\int_{1}^{\frac{1}{2n}}e^{-y^2}dy+\int_{1}^{\frac{3}{2n}}e^{-y^2}dy+\cdots+\int_{1}^{\frac{2n-1}{2n}}e^{-y^2}dy \right]$
\myspace{1}
\begin{solution}
	
\end{solution}
\myspace{1}

\hl{\textbf{\textit{October 5}}}

1. 设$D$是由$0\leq x\leq 1,0\leq y\leq 1$所确定的平面区域,求$\iint\limits_{D}\sqrt{x^2+y^2}dxdy$
\myspace{1}
\begin{solution}
	
\end{solution}
\myspace{1}

\hl{\textbf{\textit{October 6}}}

1.计算二重积分$\int_{\frac{\pi}{4}}^{\frac{3\pi}{4}}d\theta\int_{0}^{2\sin\theta}\left[ \sin\theta+\cos\theta\sqrt{1+r^2\sin^{2}\theta}\right]r^2dr $
\myspace{1}
\begin{solution}
	
\end{solution}
\myspace{1}

\hl{\textbf{\textit{October 7}}}

1. 已知平面区域$D=\{(x,y)||x|+|y|\leq \dfrac{\pi}{2}\}$,记$I_{1}=\iint\limits_{D}\sqrt{x^2+y^2}d\sigma$,$I_{2}=\iint\limits_{D}\sin\sqrt{x^2+y^2}d\sigma$,$I_{3}=\iint\limits_{D}(1-\cos\sqrt{x^2+y^2})d\sigma$,比较$I_{1},I_{2},I_{3}$的大小
\myspace{1}
\begin{solution}
	
\end{solution}
\myspace{1}

\section{Week \Rmnum{2}}
\hl{\textbf{\textit{October 8}}}

1. 已知平面区域$D=\{(x,y)||x|+|y|\leq 1\}$,记$I_{1}=\iint\limits_{D}(2x^{2}+\tan(xy^2))dxdy$,$I_{2}=\iint\limits_{D}(x^{2}y+2\tan(y^2))dxdy$,$I_{3}=\iint\limits_{D}(|xy|+y^2)dxdy$,比较$I_{1},I_{2},I_{3}$的大小
\myspace{1}
\begin{solution}
	
\end{solution}
\myspace{1}

\hl{\textbf{\textit{October 9}}}

1. 可微函数 $f(x)$ 满足 $f'(x)=f(x)+\int_{0}^{1}f(x)dx$,且$f(0)=1$,求$f(x)$
\myspace{1}
\begin{solution}
	
\end{solution}
\myspace{1}

\hl{\textbf{\textit{October 10}}}

1. 设 $f(x)$ 是可导函数,且$f(0)=0$,$g(x)=\int_{0}^{1}xf(tx)dt$满足方程$f'(x)+g'(x)=x$,则由曲线$y=f(x),y=e^{-x}$及直线$x=0,x=2$围成的平面图形的面积
\myspace{1}
\begin{solution}
	
\end{solution}
\myspace{1}

\hl{\textbf{\textit{October 11}}}

1. 设函数 $f(x)$二阶可导,且$f'(x)=f(1-x),f(0)=1$,求$f(x)$
\myspace{1}
\begin{solution}
	
\end{solution}
\myspace{1}

\hl{\textbf{\textit{October 12}}}

1. 设函数$f(x)$连续,且对任意实数$x,h$满足$f(x+h)=\int_{x}^{x+h}t\left[f(t+h)+t^2 \right]dt+f(x)$,$\lim\limits_{x\rightarrow 0 }\left[ 1+f(x)\right]^{\frac{1}{x^4}}=a(a>0) $,求$f(x)$表达式和常数 $a$
\myspace{1}
\begin{solution}
	
\end{solution}
\myspace{1}

\hl{\textbf{\textit{October 13}}}

1. 设$f(x)$为$[0,+\infty)$上的正值连续函数,已知曲线$y=\int_{0}^{x}f(u)du$和$x$轴及直线$x=t(t>0)$所围成区域绕$y$轴旋转所得体积与曲线$y=f(x)$和两坐标轴及直线$x=t(t>0)$所围区域的面积之和为$t^2$,求曲线$y=f(x)$的方程
\myspace{1}
\begin{solution}
	
\end{solution}
\myspace{1}

\hl{\textbf{\textit{October 14}}}

1.下列级数收敛的是:
\begin{itemize}
	\item A. $\sum\limits_{n=2}^{+\infty}\dfrac{1}{\ln(n!)}$
	\item B. $\sum\limits_{n=1}^{+\infty}(2-\dfrac{1}{n})^{n}\ln(1+\dfrac{1}{n2^{n}})$
	\item C. $\sum\limits_{n=1}^{+\infty}\dfrac{(-2)^{n}n^2+e^{n}}{ne^{n}}$
	\item D. $\sum\limits_{n=2}^{+\infty}\dfrac{(-1)^{n}e^{n}}{3^{n}-2^{n}}$
\end{itemize}
\myspace{1}
\begin{solution}
	
\end{solution}
\myspace{1}

\section{Week \Rmnum{3}}

\hl{\textbf{\textit{October 15}}}

1.下列级数条件收敛的是:
\begin{itemize}
	\item A. $\sum\limits_{n=1}^{+\infty}\ln\left( 1+\dfrac{(-1)^n}{\sqrt{n}}\right) $
	\item B. $\sum\limits_{n=1}^{+\infty}\dfrac{(-1)^n}{\sqrt{n}}\ln(1+\dfrac{1}{n})$
	\item C. $\sum\limits_{n=1}^{+\infty}\dfrac{(-1)^{n}\left[(-1)^{n}+\ln n \right] }{n}$
	\item D. $\sum\limits_{n=2}^{+\infty}\dfrac{(-1)^{n}}{n\ln n}$
\end{itemize}
\myspace{1}
\begin{solution}
	
\end{solution}
\myspace{1}

\hl{\textbf{\textit{October 16}}}

1. 讨论级数 $\sum\limits_{n=1}^{\infty}\dfrac{n}{1^{\alpha}+2^{\alpha}+\cdots+n^{\alpha}}$的敛散性
\myspace{1}
\begin{solution}
	
\end{solution}
\myspace{1}

\hl{\textbf{\textit{October 17}}}

1.已知级数 $\sum\limits_{n=1}^{\infty}a_{n}$绝对收敛,级数$\sum\limits_{n=1}^{\infty}(b_{n+1}-b_{n})$条件收敛,判断级数$\sum\limits_{n=1}^{\infty}b_{n}a_{n}^{2}$的敛散性
\myspace{1}
\begin{solution}
	
\end{solution}
\myspace{1}
 
\hl{\textbf{\textit{October 18}}}

1.已知正项级数$\sum\limits_{n=1}^{\infty}a_{n}$发散,则下列级数一定收敛的是:
\begin{itemize}
	\item A. $\sum\limits_{n=1}^{\infty}\dfrac{(-1)^{n}a_{n}}{n}$
	\item B. $\sum\limits_{n=1}^{\infty}\dfrac{a_{n}}{n}$
	\item C. $\sum\limits_{n=2}^{\infty}\dfrac{a_{n}}{a_{1}+a_{2}+\cdots+a_{n}}$
	\item D. $\sum\limits_{n=1}^{\infty}\dfrac{a_{n}}{n^{3}+a_{n}^{2}}$
\end{itemize}
\myspace{1}
\begin{solution}
	
\end{solution}
\myspace{1}

\hl{\textbf{\textit{October 19}}}

1.已知级数$\sum\limits_{n=1}^{\infty}a_{n}$收敛,下列四个级数一定收敛的个数:
\begin{itemize}
	\item A. $\sum\limits_{n=1}^{\infty}a_{n}^{2}$
	\item B. $\sum\limits_{n=1}^{\infty}\ln(1+a_{n})$
	\item C. $\sum\limits_{n=1}^{\infty}(a_{2n}-a_{2n-1})$
	\item D. $\sum\limits_{n=1}^{\infty}(a_{n+1}^{2}-a_{n}^{2})$
\end{itemize}
\myspace{1}
\begin{solution}
	
\end{solution}
\myspace{1}

\hl{\textbf{\textit{October 20}}}

1.设$a_{n}$为曲线 $y=\sin x,(0\leq x\leq n\pi)$与$x$轴所围区域绕$x$轴旋转所得到旋转体的体积,求级数$\sum\limits_{n=2}^{\infty}\dfrac{(-1)^{n}\pi^{2}}{2a_{n+1}}$的和
\myspace{1}
\begin{solution}
	
\end{solution}
\myspace{1}


\hl{\textbf{\textit{October 21}}}
1. 若$f(x)=\lim\limits_{n\rightarrow+\infty}\int_{0}^{1}\dfrac{nt^{n-1}}{1+e^{xt}}dt$,求$\int_{0}^{+\infty}f(x)dx$
\myspace{1}
\begin{solution}
	
\end{solution}
\myspace{1}

2.证明级数$\sum\limits_{n=1}^{\infty}\dfrac{1+\frac{1}{2}+\cdots+\frac{1}{n}}{(n+1)(n+2)}$收敛并求其和
\myspace{1}
\begin{solution}
	
\end{solution}
\myspace{1}
\section{Week \Rmnum{4}}

\hl{\textbf{\textit{October 22}}}

1. 设$I_{n}=\int_{0}^{\frac{\pi}{4}}\tan^{n}xdx$,其中$n$为正整数

(1). 若$n\geq 2$,计算$I_{n}+I_{n-2}$

(2). 设$p$为实数,讨论级数$\sum\limits_{n=1}^{+\infty}(-1)^{n}I_{n}^{p}$的绝对收敛性和条件收敛性
\myspace{1}
\begin{solution}
	
\end{solution}
\myspace{1}

\hl{\textbf{\textit{October 23}}}

1. 设幂级数$\sum\limits_{n=0}^{+\infty}\dfrac{(2n)!!}{(2n+1)!!}\dfrac{x^{2n+2}}{n+1}$

(1).求该幂级数的收敛区间以及和函数

(2).求级数$\sum\limits_{n=1}^{+\infty}\dfrac{n!}{(2n+1)!!}\dfrac{1}{n+1}$的和

(3).$2f(\dfrac{1}{\sqrt{2}})-1=\dfrac{\pi^{2}}{8}-1$
\myspace{1}
\begin{solution}
	
\end{solution}
\myspace{1}

\hl{\textbf{\textit{October 24}}}

1. 设函数$f(x)$在$(0,+\infty)$上连续可微,$\lambda$为实数,证明:当且仅当$f(x)e^{\lambda x}$单调不减时,$f'(x)+\lambda f(x)$单调不减
\myspace{1}
\begin{solution}
	
\end{solution}
\myspace{1}

2.设$\Omega$为区域$\dfrac{x^{2}}{a^{2}}+\dfrac{y^{2}}{b^{2}}+\dfrac{z^{2}}{c^{2}}\leq 1$,求$\iiint\limits_{\Omega}(3x+2y+z)^{2}dv$
\myspace{1}
\begin{solution}
	
\end{solution}
\myspace{1}

\hl{\textbf{\textit{October 25}}}

1. 设曲线$C:x^{2}+y^{2}=2x$,求$\oint_{C}\dfrac{(x+y+1)^{2}}{(x-1)^{2}+y^{2}}ds$
\myspace{1}
\begin{solution}
	
\end{solution}
\myspace{1}

\hl{\textbf{\textit{October 26}}}

1.设曲面$\Sigma$为$x^{2}+y^{2}+z^{2}=2y$,求$\oiint\limits_{\Sigma}(x^{2}+2y^{2}+3z^{2})dS$
\myspace{1}
\begin{solution}
	
\end{solution}
\myspace{1}

\hl{\textbf{\textit{October 27}}}

1. 计算线积分$I=\oint_{L}\dfrac{xdy-ydx}{x^{2}+xy+y^{2}}$,其中$L$为$|x|+|y|=1$,其方向为逆时针方向
\myspace{1}
\begin{solution}
	
\end{solution}
\myspace{1}

\hl{\textbf{\textit{October 28}}}

1.计算曲线积分$I=\oint_{L}\left[\dfrac{4x-y}{4x^{2}+y^{2}}-\dfrac{y}{(x-1)^{2}+y^{2}} \right]dx+\left[\dfrac{x+y}{4x^{2}+y^{2}}+\dfrac{x-1}{(x-1)^{2}+y^{2}} \right]dy $,其中$L$是$x^{2}+y^{2}=4$,方向为逆时针方向
\myspace{1}
\begin{solution}
	
\end{solution}
\myspace{1}

\hl{\textbf{\textit{October 29}}}

1. 设$\Omega$是由平面曲线$\left\lbrace 
\begin{array}{l}
	4y^{2}+z^{2}=4\\
	x=0
\end{array}
\right.(z\geq 0)$绕$z$轴旋转一周形成的空间曲面,取上侧,计算曲面积分$I=\iint\limits_{\Omega}\dfrac{x^{2}ydydz+y^{2}zdzdx+(z^{2}+1)dxdy}{\sqrt{x^{2}+y^{2}+(\frac{z}{2})^{2}+3}}$
\myspace{1}
\begin{solution}
	
\end{solution}
\myspace{1}

\hl{\textbf{\textit{October 30}}}

1. 设函数$f(x,y,z)$在区域$\Omega=\{(x,y,z)|x^{2}+y^{2}+z^{2}\leq 1\}$上具有连续一阶偏导数,且满足$\dfrac{\partial^{2}f}{\partial x^{2}}+\dfrac{\partial^{2}f}{\partial y^{2}}+\dfrac{\partial^{2}f}{\partial z^{2}}=\sqrt{x^{2}+y^{2}+z^{2}}$,计算$\iiint\limits_{\Omega}(x\dfrac{\partial f}{\partial x}+y\dfrac{\partial f}{\partial y}+z\dfrac{\partial f}{\partial z})dxdydz$
\myspace{1}
\begin{solution}
	
\end{solution}
\myspace{1}

\hl{\textbf{\textit{October 31}}}

1. 设$A$为三阶方阵,并有可逆矩阵$P=(p_{1},p_{2},p_{3})$,$p_{i}(i=1,2,3)$为三维列向量,使得
$$P^{-1}AP=\begin{pmatrix}
	1&0&0\\0&1&1\\0&0&1
\end{pmatrix}$$

(1).证明:$p_{1},p_{2}$是方程$(E-A)x=0$的解,$p_{3}$是方程$(E-A)x=-p_{2}$的解,且$A$不可相似对角化

(2). 当$A=\begin{pmatrix}
	2&-1&-1\\2&-1&-2\\-1&1&2
\end{pmatrix}$时,求可逆矩阵$P$,使得$P^{-1}AP=\begin{pmatrix}
1&0&0\\0&1&1\\0&0&1
\end{pmatrix}$
\myspace{1}
\begin{solution}
	
\end{solution}
\myspace{1}