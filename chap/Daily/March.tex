\chapterimage{chap29.jpg}
\chapter{March}
\section{Week \Rmnum{1}}
\textcolor{orange}{March 1}

\begin{example}[][Exam: 29.1.1]
	求不定积分 $\int\dfrac{1}{\cos^{3} x}dx$
\end{example}

\begin{example}[][Exam: 29.1.2]
	求不定积分 $\int\dfrac{2+x}{(1+x^{2})^{2}}dx$
\end{example}
\myspace{1}

\textcolor{orange}{March 2}

\begin{example}[][Exam: 29.1.3]
	求不定积分 $\int\dfrac{\arcsin\sqrt{x}+\ln x}{\sqrt{x}}dx$
\end{example}

\begin{example}[][Exam: 29.1.4]
	求不定积分 $\int e^{x}\arcsin\sqrt{1-e^{2x}}dx$
\end{example}
\myspace{1}

\textcolor{orange}{March 3}

\begin{example}[][Exam: 29.1.5]
	已知 $f(\ln x)=\dfrac{\ln(1+x)}{x}$, 求 $\int f(x)dx$
\end{example}

\begin{example}[][Exam: 29.1.6]
	求定积分 $\int_{0}^{\frac{\pi}{2}}x\sin^{2}x\cos^{2}xdx$
\end{example}
\myspace{1}

\textcolor{orange}{March 4}

\begin{example}[][Exam: 29.1.7]
	求定积分 $\int_{0}^{\pi^{2}}\sqrt{x}\cos\sqrt{x}dx$
\end{example}

\begin{example}[][Exam: 29.1.8]
	求定积分 $\int_{0}^{2}x\sqrt{2x-x^{2}}dx$
\end{example}
\myspace{1}

\textcolor{orange}{March 5}

\begin{example}[][Exam: 29.1.9]
	求定积分 $\int_{0}^{1}x(1-x^{4})^{\frac{3}{2}}dx$
\end{example}

\begin{example}[][Exam: 29.1.10]
	求定积分 $\int_{1}^{2}(x-1)^{2}(x-2)^{2}dx$
\end{example}
\myspace{1}

\textcolor{orange}{March 6}

\begin{example}[][Exam: 29.1.11]
	求定积分 $\int_{-2}^{2}x\ln(1+e^{x})dx$
\end{example}

\begin{example}[][Exam: 29.1.12]
	设 $D$ 是由曲线 $xy+1=0$ 与直线 $y+x=0$ 及 $y=2$ 围成的有界区域,求 $D$ 的面积
\end{example}
\myspace{1}

\textcolor{orange}{March 7}

\begin{example}[][Exam: 29.1.13]
	设 $D$ 是由曲线 $y=x^{2}$ 与 $y=x$ 围成的有界区域,求区域 $D$ 分别绕直线 $y=0,x=0,x=1,x=2$ 旋转所得旋转体的体积
\end{example}

\begin{example}[][Exam: 29.1.14]
	方程 $y''+4y'+4y=e^{-2x}$ 满足条件 $y(0)=1,y'(0)=0$ 的特解
\end{example}

\section{Week \Rmnum{2}}
\textcolor{blue}{March 8}

\begin{example}[][Exam: 29.2.1]
	具有特解 $y_{1}=e^{-x},y_{2}=2xe^{-x},y_{3}=3e^{x}$ 的三阶常系数线性齐次方程为:
\begin{itemize}
	\item A. $y'''-y''-y'+y=0$
	\item B. $y'''+y''-y'-y=0$
	\item C. $y'''-6y''+11y'-6y=0$
	\item D. $y'''-2y''-y'+2y=0$
\end{itemize}
\end{example}

\begin{example}[][Exam: 29.2.2]
	方程 $y''-2y'=xe^{2x}$ 的特解形式为:
\begin{itemize}
	\item A. $y=axe^{2x}$
	\item B. $y=(ax+b)e^{2x}$
	\item C. $y=x(ax+b)e^{2x}$
	\item D. $y=x^{2}(ax+b)e^{2x}$
\end{itemize}
\end{example}
\myspace{1}

\textcolor{blue}{March 9}

\begin{example}[][Exam: 29.2.3]
	方程 $y''+y=e^{x}+1+\sin x$ 的特解形式为:
\begin{itemize}
	\item A. $ae^{x}+b+c\sin x$
	\item B. $ae^{x}+b+c\cos x+d\sin x$
	\item C. $ae^{x}+b+x(c\cos x+d\sin x)$
	\item D. $y=ae^{x}+b+cx\sin x$
\end{itemize}
\end{example}

\begin{example}[][Exam: 29.2.4]
	设函数 $f(x)$ 具有一阶连续导数, 且满足 $f(x)=\int_{0}^{x}(x^{2}-t^{2})f'(t)dt+x^{2}$, 求 $f(x)$ 表达式
\end{example}
\myspace{1}

\textcolor{blue}{March 10}

\begin{example}[][Exam: 29.2.5]
	设 $L$ 是一条平面曲线, 其上任意一点 $P(x,y)(x>0)$ 到坐标原点的距离恒等于该点处切线在 $y$ 轴上的截距, 且 $L$ 经过 $(\dfrac{1}{2},0)$, 求曲线 $L$ 的渐近线方程为
\end{example}

\begin{example}[][Exam: 29.2.6]
	二元函数 $f(x,y)=
\begin{cases}
	\dfrac{x^{2}y}{x^{2}+y^{2}},&(x,y)\neq (0,0)\\
	0,&(x,y)=(0,0)
\end{cases}$ 在点 $(0,0)$ 处:
\begin{itemize}
	\item A. 不连续
	\item B. 两个偏导数都不存在
	\item C. 偏导数存在但不可微
	\item D. 可微
\end{itemize}
\end{example}
\myspace{1}

\textcolor{blue}{March 11}

\begin{example}[][Exam: 29.2.7]
	二元函数 $f(x)$ 在点 $(x_{0},y_{0})$ 处两个偏导数 $f'_{x}(x_{0},y_{0}),f'_{y}(x_{0},y_{0})$ 存在, 是 $f(x)$ 在点 $(x_{0},y_{0})$ 处连续的:
\begin{itemize}
	\item A. 充分不必要条件
	\item B. 必要不充分条件
	\item C. 充分必要条件
	\item D. 不充分不必要条件
\end{itemize}
\end{example}

\begin{example}[][Exam: 29.2.8]
	已知 $f(x,y)=\sin\sqrt{x^{4}+y^{4}}$
\begin{itemize}
	\item A. $f'_{x}(0,0),f'_{y}(0,0)$ 都存在
	\item B. $f'_{x}(0,0)$ 不存在, $f'_{y}(0,0)$ 存在
	\item C. $f'_{x}(0,0)$ 存在, $f'_{y}(0,0)$ 不存在
	\item D. $f'_{x}(0,0),f'_{y}(0,0)$ 都不存在
\end{itemize}
\end{example}
\myspace{1}

\textcolor{blue}{March 12}

\begin{example}[][Exam: 29.2.9]
	设 $f(x,y)=\dfrac{2x+y^{2}}{1+y^{2}\sqrt{1+x^{2}+y^{2}}}$, 求 $d f(0,0)$
\end{example}

\begin{example}[][Exam: 29.2.10]
	已知 $dF(x,y)=xye^{x}dx+(f(x)+y^{2})dy$, 且 $f(x)$ 有连续一阶导数, $f(x)=0$, 求 $F(x,y)$
\end{example}
\myspace{1}

\textcolor{blue}{March 13}

\begin{example}[][Exam: 29.2.11]
	设函数 $f(x,y)$ 可微, 且对于任意 $x,y$ 都有 $\dfrac{\partial f(x,y)}{\partial x}>0,\dfrac{\partial f(x,y)}{\partial y}<0$, 则下列结论正确的是:
\begin{itemize}
	\item A. $f(1,1)>f(0,0)$
	\item B. $f(-1,1)>f(0,0)$
	\item C. $f(-1,-1)>f(0,0)$
	\item D. $f(1,-1)>f(0,0)$
\end{itemize}
\end{example}

\begin{example}[][Exam: 29.2.12]
	设 $z=(x+e^{y})^{x}$, 求 $\dfrac{\partial z}{\partial x}\big|_{(1,0)}$
\end{example}
\myspace{1}

\textcolor{blue}{March 14}

\begin{example}[][Exam: 29.2.13]
	设函数 $z=z(x,y)$ 由方程 $(x+1)z+y\ln z-\arctan(2xy)=1$ 确定, 
	求$\dfrac{\partial z}{\partial x}\big|_{(0,2)}$
\end{example}

\begin{example}[][Exam: 29.2.14]
	设 $f(x,y,z)=e^{x}+y^{2}z$, 其中 $z=z(x,y)$ 是由方程 $x+y+z+xyz=0$ 所确定的隐函数, 求 $f'_{x}(0,1,-1)$
\end{example}
\section{Week \Rmnum{3}}
\textcolor{cyan}{March 15}

\begin{example}[][Exam: 29.3.1]
	设 $z=xyf(\dfrac{y}{x})$, 其中 $f(u)$ 可导, 求 $xz'_{x}+yz'_{y}$
\end{example}

\begin{example}[][Exam: 29.3.2]
	设 $z=e^{xy}+f(x+y,xy)$, 求 $\dfrac{\partial^{2} z}{\partial x\partial y}$, 其中 $f(u,v)$ 有二阶连续偏导数
\end{example}
\myspace{1}

\textcolor{cyan}{March 16}

\begin{example}[][Exam: 29.3.3]
	已知 $z=f(x,y)$ 在 $(x_{0},y_{0})$ 处取得极小值
\begin{itemize}
	\item A. $f'_{x}(x_{0},y_{0})=f'_{y}(x_{0},y_{0})=0$
	\item B. $f''_{xx}(x_{0},y_{0})f''_{yy}(x_{0},y_{0})-(f''_{xy}(x_{0},y_{0}))^{2}>0$, 且 $f''_{xx}(x_{0},y_{0})>0$
	\item C. $f(x_{0},y)$ 在 $y_{0}$ 处取得极大值
	\item D. $f(x,y_{0})$ 在 $x_{0}$ 处取得极小值
\end{itemize}
\end{example}

\begin{example}[][Exam: 29.3.4]
	设函数 $f(x),g(x)$ 均有二阶连续导数,满足 $f(0)>0,g(0)<0$, 且 $f'(0)=g'(0)=0$, 
	则函数 $z=f(x)g(y)$ 在点 $(0,0)$ 处取得极小值的一个充分条件
\begin{itemize}
	\item A. $f''(0)<0,g''(0)>0$
	\item B. $f''(0)<0,g''(0)<0$
	\item C. $f''(0)>0,g''(0)>0$
	\item D. $f''(0)>0,g''(0)<0$
\end{itemize}
\end{example}
\myspace{1}

\textcolor{cyan}{March 17}

\begin{example}[][Exam: 29.3.5]
	已知函数 $z=f(x,y)$ 的全微分 $dz=(ay-x^{2})dx+(ax-y^{2})dy,(a>0)$, 函数 $f(x,y)$
\begin{itemize}
	\item A. 无极值点
	\item B. 点 $(a,a)$ 为极小值点
	\item C. 点 $(a,a)$ 为极大值点
	\item D. 是否有极值点与 $a$ 的取值有关
\end{itemize}
\end{example}

\begin{example}[][Exam: 29.3.6]
	设函数 $z=f(xy,yg(x))$, 其中 $f$ 函数具有二阶连续偏导数, 函数 $g(x)$ 可导且在 $x=1$ 处取得极值 $g(1)=1$, 求 $\dfrac{\partial^{2}z}{\partial x\partial y}|_{(1,1)}$
\end{example}
\myspace{1}

\textcolor{cyan}{March 18}

\begin{example}[][Exam: 29.3.7]
	求函数 $f(x,y)=xe^{-\frac{x^{2}+y^{2}}{2}}$ 的极值
\end{example}

\begin{example}[][Exam: 29.3.8]
	求 $f(x,y)=x^{2}-y^{2}+2$ 在椭圆域 $D=\left\lbrace (x,y)|x^{2}+\dfrac{y^{2}}{4}\leq 1\right\rbrace$ 上的最大值和最小值
\end{example}
\myspace{1}

\textcolor{cyan}{March 19}

\begin{example}[][Exam: 29.3.9]
	交换二次积分的积分次序($a>0$)

(1). $\int_{0}^{2}dx\int_{\frac{x^{2}}{4}}^{3-x}f(x,y)dy$

(2). $\int_{0}^{a}dy\int_{0}^{\sqrt{ay}}f(x,y)dx+\int_{a}^{2a}dy\int_{0}^{2a-y}dx$
\end{example}

\begin{example}[][Exam: 29.3.10]
	设 $f(x)$ 是连续函数, 则 $\int_{0}^{1}dy\int_{-\sqrt{1-y^{2}}}^{1-y}f(x,y)dx$ 等价于:
\begin{itemize}
	\item A. $\int_{0}^{1}dx\int_{0}^{x-1}f(x,y)dy+\int_{-1}^{0}dx\int_{0}^{\sqrt{1-x^{2}}}f(x,y)dy$
	\item B. $\int_{0}^{1}dx\int_{0}^{1-x}f(x,y)dy+\int_{-1}^{0}dx\int_{-\sqrt{1-x^{2}}}^{0}f(x,y)dy$
	\item C. $\int_{0}^{\frac{\pi}{2}}d\theta\int_{0}^{\frac{1}{\cos\theta+\sin\theta}}f(r\cos\theta,r\sin\theta)dr-\int_{\frac{\pi}{2}}^{\pi}d\theta\int_{0}^{1}f(r\cos\theta,r\sin\theta)dr$
	\item D. $\int_{0}^{\frac{\pi}{2}}d\theta\int_{0}^{\frac{1}{\cos\theta+\sin\theta}}f(r\cos\theta,r\sin\theta)rdr-\int_{\frac{\pi}{2}}^{\pi}d\theta\int_{0}^{1}f(r\cos\theta,r\sin\theta)rdr$
\end{itemize}
\end{example}
\myspace{1}

\textcolor{cyan}{March 20}

\begin{example}[][Exam: 29.3.11]
	设函数 $f(t)$ 连续,则二次积分 $\int_{0}^{\frac{\pi}{2}}d\theta\int_{2\cos\theta}^{2}f(r^{2})rdr$
\begin{itemize}
	\item A. $\int_{0}^{2}dx\int_{\sqrt{2x-x^{2}}}^{\sqrt{4-x^{2}}}\sqrt{x^{2}+y^{2}}f(x^{2}+y^{2})dy$
	\item B. $\int_{0}^{2}dx\int_{\sqrt{2x-x^{2}}}^{\sqrt{4-x^{2}}}f(x^{2}+y^{2})dy$
	\item C. $\int_{0}^{2}dy\int_{1+\sqrt{1-y^{2}}}^{\sqrt{4-y^{2}}}\sqrt{x^{2}+y^{2}}f(x^{2}+y^{2})dx$
	\item D. $\int_{0}^{2}dy\int_{1+\sqrt{1-y^{2}}}^{\sqrt{4-y^{2}}}f(x^{2}+y^{2})dx$
\end{itemize}
\end{example}

\begin{example}[][Exam: 29.3.12]
	计算二重积分

(1). $\iint\limits_{x^{2}+y^{2}\leq 1}(2x+3y)^{2}d\sigma$

(2). $\int_{\frac{1}{4}}^{\frac{1}{2}}dy\int_{\frac{1}{2}}^{\sqrt{y}}e^{\frac{y}{x}}dx+\int_{\frac{1}{2}}^{1}dy\int_{y}^{\sqrt{y}}e^{\frac{y}{x}}dx$

(3). $\int_{0}^{1}dy\int_{y}^{1}\sqrt{x^{2}-y^{2}}dx$

(4). $\int_{0}^{1}dy\int_{y}^{1}\left(\frac{e^{x^{2}}}{x}-e^{y^{3}}\right)dx$

(5). $\iint\limits_{D}\left(xy^{5}-1\right)dxdy,D=\left\{(x,y)|-\frac{\pi}{2}\leq x \leq \frac{\pi}{2},\sin x\leq y \leq 1 \right\}$

(6). $\iint\limits_{D}x^{2}ydxdy$, 其中 $D$ 是由双曲线 $x^{2}-y^{2}=1$以及直线 $y=0,y=1$ 所围成的平面区域

(7). $\iint\limits_{D}\sqrt{x^{2}+y^{2}}dxdy,D=\left\{(x,y)|0\leq y \leq x, x^{2}+y^{2}\leq 2x\right\}$

(8). $\iint\limits_{D}r^{2}\sin\theta\sqrt{1-r^{2}\cos 2\theta}drd\theta,D=\left\{(r,\theta)|0\leq r\leq \sec\theta,0\leq \theta \leq \frac{\pi}{4}\right\}$
\end{example}
\myspace{1}

\textcolor{cyan}{March 21}

\begin{lemma}[估值定理]\label{lem: 估值定理}
	
	$$\lim\limits_{n\rightarrow +\infty}\int_{0}^{1}x^nf(x)dx$$
	
	其中$f(x)$在$[0,1]$上连续可导
	

	$f(x)$在$[0,1]$上连续可导 $\Rightarrow |f(x)|\leq M$, $M$ 是 $|f(x)|$ 在$[0,1]$上的最大值

	$$0\leq\int_{0}^{1}x^nf(x)dx\leq\int_{0}^{1}Mx^ndx = \dfrac{M}{n+1}$$
	
	夹逼准则:
	$$\begin{cases}
	  \lim\limits_{n \to +\infty} 0 = 0\\
	  \lim\limits_{n \to +\infty}\dfrac{M}{n+1} = 0
	\end{cases}\Rightarrow \lim\limits_{n \to +\infty}\int_{0}^{1}x^nf(x)dx=0$$ 
	
	$$\begin{cases}
	  \sin x < x\\
	  \ln(1+x) < x
	\end{cases}\Rightarrow
	\begin{cases}
	  \lim\limits_{n \to +\infty}\int_{0}^{1}\sin ^{n}xf(x)dx=0 \\
	  \lim\limits_{n \to +\infty}\int_{0}^{1}\ln^{n}(1+x)f(x)dx=0 
	\end{cases}$$

	$$\lim\limits_{n\rightarrow +\infty}\int_{0}^{1}nx^nf(x)dx$$
	 其中 $f(x)$ 在 $[0,1]$ 上连续可导

	\begin{eqnarray*}
		I & = & \int_{0}^{1}(n+1)x^{n}f(x)dx - \int_{0}^{1}x^nf(x)dx\\
		  & = & \int_{0}^{1}f(x)d(x^{n+1}) - \int_{0}^{1}x^nf(x)dx\\
		  & = & f(x)x^{n+1}\big|_{x=0}^{x=1} - \int_{0}^{1}x^{n+1}f'(x)dx\\
		  & = & f(1) - \int_{0}^{1}x^{n+1}f'(x)dx
	\end{eqnarray*}
	$$\lim\limits_{n \to +\infty}I = f(1) - \lim\limits_{n \to oo}\int_{0}^{1}x^{n+1}f'(x)dx = f(1)$$
\end{lemma}

\begin{example}[][Exam: 29.3.13]
	$\lim\limits_{n\rightarrow +\infty}\int_{0}^{1}\dfrac{nx^n}{1+e^x}dx$
\end{example}

\begin{solution}
	
	令: $f(x)=\dfrac{1}{1+e^x}$
	
	$\mathbf{lem: }$\ref{lem: 估值定理}: 
	$$\lim\limits_{n\rightarrow +\infty}\int_{0}^{1}\dfrac{nx^n}{1+e^x}dx = \lim\limits_{n\rightarrow +\infty}\int_{0}^{1}nx^{n}f(x)dx = 
	 f(1) = \dfrac{1}{1+e}$$
\end{solution}

\begin{example}[][Exam: 29.3.14]
	$\lim\limits_{x\rightarrow 0}\left[ \dfrac{1}{\ln(x+\sqrt{1+x^2})}-\dfrac{1}{\ln(1+x)}\right] $
\end{example}
\begin{solution}
	
	$$\begin{cases}
		x\to 0 & \sqrt{1+x^{2}}-1\sim \dfrac{1}{2}x^{2} \\
		x\to 0 & \ln(x+\sqrt{1+x^2}) \sim x
	\end{cases}$$

	
	$$f(x)=\ln x  \qquad f'(x) = \dfrac{1}{x}$$
	
	\begin{eqnarray*}
		I & = & -\dfrac{f(x+\sqrt{1+x^2})-f(1+x)}{x^{2}}\\
		  & = & -\dfrac{1+\sqrt{1+x^{2}}-x}{x^{2}}f'(\xi)\\
		  & = & -\dfrac{1}{2}f'(\xi)
	\end{eqnarray*} 
	$$\begin{cases}
	  \lim\limits_{x \to 0}\dfrac{1}{1+x} = 1\\
	  \lim\limits_{x \to 0}\dfrac{1}{x+\sqrt{1+x^{2}}} = \lim\limits_{x \to 0}(\sqrt{1+x^{2}}-x) = 1
	\end{cases}$$
	夹逼定理:

	$$\lim\limits_{x \to 0}f'(\xi) = 1$$
	
	综上, $\lim\limits_{x\rightarrow 0}\left[ \dfrac{1}{\ln(x+\sqrt{1+x^2})}-\dfrac{1}{\ln(1+x)}\right] =-\dfrac{1}{2}$
	
\end{solution}
\myspace{1}

\section{Week \Rmnum{4}}

\textcolor{purplea}{March 22}

\begin{example}[][Exam: 29.4.1]
	$\int\dfrac{\sin x}{\sqrt{2+\sin 2x}}dx$
\end{example}

\begin{solution}
	
	$$\begin{cases}
	  I = A(\sin x +\cos x)\\
	  J = B(\cos x -\sin x)\\
	  I + J = \sin x
	\end{cases}\Rightarrow 
	\begin{cases}
		A = \dfrac{1}{2}\\
		B = -\dfrac{1}{2}
	\end{cases}$$
	
	\begin{eqnarray*}
		I & = & \int\dfrac{I+J}{\sqrt{1+(\sin x+\cos x)^2}}dx\\
		  & = & \dfrac{1}{2}\int\dfrac{\sin x+\cos x}{\sqrt{1+(\sin x+\cos x)^2}}dx - \dfrac{1}{2}\int\dfrac{\cos x-\sin x}{\sqrt{1+(\sin x+\cos x)^2}}dx\\
		  & = & \dfrac{1}{2}\int\dfrac{\sin x+\cos x}{\sqrt{3-(\sin x-\cos x)^2}}dx - \dfrac{1}{2}\int\dfrac{\cos x-\sin x}{\sqrt{1+(\sin x+\cos x)^2}}dx\\
		  & = & \dfrac{1}{2}\left[ \arcsin \frac{\sin x-\cos x}{\sqrt{3}}-\ln\left(\sin x+\cos x+\sqrt{1+(\sin x+\cos x)^2} \right)\right] +C
	\end{eqnarray*}
\end{solution}

\begin{example}[][Exam: 29.4.2]
	$\int_{\frac{1}{6}}^{+\infty}\dfrac{1}{x}\left[ \dfrac{1}{\sqrt{x}}\right]dx$
\end{example}

\begin{solution}
	
	$$\begin{cases}
		\left[ \dfrac{1}{\sqrt{x}}\right]=0, &x\geq 1\\
		\left[ \dfrac{1}{\sqrt{x}}\right]=1, &\dfrac{1}{4}\leq x\leq 1\\
		\left[ \dfrac{1}{\sqrt{x}}\right]=2, &\dfrac{1}{6}\leq x\leq \dfrac{1}{4}
	\end{cases}$$
	
	$$\int_{\frac{1}{6}}^{+\infty}\dfrac{1}{x}\left[ \dfrac{1}{\sqrt{x}}\right]dx = 
	\int_{\frac{1}{6}}^{\frac{1}{4}}\dfrac{2}{x}dx+\int_{\frac{1}{4}}^{1}\dfrac{1}{x}dx=2\ln 3$$
\end{solution}
\myspace{1}

\textcolor{purplea}{March 23}

\begin{example}[][Exam: 29.4.3]
	已知函数 $f(x,y)=
\begin{cases}
	\dfrac{\sin x^2\cos y^2}{\sqrt{x^2+y^2}} & (x,y)\neq (0,0)\\
	0  & (x,y)=(0,0)
\end{cases}$ 求 $f'_{x}(0,0),f'_{y}(0,0)$
\end{example}

\begin{solution}
	
	$$\begin{cases}
	  f'_{x}(0,0) = \lim\limits_{\Delta x \to 0} \dfrac{f(x+\Delta x,0)-f(x,0)}{\Delta x} = 
	  \lim\limits_{\Delta x \to 0} \dfrac{\sin (\Delta x)^2}{\big|\Delta x\big|\Delta x}\\
	  f'_{y}(0,0) = \lim\limits_{\Delta y \to 0} \dfrac{f(0,y+\Delta y)-f(0,y)}{\Delta y} = 0
	\end{cases}$$
	
	综上, $f'_{x}(0,0)$ 不存在, $f'_{y}(0,0)=0$
\end{solution}

\begin{example}[][Exam: 29.4.4]
	设 $z=e^{xy}+f(x+y,xy)$, 求 $\dfrac{\partial^2 z}{\partial x\partial y}$,其中 $f(u,v)$ 有二阶连续偏导数
\end{example}

\begin{solution}
	
	$$\begin{cases}
	  \dfrac{\partial u}{\partial x} = 1\\
	  \dfrac{\partial u}{\partial y} = 1\\
	  \dfrac{\partial v}{\partial x} = y\\
	  \dfrac{\partial v}{\partial y} = x
	\end{cases}\Rightarrow \dfrac{\partial z}{\partial x} = ye^{xy} + f_{1}'(x+y,xy) + yf_{2}'(x+y,xy)$$
	
	\begin{eqnarray*}
		\dfrac{\partial^{2} z}{\partial x\partial y} & = & e^{xy} + xye^{xy} \\
		& + & f''_{11}(x+y,xy) + xf''_{12}(x+y,xy)\\
		& + & f'_{2}(x+y,xy) + y\left[f''_{21}(x+y,xy) + xf''_{22}(x+y,xy)\right]\\
		& = & (1+xy)e^{xy} + f'_{2}(x+y,xy) + f''_{11}(x+y,xy) + xyf''_{22}(x+y.xy) + (x+y)f''_{12}(x+y,xy)
	\end{eqnarray*}
\end{solution}
\myspace{1}

\textcolor{purplea}{March 24}

\begin{example}[][Exam: 29.4.5]
	$\lim\limits_{n\rightarrow +\infty}\dfrac{1}{n^4}\prod\limits_{k=1}^{2n}(n^2+k^2)^{\frac{1}{n}}$
\end{example}

\begin{solution}
	
	\begin{eqnarray*}
		I & = & \lim\limits_{n \to +\infty}e^{\frac{1}{n}\ln(\prod\limits_{k=1}^{2n}(n^2+k^2)) - 4\ln n}\\
		  & = & \lim\limits_{n \to +\infty}e^{\frac{1}{n}\ln n^{4n}\left[\prod\limits_{k=1}^{2n}\left(1+(\frac{k}{n})^{2}\right)\right] - 4\ln n}\\
		  & = & \lim\limits_{n \to +\infty}e^{\frac{1}{n}\ln\left[\prod\limits_{k=1}^{2n}\left(1+(\frac{k}{n})^{2}\right)\right]}\\
		  & = & \lim\limits_{n \to +\infty}e^{\frac{1}{n}\sum\limits_{k=1}^{2n}\ln\left[1+(\frac{k}{n})^{2}\right]}\\
		  & = & e^{\int_{0}^{2}\ln(1+x^{2})dx}
	\end{eqnarray*}
	
	\begin{eqnarray*}
		\int_{0}^{2}\ln(1+x^{2})dx & = & x\ln(1+x^{2})\big|_{x=0}^{x=2} - \int_{0}^{2}(2-\dfrac{2}{1+x^{2}})dx\\
		& = & 2\ln 5 - 4 +2\arctan 2\\
		I & = & 25e^{2\arctan 2-4}
	\end{eqnarray*}
\end{solution}

\begin{example}[][Exam: 29.4.6]
	设连续函数 $z=f(x,y)$ 满足 $\lim\limits_{\substack{x\to 0\\y \to 1}}\dfrac{f(x,y)-2x+y-2}{\sqrt{x^2+(y-1)^2}}=0$, 求 $dz\big|_{(0,1)}$
\end{example}

\begin{solution}
	
	$$\begin{cases}
	  f(0,1) = \lim\limits_{\substack{x\to 0\\y\to 1}}f(x,y) \\
	  \lim\limits_{\substack{x\to 0\\y\to 1}}f(x,y)-2x+y-2 = 0
	\end{cases}\Rightarrow f(0,1) = 1$$

	函数 $z = f(x,y)$ 在 $(0,1)$ 邻域内有定义, $z = f(x,y)$ 在 $(0,1)$ 处全增量

	$$\Delta z = f(x,y) - f(0,1) = 2(x-0) -(y-1) + o(\sqrt{x^{2}+(y-1)^{2}})$$

	$z = f(x,y)$ 在 $(0,1)$ 处可微, 且 $dz\big|_{(0,1)} = 2dx -dy$
\end{solution}
\myspace{1}

\textcolor{purplea}{March 25}

\begin{lemma}[特殊反常积分]
	
	$$\int_{1}^{+\infty}\dfrac{1}{x^p}dx 
	\begin{cases}
		p > 1 & \text{收敛}\\
		p\leq 1 & \text{发散}
	\end{cases}$$

	$$\int_{0}^{1}\dfrac{1}{x^p}dx
	\begin{cases}
		0 < p < 1 & \text{收敛}\\
		p\geq 1  & \text{发散}
	\end{cases}$$

	$$\int_{0}^{1}\dfrac{\ln x}{x^p}dx
	\begin{cases}
		0 < p < 1 & \text{收敛}\\
		p\geq 1 & \text{发散}
	\end{cases}$$
\end{lemma}
\begin{example}[][Exam: 29.4.7]
	反常积分 $\int_{0}^{1}x^{a}(1-x)^{b}\ln xdx$ 收敛, 求 $a,b$ 取值范围
\end{example}

\begin{solution}
	
	可能的瑕点为 $x=0$ 和  $x=1$
	$$\begin{cases}
	  x\to 1^{-} & f(x) = x^{a}(1-x)^{b}\ln x \sim -(1-x)^{b+1}\Rightarrow 
	  \begin{cases}
		0 < -(b+1) < 1 & \text{瑕点且收敛}\\
		-(b+1) \geq 1 & \text{瑕点且发散}\\
		-(b+1) \leq 0 & \text{非瑕点且收敛}	
	  \end{cases} \\
	  x\to 0^{+} & f(x) = x^{a}(1-x)^{b}\ln x \sim x^{a}\ln x\Rightarrow
	  \begin{cases}
		a > 0 & \text{非瑕点且收敛}\\
		
	  \end{cases}
	\end{cases}$$
	
	(1). $x=1,x=0$均为瑕点,我们得到: 
	$$\left\lbrace 
	\begin{array}{l}
		-2<b<-1\\
		-1<a<0
	\end{array}
	\right. $$
	
	(2). $x=1$是瑕点,$x=0$不是瑕点,我们有: 
	$$\left\lbrace 
	\begin{array}{l}
		a>0\\-2<b<-1
	\end{array}
	\right. $$
	
	(3). $x=0$是瑕点,$x=1$不是瑕点,我们有: 
	$$\left\lbrace 
	\begin{array}{l}
		-1<a<0\\b>-1
	\end{array}
	\right. $$
\end{solution}

\begin{example}[][Exam: 29.4.8]
	$\int\dfrac{x^2}{(x\cos x-\sin x)(x\sin x+\cos x)}dx$
\end{example}

\begin{solution}
	
	$$\int\frac{x(x\cos x-x\sin x)+x\sin x(x\sin x+\cos x)}{(x\cos x-\sin x)(x\sin x+\cos x)}dx=\int\frac{x\cos x}{x\sin x+\cos x}dx+\int\frac{x\sin x}{x\cos x-\sin x}dx$$
	$$\int\frac{x\cos x}{x\sin x+\cos x}dx=\ln(x\sin x+\cos x)+C$$
	$$\int\frac{x\sin x}{x\cos x-\sin x}dx=\ln(x\cos x+\sin x)+C$$
	
	原不定积分为: 
	$$\int\frac{x^2}{(x\cos x-\sin x)(x\sin x+\cos x)}dx=\ln(x\sin x+\cos x)+\ln(x\cos x+\sin x)+C$$
\end{solution}
\myspace{1}

\textcolor{purplea}{March 26}

\begin{example}[][Exam: 29.4.9]
	$\int_{0}^{+\infty}\dfrac{e^{-x^2}}{(x^2+\frac{1}{2})^2}dx$
\end{example}

\begin{lemma}[特殊积分]
	
	$$I=\int_{0}^{+\infty}e^{-x^2}dx=\frac{\sqrt{\pi}}{2}$$
	$$I^2=\int_{0}^{+\infty}e^{-x^2}dx\int_{0}^{+\infty}e^{-y^2}dy=\iint_{D}e^{-x^2-y^2}dxdy$$
	$$I^2=\int_{0}^{\frac{\pi}{2}}d\theta\int_{0}^{+\infty}re^{-r^2}dr=\frac{\pi}{2}(-\frac{1}{2}e^{-r^2})|_{r=0}^{r=+\infty}=\frac{\pi}{4}$$
	$$I=\frac{\sqrt{\pi}}{2}$$
\end{lemma}

\begin{solution}
	$$\int_{0}^{+\infty}\frac{e^{-x^2}}{(x^2+\frac{1}{2})^2}dx=\int_{0}^{+\infty}\frac{e^{-x^2}}{2x}d(\frac{4x^2}{2x^2+1})=\frac{e^{-x^2}}{2x}\frac{4x^2}{2x^2+1}|_{x=0}^{x=+\infty}-\int_{0}^{+\infty}\frac{4x^2}{2x^2+1}\frac{e^{-x^2}(-4x^2-2)}{4x^2}dx$$
	$$\downdownarrows$$
	$$I=\frac{2xe^{-x^2}}{2x^2+1}|_{x=0}^{x=+\infty}+2\int_{0}^{+\infty}e^{-x^2}dx$$
	$$\lim\limits_{x\rightarrow +\infty}\frac{2xe^{-x^2}}{2x^2+1}=\frac{2e^{-x^2}}{2x+\frac{1}{x}}=0 \quad \lim\limits_{x\rightarrow 0}\frac{2xe^{-x^2}}{2x^2+1}=0$$
	
	因此,我们得到原定积分为: 
	$$I=2\int_{0}^{+\infty}e^{-x^2}dx=\sqrt{\pi}$$
\end{solution}
\myspace{1}

\textcolor{purplea}{March 27}

\begin{example}[][Exam: 29.4.10]
	已知级数 $\sum\limits_{n=1}^{\infty}(-1)^{n-1}a_{n}=2,\ \sum\limits_{n=1}^{\infty}a_{2n-1}=5$,求级数 $\sum\limits_{n=1}^{\infty}a_{n}$
\end{example}
\begin{solution}
	
	$$\sum\limits_{n=1}^{\infty}(-1)^{n-1}a_{n}=2,\sum\limits_{n=1}^{\infty}a_{2n-1}=5\Rightarrow \sum\limits_{n=1}^{\infty}a_{2n}=3$$
	$$\sum\limits_{n=1}^{\infty}a_{n}=\sum\limits_{n=1}^{\infty}a_{2n}+\sum\limits_{n=1}^{\infty}a_{2n-1}=8$$
\end{solution}

\begin{example}[][Exam: 29.4.11]
	$\int_{0}^{+\infty}\dfrac{\arctan x}{x(1+\ln^2 x)}dx$
\end{example}

\begin{solution}
	
	令 $t=\dfrac{1}{x},x=\dfrac{1}{t},dx=-\dfrac{1}{t^2}dt$
	
	原反常积分等价于: 
	$$\int_{0}^{+\infty}\frac{t\arctan \frac{1}{t}}{1+\ln^2 t}\frac{1}{t^2}dt=\int_{0}^{+\infty}\frac{\arctan \frac{1}{x}}{x(1+\ln^2 x)}dx$$
	
	两式相加得到: 
	$$2I=\int_{0}^{+\infty}\frac{\frac{\pi}{2}}{x(1+\ln^2 x)}dx=\frac{\pi}{2}\arctan \ln x|_{0}^{+\infty}=\frac{\pi^2}{2}$$
	$$I=\frac{\pi^2}{4}$$
\end{solution}
\myspace{1}

\textcolor{purplea}{March 28}

\begin{example}[][Exam: 29.4.12]
	$\int\dfrac{2x^4}{1+x^6}dx$
\end{example}

\begin{solution}
	
	$$\int\frac{2x^4}{1+x^6}dx=\int\frac{x^4-1}{1+x^6}dx+\int\frac{x^4+1}{1+x^6}dx=\int\frac{x^2+1}{1+x^4-x^2}dx+\int\frac{x^4-x^2+1+x^2}{1+x^6}dx$$
	$$\int\frac{x^2-1}{1+x^4-x^2}dx=\frac{1}{2\sqrt{3}}\ln|\frac{x+\frac{1}{x}-\sqrt{3}}{x+\frac{1}{x}+\sqrt{3}}|+C$$
	$$\int\frac{x^4-x^2+1+x^2}{1+x^6}dx=\arctan x+\frac{1}{3}\arctan x^3+C$$
\end{solution}

\begin{example}[][Exam: 29.4.13]
	级数 $\sum\limits_{n=1}^{+\infty}(-1)^n(1-\cos \frac{\alpha}{n}),\quad \alpha >0$ 绝对收敛
\end{example}

\begin{solution}
	
	$$n\rightarrow +\infty,1-\cos \frac{\alpha}{n}\sim \frac{\alpha^2}{2n^2}$$
	
	原级数和 $\sum\limits_{n=1}^{+\infty}(-1)^n\dfrac{\alpha^2}{2n^2}$ 同敛散性,后者绝对收敛.
\end{solution}
\myspace{1}

\textcolor{purplea}{March 29}

\begin{example}[][Exam: 29.4.14]
	设 $u_{n}=(-1)^{n}\ln(1+\dfrac{1}{\sqrt{n}})$,判断级数 $\sum\limits_{n=1}^{+\infty}u_{n}$和级数$\sum\limits_{n=1}^{+\infty}u^{2}_{n}$ 的敛散性
\end{example}

\begin{solution}
	
	比较判别法的极限形式: 
	$$\lim\limits_{n\rightarrow +\infty}\frac{\ln(1+\frac{1}{\sqrt{n}})}{\frac{1}{\sqrt{n}}}=1$$
	
	级数 $\sum\limits_{n=1}^{+\infty}\ln(1+\dfrac{1}{\sqrt{n}})$和级数 $\sum\limits_{n=1}^{+\infty}\dfrac{1}{\sqrt{n}}$同敛散性.
	
	判断交错级数 $u_{n}$ 敛散性,我们有: $\ln(1+\dfrac{1}{\sqrt{n}})$ 单调递减,且 $\lim\limits_{n\rightarrow+\infty}\ln(1+\dfrac{1}{\sqrt{n}})=0$
	
	我们有级数 $\sum\limits_{n=1}^{+\infty}u_{n}$ 收敛,级数 $\sum\limits_{n=1}^{+\infty}|u_{n}|$ 发散,级数 $\sum\limits_{n=1}^{+\infty}u_{n}$条件收敛.
	
	比较判别法的极限形式: 
	$$\lim\limits_{n\rightarrow +\infty}\frac{\ln^{2}(1+\frac{1}{\sqrt{n}})}{\frac{1}{n}}=1$$
	
	级数$\sum\limits_{n=1}^{+\infty}u^{2}_{n}$ 发散.
\end{solution}

\begin{example}[][Exam: 29.4.15]
	已知 $y^{2}(x-y)=x^2$,求 $\int\dfrac{1}{y^2}dx$
\end{example}

\begin{solution}
	
	隐函数转为参数方程
	
	令 $\dfrac{y}{x}=t$,我们有 $xt^2(1-t)=1$,我们得到参数方程: 
	$$\left\lbrace\begin{array}{l}
		x=\dfrac{1}{t^2(1-t)}\\y=\dfrac{1}{t(1-t)}
	\end{array} \right. $$
	
	原不定积分:  
	$$ \int t^2(t-1)^2\frac{t(3t-2)}{t^4(1-t)^2}dt=\int(3-\frac{2}{t})dt=3t-2\ln t+C$$
\end{solution}
\myspace{1}

\textcolor{purplea}{March 30}

\begin{example}[][Exam: 29.4.16]
	$\int_{-\frac{\pi}{2}}^{\frac{\pi}{2}}\dfrac{\cos^3 x}{1+\cos x-\sin x}dx$
\end{example}

\begin{lemma}[对称积分变换]\label{lem: 对称积分变换}
	$$\int_{-a}^{a}f(x)dx=\int_{0}^{a}(f(x)+f(-x))dx$$
\end{lemma}
\begin{solution}
	
	由 $\mathbf{lem: }$ \ref{lem: 对称积分变换} ,我们得到 : 
	$$f(x)=\frac{\cos^3 x}{1+\cos x-\sin x},f(-x)=\frac{\cos^3 x}{1+\cos x+\sin x}$$
	
	原定积分等价于: 
	$$\int_{0}^{\frac{\pi}{2}}(\frac{\cos^3 x}{1+\cos x-\sin x}+\frac{\cos^3 x}{1+\cos x+\sin x})dx=\int_{0}^{\frac{\pi}{2}}\frac{2(1+\cos x)\cos^3x}{2\cos x(1+\cos x)}dx=\int_{0}^{\frac{\pi}{2}}\cos^2 xdx=\frac{\pi}{4}$$
\end{solution}

\begin{example}[][Exam: 29.4.17]
	已知级数 $\sum\limits_{n=1}^{+\infty}(-1)^n\sqrt{n}\sin \dfrac{1}{n^{\alpha}}$ 绝对收敛,级数 $\sum\limits_{n=1}^{+\infty}\dfrac{(-1)^n}{n^{2-\alpha}}$ 条件收敛,求 $\alpha$ 取值范围
\end{example}

\begin{solution}
	
	$p$ 级数敛散性 : 
	$$\left\lbrace \begin{array}{l}
		2-\alpha>0\\2-\alpha\leq 1
	\end{array}\right. \Rightarrow 1\leq\alpha<2$$
	
	比较判别法的极限形式: 
	$$\lim\limits_{n\rightarrow+\infty}\dfrac{\sqrt{n}\sin\frac{1}{n^{\alpha}}}{\frac{1}{n^{\alpha-\frac{1}{2}}}}=1\Rightarrow \alpha-\frac{1}{2}>1\Rightarrow \alpha >\frac{3}{2}$$
	
	我们得到 $\alpha$ 取值范围 $\dfrac{3}{2}<\alpha<2$
\end{solution}
\myspace{1}

\textcolor{purplea}{March 31}

\begin{example}[][Exam: 29.4.18]
	$\sum\limits_{n=1}^{+\infty}\dfrac{1}{(2n+1)!}x^{2n+1}$
\end{example}

\begin{solution}
	
	$S(x)=\sum\limits_{n=1}^{+\infty}\dfrac{1}{(2n+1)!}x^{2n+1}$,$S'(x)=\sum\limits_{n=0}^{+\infty}\dfrac{1}{2n!}x^{2n}$.
	
	我们得到: $$S(x)+S'(x)=\sum\limits_{n=0}^{+\infty}\frac{1}{n!}x^{n}=e^{x}$$
	
	原问题转化为微分方程: $y'+y=e^x$ 的求解,且 $y(0)=0$
	
	一阶微分方程的求解公式: $$y'+p(x)y=q(x)\Rightarrow y=e^{-\int p(x)dx}(e^{\int p(x)dx}q(x)+C)$$
	
	我们得到:  $S(x)=\dfrac{1}{2}(e^{x}-e^{-x})$
\end{solution}

\begin{example}[][Exam: 29.4.19]
	求二重积分 $\iint\limits_{D}y^2dxdy$ 和 $\iint\limits_{D}(x+2y)dxdy$,其中 $D$ 是由参数方程 $\left\lbrace\begin{array}{l}
		x=a(t-\sin t)\\y=a(1-\cos t)
	\end{array} \right. ,t\in [0,2\pi]$
\end{example}

\begin{solution}
	
	积分区域是摆线,\ $x\in [0,2\pi a]$,二重积分可以化为: 
	$$\iint\limits_{D}y^2dxdy=\int_{0}^{2\pi a}dx\int_{0}^{y(x)}y^2dy=\int_{0}^{2\pi a}\frac{1}{3}y^{3}(x)dx$$
	$$\int_{0}^{2\pi a}\frac{1}{3}y^{3}(x)dx=\int_{0}^{2\pi }\frac{1}{3}a^3(1-\cos t)^3(a-a\cos t)dt=\frac{2}{3}\int_{0}^{\pi}(2\sin^2 t)^4dt$$
	
	华里士公式:  $$\frac{2}{3}\int_{0}^{\pi}(2\sin^2 t)^4dt=\frac{2}{3}\times 16\times\frac{7}{8}\times\frac{5}{6}\times\frac{3}{4}\times\frac{1}{2}\times\frac{\pi}{2}=\frac{35\pi}{12}$$
	
	二重积分$\iint\limits_{D}(x+2y)dxdy$ 可化为: 
	$$\int_{0}^{2\pi a}dx\int_{0}^{y(x)}(x+2y)dy=\int_{0}^{2\pi a}(y^{2}(x)+xy(x))dx$$
	$$I=\int_{0}^{2\pi }[a^2(1-\cos t)^2+a^2(t-\sin t)(1-\cos t)](a-a\cos t)dt$$
	$$I=a^3\int_{0}^{2\pi }(1-\cos t)^3dt+a^3\int_{0}^{2\pi }(1-\cos t)^2(t-\sin t)dt$$
	$$I_{1}=2\int_{0}^{\pi}(2\sin^2 t)^3dt=5\pi$$
	$$I_{2}=\int_{-\pi}^{\pi}(1+\cos x)^2(x+\pi+\sin x)dx=2\pi\int_{0}^{\pi}(1+\cos x)^2dx=3\pi^2$$
\end{solution}


