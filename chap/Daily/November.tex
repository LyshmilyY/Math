\chapterimage{chap37.jpg}
\chapter{November}
\section{Week \Rmnum{1}}
\textcolor{orange}{November 1}

1. 设 $f(x,y)=
\begin{cases}
	\dfrac{y^2\sin x}{x^2+y^2}, &(x,y)\neq (0,0)\\
	0, &(x,y)=(0,0)
\end{cases}$,$f(x,y)$ 在 $(0,0)$ 点处
\begin{itemize}
	\item A. 不连续
	\item B. 连续但不可导
	\item C. 可导但不可微
	\item D. 可微
\end{itemize}
\myspace{1}
\begin{solution}
	
\end{solution}
\myspace{1}

2. 求 $\displaystyle{\iint\limits_{D}\dfrac{1-x^3y^2}{(y+2\sqrt{1-x^2})^2}dxdy}$, 其中 $D:\{(x,y)|x^2+y^2\leq 1,-y\leq x\leq y\}$
\myspace{1}
\begin{solution}
	
\end{solution}
\myspace{1}

\textcolor{orange}{November 2}

1. 求 $\displaystyle{\iiint\limits_{\Omega}(mx+ly+nz)^2dv}$,其中 $\Omega=\{(x,y,x)|x^2+y^2+z^2\leq a^2(a>0)\}$
\myspace{1}
\begin{solution}
	
\end{solution}
\myspace{1}

2. 设 $f(x)$ 在 $[a,b]$ 上导函数连续,$f'(a)=f'(b)$, 证明: $\exists \xi\in(a,b),\ s.t.\ f'(\xi)=\dfrac{f(\xi)-f(a)}{\xi-a}$
\myspace{1}
\begin{solution}
	
\end{solution}
\myspace{1}

\textcolor{orange}{November 3}

1. 下列函数在$(0,0)$点可微的是:
\begin{itemize}
	\item A. $f(x,y)=\sqrt{x^2+y^2}$
	\item B. $g(x,y)=
	\begin{cases}
		\dfrac{xy}{x^2+y^2}, &(x,y)\neq (0,0)\\
		0, &(x,y)=(0,0)
	\end{cases}$
	\item C. $\varphi(x,y)=
	\begin{cases}
		\dfrac{xy}{\sqrt{x^2+y^2}}, &(x,y)\neq (0,0)\\
		0, &(x,y)=(0,0)
	\end{cases}$
	\item D. $\psi(x,y)=
	\begin{cases}
		(x^2+y^2)\sin\frac{1}{x^2+y^2}, &(x,y)\neq (0,0)\\
		0, &(x,y)=(0,0)
	\end{cases}$
\end{itemize}
\myspace{1}
\begin{solution}
	
\end{solution}
\myspace{1}

2. 已知 $(b-a)(c-a)(d-a)(c-b)(d-c)=k$, 求行列式
$$\begin{vmatrix}
	  1   &   1   &  1    &  1    \\
	a^{2} & b^{2} & c^{2} & d^{2} \\
	a^{3} & b^{3} & c^{3} & d^{3} \\
	a^{4} & b^{4} & c^{4} & d^{4} \\
\end{vmatrix}$$
\myspace{1}
\begin{solution}
	
\end{solution}
\myspace{1}

\textcolor{orange}{November 4}

1. 求 $\displaystyle{\int_{0}^{1}dx\int_{0}^{1-x}dy\int_{x+y}^{1}\dfrac{\sin z}{z}dz}$
\myspace{1}
\begin{solution}
	
\end{solution}
\myspace{1}

2. 已知函数 $f(x,y)$ 在 $(0,0)$ 点的某邻域内有定义,则 
$\begin{cases}
	\lim\limits_{x\to 0}f_{x}'(x,0)  = & f_{x}'(0,0)\\
	\lim\limits_{y\to 0}f_{y}'(0,y)  = & f_{y}'(0,0)
\end{cases}$ 是 $f(x,y)$ 在 $(0,0)$ 处可微的什么条件?
\myspace{1}
\begin{solution}
	
\end{solution}
\myspace{1}

\textcolor{orange}{November 5}

1. 设 $n$ 阶可逆矩阵 
$A=\begin{pmatrix}
	   a   &    b   & \cdots &    b   \\
	   b   &    a   & \cdots &    b   \\
	\vdots & \vdots &        & \vdots \\
	   b   &    b   & \cdots &    a
\end{pmatrix}$, 其中 $b\neq 0$, 求 $A^{-1}$
\myspace{1} 
\begin{solution}
	
\end{solution}
\myspace{1}

2. 求微分方程 $y''+(4x+e^{2y})(y')^3=0$ 的通解, 其中 $y'\neq 0$
\myspace{1}
\begin{solution}
	
\end{solution}
\myspace{1}

\textcolor{orange}{November 6}

1. 设 $f(x)$ 二阶可导, $f^{2}(x)-f^{2}(y)=f(x+y)-f(x-y)$

(1) 证明: $f''(x)f(y)=f(x)f''(y)$

(2)若 $f''(1)=f(1)=1$, 求 $f(x)$
\myspace{1}
\begin{solution}
	
\end{solution}
\myspace{1}

2. 设 $\displaystyle{z=\dfrac{x\cos(y-1)-(y-1)\cos x}{1+\sin x+\sin(y-1)}}$, 求 $\dfrac{\partial z}{\partial y}|_{(0,1)}$
\myspace{1}
\begin{solution}
	
\end{solution}
\myspace{1}

\textcolor{orange}{November 7}

1. 已知 $A$ 是正交矩阵, 则 $A^{*}=A^{T}$ 是 $|A|=1$ 的什么条件?
\myspace{1}
\begin{solution}
	
\end{solution}
\myspace{1}

2. 求 $\displaystyle{\lim\limits_{n\to +\infty}\sqrt{n}\left( 1-\sum\limits_{i=1}^{n}\dfrac{1}{n+\sqrt{i}}\right)}$
\myspace{1}
\begin{solution}
	
\end{solution}
\myspace{1}

\section{Week \Rmnum{2}}
\textcolor{blue}{November 8}

1. 已知$f(x,y)=
\begin{cases}
	xy\dfrac{x^2-y^2}{x^2+y^2}, &(x,y)\neq(0,0)\\
	0, &(x,y)=(0,0)
\end{cases}$, 求 $f_{xy}''(0,0)\cdot f_{yx}''(0,0)$
\myspace{1}
\begin{solution}
	
\end{solution}
\myspace{1}

2. 设矩阵 $A=(a_{ij})$ 满足 $a_{ij}=A_{ij},a_{11}=-1$,
求 $Ax=
\begin{pmatrix}
	1 \\
	0 \\
	0
\end{pmatrix}$ 的解
\myspace{1}
\begin{solution}
	
\end{solution}
\myspace{1}

\textcolor{blue}{November 9}

1. 求 $\displaystyle{\int_{0}^{+\infty}\dfrac{1-e^{-x}}{\sqrt{x^3}}dx}$
\myspace{1}
\begin{solution}
	
\end{solution}
\myspace{1}

2. 方程 $\displaystyle{x=e^{\sin^{n}x}(n=1,2,\cdots)}$

(1). 证明方程在 $(\dfrac{\pi}{2},e)$ 内有唯一实数根

(2). 计算 $\displaystyle{\lim\limits_{n\to +\infty}\left(\dfrac{\pi}{2}\cdot \dfrac{\sin x_{n}}{x_{n}} \right)^{\frac{1}{x_{n}-\frac{\pi}{2}}}}$
\myspace{1}
\begin{solution}
	
\end{solution}
\myspace{1}

\textcolor{blue}{November 10}

1. 若 $\dfrac{\partial^2 z}{\partial x\partial y}=1$, 
$\begin{cases}
	z = \sin y, &x=0 \\
	z = \sin x, &y=0
\end{cases}$, 求 $z(x,y)$
\myspace{1}
\begin{solution}
	
\end{solution}
\myspace{1}

2. 设矩阵
$\begin{bmatrix}
	a_{1} & b_{1} & c_{1} \\
	a_{2} & b_{2} & c_{2} \\
	a_{3} & b_{3} & c_{3}
\end{bmatrix}$ 满秩,则直线
 $l_{1}:\ \dfrac{x-a_{1}}{a_{1}-a_{2}}=\dfrac{y-b_{1}}{b_{1}-b_{2}}=\dfrac{z-c_{1}}{c_{1}-c_{2}}$
 与直线 $l_{2}:\ \dfrac{x-a_{1}}{a_{2}-a_{3}}=\dfrac{y-b_{1}}{b_{2}-b_{3}}=\dfrac{z-c_{1}}{c_{2}-c_{3}}$ 关系
\myspace{1}
\begin{solution}
	
\end{solution}
\myspace{1}

\textcolor{blue}{November 11}

1. 设 $ A=
\begin{bmatrix}
	-3 & 4 & 0 & 0 \\
	-2 & 3 & 0 & 0 \\
	 0 & 0 & 1 & 1 \\
	 0 & 0 & 0 & 1
\end{bmatrix}$, 求 $A^{n}$
\myspace{1}
\begin{solution}
	
\end{solution}
\myspace{1}

2. 求 $\displaystyle{\sum\limits_{n=1}^{+\infty}\dfrac{(-1)^{n}n^{2}}{(n+1)!}x^{n}}$ 的和函数 $S(x)$
\myspace{1}
\begin{solution}
	
\end{solution}
\myspace{1}

\textcolor{blue}{November 12}

1. 设可微函数 $f(x,y)$ 满足$\dfrac{\partial f}{\partial x}=-f(x,y)$,$f(0,\dfrac{\pi}{2})=1$,
且 $\lim\limits_{n\to +\infty}\left[\dfrac{f(0,y+\frac{1}{n})}{f(0,y)} \right]^{n}=e^{\cot y} $, 求$f(x,y)$
\myspace{1}
\begin{solution}
	
\end{solution}
\myspace{1}

2. $f(x)$ 在 $(-\infty,+\infty)$ 可微, $|f'(x)|<kf(x)(0<k<1)$

(1).$\exists \xi\in(-\infty,+\infty),\ s.t.\ \ln f(\xi)=\xi$

(2).设$a_{n}=\ln f(a_{n-1})(n=1,2,\cdots)$, 证明 $\{a_{n}\}$ 收敛
\myspace{1}
\begin{solution}
	
\end{solution}
\myspace{1}

\textcolor{blue}{November 13}

1. 设 $f(x)$ 有连续一阶导数,$(xy-yf(x))dx+(f(x)+y^2)dy=du(x,y)$, 且 $f(0)=-1$, 求 $u(x,y)$
\myspace{1}
\begin{solution}
	
\end{solution}
\myspace{1}

2. 积分计算

(1). $\displaystyle{\int_{0}^{\ln 2}dy\int_{e^{y}}^{2}\dfrac{e^{xy}}{x^{x}-1}dx}$

(2). $\displaystyle{\int_{0}^{1}dx\int_{x}^{1}ydy\int_{y}^{1}\sqrt{1+z^4}dz}$
\myspace{1}
\begin{solution}
	
\end{solution}
\myspace{1}

\textcolor{blue}{November 14}

1. $f(x)$ 在 $x=0$ 处 $n+1$ 阶可导, $f(0)=f'(0)=\cdots=f^{(n-1)}(0)=0$,$f^{(n)}(0)=a$,
求 $\lim\limits_{x\to 0}\dfrac{f(e^{x}-1)-f(x)}{x^{n+1}}$
\myspace{1}
\begin{solution}
	
\end{solution}
\myspace{1}

2. 若函数 $z=z(x,y)$ 由方程 $\displaystyle{e^{x+2y+3z}+\dfrac{xyz}{\sqrt{1+x^2+y^2+z^2}}=1}$ 确定, 求 $dz|_{(0,0)}$
\myspace{1}
\begin{solution}
	
\end{solution}
\myspace{1}

\section{Week \Rmnum{3}}
\textcolor{cyan}{November 15}

1. 设 $f(x)$ 在 $[0,+\infty)$ 上有一阶连续导数,对半空间$x\geq 0$中任意光滑闭曲面 $\varSigma$, 
我们有 $\oiint\lim\limits_{\varSigma}e^{-x}f(x)dydz+y\sqrt{e^{x}-1}f^{2}(x)dzdx=0$, 求 $f(x)$
\myspace{1}
\begin{solution}
	
\end{solution}
\myspace{1}

2. 设 $f(t)$ 在 $[t,+\infty)$ 上有连续二阶导数, 且$f(1)=0,f'(1)=1$,$z=(x^2+y^2)f(x^2+y^2)$ 
满足 $\dfrac{\partial^2 z}{\partial x^2}+\dfrac{\partial^2 z}{\partial y^2}=0$, 求 $f(x)$ 
在 $[1,+\infty)$ 上的最大值
\myspace{1}
\begin{solution}
	
\end{solution}
\myspace{1}

\textcolor{cyan}{November 16}

1. 设 $u=f(x,y,z)$,$z=z(x,y)$ 是由方程 $\varphi(x+y,z)=1$ 所确定的隐函数,
其中 $f$ 和 $\varphi$ 有二阶连续偏导数且 $\varphi_{2}\neq 0$,
求 $\dfrac{\partial u}{\partial x},du,\dfrac{\partial^2 u}{\partial x\partial y}$
\myspace{1}
\begin{solution}
	
\end{solution}
\myspace{1}

2. 设函数 $z=z(x,y)$ 的微分 $dz=(2x+12y)dx+(12x+4y)dy$ 且 $z(0,0)=0$,
求函数 $z=z(x,y)$ 在 $4x^2+y^2\leq 25$ 上的最大值
\myspace{1}
\begin{solution}
	
\end{solution}
\myspace{1}

\textcolor{cyan}{November 17}

1. 求累次积分 $\displaystyle{\int_{0}^{\frac{\pi}{4}}d\theta\int_{0}^{2\cos\theta}f(\rho\cos\theta,\rho\sin\theta)\rho d\rho}$ 的等价形式
\myspace{1}
\begin{solution}
	
\end{solution}
\myspace{1}

2. 求: $\displaystyle{\int_{0}^{\frac{\pi}{4}}d\theta\int_{0}^{\frac{1}{\cos\theta}}\rho^2d\rho+\int_{1}^{\sqrt{2}}dx\int_{0}^{\sqrt{2-x^2}}\sqrt{x^2+y^2}dy}$
\myspace{1}
\begin{solution}
	
\end{solution}
\myspace{1}

\textcolor{cyan}{November 18}

1. 求: $\displaystyle{\int_{0}^{1}dy\int_{y}^{1}\sqrt{2xy-y^2}dx}$
\myspace{1}
\begin{solution}
	
\end{solution}
\myspace{1}

2. 求: $\displaystyle{\lim\limits_{n\to +\infty}\dfrac{\sqrt[n]{2}-1}{\sqrt[n]{2n+1}}\left[\int_{1}^{\frac{1}{2n}}e^{-y^2}dy+\int_{1}^{\frac{3}{2n}}e^{-y^2}dy+\cdots+\int_{1}^{\frac{2n-1}{2n}}e^{-y^2}dy \right]}$
\myspace{1}
\begin{solution}
	
\end{solution}
\myspace{1}

\textcolor{cyan}{November 19}

1. 设 $D$ 是由 $
\begin{cases}
	0\leq x\leq 1 \\
	0\leq y\leq 1
\end{cases}$ 所确定的平面区域, 求 $\iint\limits_{D}\sqrt{x^2+y^2}dxdy$
\myspace{1}
\begin{solution}
	
\end{solution}
\myspace{1}


2. 计算二重积分: $\displaystyle{\int_{\frac{\pi}{4}}^{\frac{3\pi}{4}}d\theta\int_{0}^{2\sin\theta}\left[ \sin\theta+\cos\theta\sqrt{1+r^2\sin^{2}\theta}\right]r^2dr}$
\myspace{1}
\begin{solution}
	
\end{solution}
\myspace{1}

\textcolor{cyan}{November 20}

1. 已知平面区域 $\displaystyle{D=\{(x,y)||x|+|y|\leq \dfrac{\pi}{2}\}}$,
记 $\displaystyle {I_{1}=\iint\limits_{D}\sqrt{x^2+y^2}d\sigma}$,
$\displaystyle{I_{2}=\iint\limits_{D}\sin\sqrt{x^2+y^2}d\sigma}$,
$\displaystyle{I_{3}=\iint\limits_{D}(1-\cos\sqrt{x^2+y^2})d\sigma}$, 比较 $I_{1},I_{2},I_{3}$ 的大小
\myspace{1}
\begin{solution}
	
\end{solution}
\myspace{1}


2. 已知平面区域 $D=\{(x,y)||x|+|y|\leq 1\}$,
记 $\displaystyle{I_{1}=\iint\limits_{D}(2x^{2}+\tan(xy^2))dxdy}$,
$\displaystyle{I_{2}=\iint\limits_{D}(x^{2}y+2\tan(y^2))dxdy}$,
$\displaystyle{I_{3}=\iint\limits_{D}(|xy|+y^2)dxdy}$, 比较 $I_{1},I_{2},I_{3}$ 的大小
\myspace{1}
\begin{solution}
	
\end{solution}
\myspace{1}

\textcolor{cyan}{November 21}

1. 可微函数 $f(x)$ 满足 $f'(x)=f(x)+\int_{0}^{1}f(x)dx$, 且 $f(0)=1$, 求 $f(x)$
\myspace{1}
\begin{solution}
	
\end{solution}
\myspace{1}

2. 设 $f(x)$ 是可导函数, 且 $f(0)=0$,$g(x)=\int_{0}^{1}xf(tx)dt$ 满足方程 $f'(x)+g'(x)=x$,
则由曲线 $y=f(x),y=e^{-x}$ 及直线 $x=0,x=2$ 围成的平面图形的面积
\myspace{1}
\begin{solution}
	
\end{solution}
\myspace{1}

\section{Week \Rmnum{4}}
\textcolor{purplea}{November 22}

1. 设函数 $f(x)$二阶可导,且 $f'(x)=f(1-x),f(0)=1$,求 $f(x)$
\myspace{1}
\begin{solution}
	
\end{solution}
\myspace{1}

2. 设函数 $f(x)$ 连续, $\lim\limits_{x\to 0 }\left[ 1+f(x)\right]^{\frac{1}{x^4}}=a(a>0)$, 且

$$\forall x,h \in \mathbb{R},f(x+h)=\int_{x}^{x+h}t\left[f(t+h)+t^2 \right]dt+f(x)$$

求 $f(x)$ 表达式和常数 $a$
\myspace{1}
\begin{solution}
	
\end{solution}
\myspace{1}

\textcolor{purplea}{November 23}

1. 设 $f(x)$ 为 $[0,+\infty)$ 上的正值连续函数,
已知曲线 $y=\int_{0}^{x}f(u)du$ 和 $x$ 轴及直线 $x=t(t>0)$ 所围成区域绕 $y$ 轴旋转所得体积
与曲线 $y=f(x)$ 和两坐标轴及直线 $x=t(t>0)$ 所围区域的面积之和为 $t^2$,求曲线 $y=f(x)$ 的方程
\myspace{1}
\begin{solution}
	
\end{solution}
\myspace{1}

2. 下列级数收敛的是:
\begin{itemize}
	\item A. $\sum\limits_{n=2}^{+\infty}\dfrac{1}{\ln(n!)}$
	\item B. $\sum\limits_{n=1}^{+\infty}(2-\dfrac{1}{n})^{n}\ln(1+\dfrac{1}{n2^{n}})$
	\item C. $\sum\limits_{n=1}^{+\infty}\dfrac{(-2)^{n}n^2+e^{n}}{ne^{n}}$
	\item D. $\sum\limits_{n=2}^{+\infty}\dfrac{(-1)^{n}e^{n}}{3^{n}-2^{n}}$
\end{itemize}
\myspace{1}
\begin{solution}
	
\end{solution}
\myspace{1}

\textcolor{purplea}{November 24}

1. 下列级数条件收敛的是:
\begin{itemize}
	\item A. $\sum\limits_{n=1}^{+\infty}\ln\left( 1+\dfrac{(-1)^n}{\sqrt{n}}\right) $
	\item B. $\sum\limits_{n=1}^{+\infty}\dfrac{(-1)^n}{\sqrt{n}}\ln(1+\dfrac{1}{n})$
	\item C. $\sum\limits_{n=1}^{+\infty}\dfrac{(-1)^{n}\left[(-1)^{n}+\ln n \right] }{n}$
	\item D. $\sum\limits_{n=2}^{+\infty}\dfrac{(-1)^{n}}{n\ln n}$
\end{itemize}
\myspace{1}
\begin{solution}
	
\end{solution}
\myspace{1}

2. 讨论级数 $\sum\limits_{n=1}^{\infty}\dfrac{n}{1^{\alpha}+2^{\alpha}+\cdots+n^{\alpha}}$的敛散性
\myspace{1}
\begin{solution}
	
\end{solution}
\myspace{1}

\textcolor{purplea}{November 25}

1. 已知级数 $\sum\limits_{n=1}^{\infty}a_{n}$ 绝对收敛,
级数 $\sum\limits_{n=1}^{\infty}(b_{n+1}-b_{n})$ 条件收敛,
判断级数 $\sum\limits_{n=1}^{\infty}b_{n}a_{n}^{2}$ 的敛散性
\myspace{1}
\begin{solution}
	
\end{solution}
\myspace{1}
 
2. 已知正项级数 $\sum\limits_{n=1}^{\infty}a_{n}$ 发散, 则下列级数一定收敛的是:
\begin{itemize}
	\item A. $\sum\limits_{n=1}^{\infty}\dfrac{(-1)^{n}a_{n}}{n}$
	\item B. $\sum\limits_{n=1}^{\infty}\dfrac{a_{n}}{n}$
	\item C. $\sum\limits_{n=2}^{\infty}\dfrac{a_{n}}{a_{1}+a_{2}+\cdots+a_{n}}$
	\item D. $\sum\limits_{n=1}^{\infty}\dfrac{a_{n}}{n^{3}+a_{n}^{2}}$
\end{itemize}
\myspace{1}
\begin{solution}
	
\end{solution}
\myspace{1}

\textcolor{purplea}{November 26}

1. 已知级数 $\sum\limits_{n=1}^{\infty}a_{n}$ 收敛,下列四个级数一定收敛的个数:
\begin{itemize}
	\item A. $\sum\limits_{n=1}^{\infty}a_{n}^{2}$
	\item B. $\sum\limits_{n=1}^{\infty}\ln(1+a_{n})$
	\item C. $\sum\limits_{n=1}^{\infty}(a_{2n}-a_{2n-1})$
	\item D. $\sum\limits_{n=1}^{\infty}(a_{n+1}^{2}-a_{n}^{2})$
\end{itemize}
\myspace{1}
\begin{solution}
	
\end{solution}
\myspace{1}


2. 设 $a_{n}$ 为曲线 $y=\sin x,(0\leq x\leq n\pi)$ 与 $x$ 轴所围区域绕 $x$ 轴旋转所得到旋转体的体积,
求级数 $\sum\limits_{n=2}^{\infty}\dfrac{(-1)^{n}\pi^{2}}{2a_{n+1}}$ 的和
\myspace{1}
\begin{solution}
	
\end{solution}
\myspace{1}

\textcolor{purplea}{November 27}

1. 若 $\displaystyle{f(x)=\lim\limits_{n\to+\infty}\int_{0}^{1}\dfrac{nt^{n-1}}{1+e^{xt}}dt}$,
求 $\int_{0}^{+\infty}f(x)dx$
\myspace{1}
\begin{solution}
	
\end{solution}
\myspace{1}

2. 证明级数 $\sum\limits_{n=1}^{\infty}\dfrac{1+\frac{1}{2}+\cdots+\frac{1}{n}}{(n+1)(n+2)}$ 收敛并求其和
\myspace{1}
\begin{solution}
	
\end{solution}
\myspace{1}

\textcolor{purplea}{November 28}

1. 设 $I_{n}=\int_{0}^{\frac{\pi}{4}}\tan^{n}xdx$, 其中 $n$ 为正整数

(1). 若 $n\geq 2$,计算 $I_{n}+I_{n-2}$

(2). 设 $p$ 为实数,讨论级数 $\sum\limits_{n=1}^{+\infty}(-1)^{n}I_{n}^{p}$ 的绝对收敛性和条件收敛性
\myspace{1}
\begin{solution}
	
\end{solution}
\myspace{1}

2. 设幂级数 $\sum\limits_{n=0}^{+\infty}\dfrac{(2n)!!}{(2n+1)!!}\dfrac{x^{2n+2}}{n+1}$

(1). 求该幂级数的收敛区间以及和函数

(2). 求级数 $\sum\limits_{n=1}^{+\infty}\dfrac{n!}{(2n+1)!!}\dfrac{1}{n+1}$ 的和

(3). $2f(\dfrac{1}{\sqrt{2}})-1=\dfrac{\pi^{2}}{8}-1$
\myspace{1}
\begin{solution}
	
\end{solution}
\myspace{1}

\textcolor{purplea}{November 29}

1. 设函数 $f(x)$ 在 $(0,+\infty)$ 上连续可微,$\lambda$ 为实数,
证明: 当且仅当 $f(x)e^{\lambda x}$ 单调不减时,$f'(x)+\lambda f(x)$ 单调不减
\myspace{1}
\begin{solution}
	
\end{solution}
\myspace{1}

2. 设 $\Omega$ 为区域 $\dfrac{x^{2}}{a^{2}}+\dfrac{y^{2}}{b^{2}}+\dfrac{z^{2}}{c^{2}}\leq 1$,
求 $\displaystyle{\iiint\limits_{\Omega}(3x+2y+z)^{2}dv}$
\myspace{1}
\begin{solution}
	
\end{solution}
\myspace{1}

\textcolor{purplea}{November 30}

1. 设曲线 $C:x^{2}+y^{2}=2x$,求 $\displaystyle{\oint_{C}\dfrac{(x+y+1)^{2}}{(x-1)^{2}+y^{2}}ds}$
\myspace{1}
\begin{solution}
	
\end{solution}
\myspace{1}

2. 设曲面 $\Sigma$ 为 $x^{2}+y^{2}+z^{2}=2y$,求 $\displaystyle{\oiint\limits_{\Sigma}(x^{2}+2y^{2}+3z^{2})dS}$
\myspace{1}
\begin{solution}
	
\end{solution}
