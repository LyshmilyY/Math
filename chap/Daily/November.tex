\chapterimage{chap37.jpg}
\chapter{November}
\section{Week \Rmnum{1}}
\hl{\textbf{\textit{November 1}}}

1. 已知$\lim\limits_{x\rightarrow 0}\dfrac{e^{x}f(x)+\sin x}{x^{2}}=1$,求$\lim\limits_{x\rightarrow 0}\dfrac{f(x)+\sin x}{x^{2}}$
\myspace{1}
\begin{solution}
	
\end{solution}
\myspace{1}

\hl{\textbf{\textit{November 2}}}

1. 设$f(x)$在$x=0$的某邻域内可导,$f(0)+3f'(0)=1$,求$\lim\limits_{x\rightarrow 0}\dfrac{\int_{x}^{\ln(1+x)}f(x+t)dt+[\sin x-\ln(1+x)]f(x)}{x^{3}}$
\myspace{1}
\begin{solution}
	
\end{solution}
\myspace{1}

2. 求极限$\lim\limits_{x\rightarrow 0^{+}}\left(\dfrac{x}{(e^{x}-1)\cos\sqrt{x}}\right)^{\dfrac{1}{(1+\sin x^{2})^{\frac{1}{x}}-1}}$
\myspace{1}
\begin{solution}
	
\end{solution}
\myspace{1}

\hl{\textbf{\textit{November 3}}}

1.设$f(x)$连续,且$f(0)\neq 0$,且$\lim\limits_{x\rightarrow 0}\left[1+\int_{0}^{x}(x-t)f(t)dt \right]^{\dfrac{1}{x\int_{0}^{x}f(x-t)dt}}$
\myspace{1}
\begin{solution}
	
\end{solution}
\myspace{1}

\hl{\textbf{\textit{November 4}}}

1. 设函数$f(x)$在$(-\infty,+\infty)$上二阶导数连续,$f(1)\leq 0$,$\lim\limits_{x\rightarrow \infty}\left[ f(x)-|x|\right]=0$,证明:

(1).存在$\xi\in(1,+\infty)$,使得$f'(\xi)>1$

(2).存在$\eta\in(-\infty,+\infty)$,使得$f''(\eta)=0$
\myspace{1}
\begin{solution}
	
\end{solution}
\myspace{1}

2. 求$\lim\limits_{x\rightarrow 0}\\dfrac{\int_{0}^{2x}|t-x|\sin tdt}{|x|^{3}}$
\myspace{1}
\begin{solution}
	
\end{solution}
\myspace{1}
\hl{\textbf{\textit{November 5}}}

1. 求使得$\oint_{L}(2y^{3}-3y)dx-x^{3}dy$的值最大的平面正向边界曲线$L$
\myspace{1}
\begin{solution}
	
\end{solution}
\myspace{1}
2. $\lim\limits_{x\rightarrow 0}\dfrac{e^{(1+x)^{\frac{1}{x}}}-(1+x)^{\frac{e}{x}}}{x^{2}}$
\myspace{1}
\begin{solution}
	
\end{solution}
\myspace{1}
3. 设曲面$z=\sqrt{1-x^{2}-y^{2}}$与平面$z=x$的交线为$L$,起点为$A(0,1,0)$,终点为$B(0,-1,0)$,求$\oint_{L}(x+y-z)dx+|y|dz$
\myspace{1}
\begin{solution}
	
\end{solution}
\myspace{1}

\hl{\textbf{\textit{November 6}}}

1. 求$\lim\limits_{n\rightarrow \infty}\sum\limits_{k=1}^{n}\left(1-\dfrac{k}{n}\right)\ln(1+\dfrac{k}{n^{2}})$
\myspace{1}
\begin{solution}
	
\end{solution}
\myspace{1}


\hl{\textbf{\textit{November 7}}}

1.求$\lim\limits_{n\rightarrow +\infty}\left[\dfrac{n}{n^{2}+n+\ln1}+\dfrac{n}{n^{2}+n+\ln2}+\cdots+\dfrac{n}{n^{2}+n+\ln n} \right]^{n}$
\myspace{1}
\begin{solution}
	
\end{solution}
\myspace{1}

\section{Week \Rmnum{2}}
\hl{\textbf{\textit{November 8}}}

1. 求$\lim\limits_{n\rightarrow +\infty}\left[ \dfrac{1}{\sqrt{n^{2}+1^{2}}}+\dfrac{1}{\sqrt{n^{2}+2^{2}}}+\cdots+\dfrac{1}{\sqrt{n^{2}+n^{2}}}\right]^{n}$
\myspace{1}
\begin{solution}
	
\end{solution}
\myspace{1}

2. 以下两个矩阵,可以用同一个可逆矩阵$\mathbf{P}$相似对角化的是:
\begin{itemize}
	\item A.$\begin{pmatrix}
		1&1\\1&0
	\end{pmatrix}$,$\begin{pmatrix}
	0&1\\1&1
	\end{pmatrix}$
	\item B.$\begin{pmatrix}
		1&1\\1&-1
	\end{pmatrix}$,$\begin{pmatrix}
		-1&1\\1&1
	\end{pmatrix}$
	\item C.$\begin{pmatrix}
		0&1\\1&1
	\end{pmatrix}$,$\begin{pmatrix}
		-1&1\\1&0
	\end{pmatrix}$
	\item D.$\begin{pmatrix}
		0&1\\1&-1
	\end{pmatrix}$,$\begin{pmatrix}
		-1&1\\1&0
	\end{pmatrix}$
\end{itemize}
\myspace{1}
\begin{solution}
	
\end{solution}
\myspace{1}

3. 设函数$f(x)$在$[0,2]$上一阶可导,$f(0)=0$,$f(x)$在$x=x_{0}$处取得最大值$Mx_{0},x_{0}\in(0,2)$,且$f'(x)\leq M$,证明:

(1).当$x\in[0,x_{0}]$时,有$f(x)=Mx$

(2).$M=0$
\myspace{1}
\begin{solution}
	
\end{solution}
\myspace{1}

4.设数列$\{a_{n}\},\{b_{n}\}$对任意的正整数,满足$a_{n}<b_{n}<a_{n+1}$,则:
\begin{itemize}
	\item A.数列$\{a_{n}\},\{b_{n}\}$均收敛,且$\lim\limits_{n\rightarrow  \infty}a_{n}=\lim\limits_{n\rightarrow \infty}b_{n}$
	\item B.数列$\{a_{n}\},\{b_{n}\}$均发散,且$\lim\limits_{n\rightarrow  \infty}a_{n}=\lim\limits_{n\rightarrow \infty}b_{n}=\infty$
	\item C.数列$\{a_{n}\},\{b_{n}\}$具有相同的敛散性
	\item D.数列$\{a_{n}\},\{b_{n}\}$具有不同的敛散性
\end{itemize}
\myspace{1}
\begin{solution}
	
\end{solution}
\myspace{1}

\hl{\textbf{\textit{November 9}}}

1. 若可逆线性变换$\mathbf{x=Py}$可将二次型$f(x_{1},x_{2})=x_{1}^{2}+2x_{2}^{2}+2x_{1}x_{2}$化为规范型$y_{1}^{2}+y_{2}^{2}$,同时将二次型$g(x_{1},x_{2})=-x_{1}^{2}+2x_{2}^{2}+2x_{1}x_{2}$化为标准型$k_{1}y_{1}^{2}+k_{2}y_{2}^{2}$,求可逆矩阵$\mathbf{P}$及$k_{1},k_{2}$的值
\myspace{1}
\begin{solution}
	
\end{solution}
\myspace{1}

\hl{\textbf{\textit{November 10}}}

1.设有数列$\{x_{n}\}$,已知$\lim\limits_{n\rightarrow +\infty}(x_{n+1}-x_{n})=0$,求下列说法正确的个数:
\begin{itemize}
	\item (1). $\{x_{n}\}$必收敛
	\item (2).若$\{x_{n}\}$单调,则$\{x_{n}\}$必收敛
	\item (3).若$\{x_{n}\}$有界,则$\{x_{n}\}$必收敛
	\item (4).若$\{x_{3n}\}$收敛,则$\{x_{n}\}$必收敛
\end{itemize}
\myspace{1}
\begin{solution}
	
\end{solution}
\myspace{1}

\hl{\textbf{\textit{November 11}}}

1. 设$f(x)$有连续一阶导数,且$0<f'(x)\leq\dfrac{\ln(2+x^{2})}{2(1+x^{2})}$,数列$x_{0}=a,x_{n}=f(x_{n-1}),n=1,2,\cdots .$

证明:

(1). 极限$\lim\limits_{x\rightarrow \infty}x_{n}$存在且是方程$f(x)=x$的唯一实根 

(2). 级数$\sum\limits_{n=1}^{\infty}\left[f(x_{n})-x_{n}\right]$收敛

(3). 级数$\sum\limits_{n=1}^{\infty}\left[x_{n}-A\right]$绝对收敛,其中$\lim\limits_{x\rightarrow\infty}x_{n}=A$
\myspace{1}
\begin{solution}
	
\end{solution}
\myspace{1}

\hl{\textbf{\textit{November 12}}}

1. 设$f(x)=x+a\ln(1+x)+\dfrac{bx\sin x}{1+x^{2}},g(x)=cx^{3}$,若$f(x)$与$g(x)$在$x\to 0$时是等价无穷小,则:
\begin{itemize}
	\item A. $a=1,b=-\dfrac{1}{2},c=\dfrac{1}{3}$ 
	\item B. $a=1,b=-\dfrac{1}{2},c=-\dfrac{1}{3}$
	\item C. $a=-1,b=\dfrac{1}{2},c=-\dfrac{1}{3}$
	\item D. $a=-1,b=-\dfrac{1}{2},c=\dfrac-{1}{3}$
\end{itemize}
\myspace{1}
\begin{solution}
	
\end{solution}
\myspace{1}

\hl{\textbf{\textit{November 13}}}

1. 设$f(x)$为连续函数,$\lim\limits_{x\rightarrow 0}\dfrac{xf(x)-\ln(1+x)}{x^{2}}=2$,$F(x)=\int_{0}^{x}tf(x-t)dt$,当$x\to 0$时,$F(x)-\dfrac{1}{2}x^{2}$与$bx^{k}$为等价无穷小,其中常数$b\neq 0$,$k$为正整数,求$k,b,f(0),f'(0)$
\myspace{1}
\begin{solution}
	
\end{solution}
\myspace{1}

\hl{\textbf{\textit{November 14}}}

1. 设$f(x)=\lim\limits_{n\rightarrow\infty}\dfrac{2e^{(n+1)x}+1}{e^{nx}+x^{n}+1}$,则$f(x)$
\begin{itemize}
	\item A. 仅有一个可去间断点 
	\item B. 仅有一个跳跃间断点
	\item C. 有两个可去间断点
	\item D. 有两个跳跃间断点
\end{itemize}
\myspace{1}
\begin{solution}
	
\end{solution}
\myspace{1}

\section{Week \Rmnum{3}}
\hl{\textbf{\textit{November 15}}}

1. 下列命题成立的是:
\begin{itemize}
	\item A. 若$\lim\limits_{x\rightarrow 0}\varphi(x)=0$,且$\lim\limits_{x\rightarrow 0}\dfrac{f\left[\varphi(x)\right]-f(0)}{\varphi(x)}$,则$f(x)$在$x=0$处可导
	\item B. 若$f(x)$在$x=0$处可导,且$\lim\limits_{x\rightarrow 0}\varphi(x)=0$,则$\lim\limits_{x\rightarrow 0}\dfrac{f\left[\varphi(x)\right]-f(0)}{\varphi(x)}=f'(0)$
	\item C. 若$\lim\limits_{x\rightarrow 0}\dfrac{f(\sin x)-f(0)}{\sqrt{x^{2}}}$存在,则$f(x)$在$x=0$处可导
	\item D. 若$\lim\limits_{x\rightarrow 0}\dfrac{f(\sqrt[3]{x})-f(0)}{\sqrt{x^{2}}}$存在,则$f(x)$在$x=0$处可导
\end{itemize}
\myspace{1}
\begin{solution}
	
\end{solution}
\myspace{1}

\hl{\textbf{\textit{November 16}}}

1. 设$f(x)$在$x_{0}$点可导,$\alpha_{n},\beta_{n}$为趋于零的正项数列,求极限$\lim\limits_{n\rightarrow\infty}\dfrac{f(x_{0}+\alpha_{n})-f(x_{0}-\beta_{n})}{\alpha_{n}+\beta_{n}}$
\myspace{1}
\begin{solution}
	
\end{solution}
\myspace{1}

\hl{\textbf{\textit{November 17}}}

1. 设函数$\varphi(x)=\int_{0}^{\sin x}f(tx^{2})dt$,其中$f(x)$是连续函数,且$f(0)=2$

(1). 求$\varphi'(x)$

(2). 讨论$\varphi'(x)$的连续性
\myspace{1}
\begin{solution}
	
\end{solution}
\myspace{1}

\hl{\textbf{\textit{November 18}}}

1. 设$x=\int_{0}^{1}e^{tu^{2}}du,y=y(t)$由方程$t-\int_{1}^{y+t}e^{-u^{2}}du=0$所确定,求:

(1). $\dfrac{dy}{dt}|_{t=0},\dfrac{d^{2}y}{dt^{2}}|_{t=0},\dfrac{dx}{dt}|_{t=0},\dfrac{d^{2}x}{dt^{2}}|_{t=0}$

(2). $\dfrac{dy}{dx}|_{t=0},\dfrac{d^{2}y}{dx^{2}}|_{t=0}$
\myspace{1}
\begin{solution}
	
\end{solution}
\myspace{1}

\hl{\textbf{\textit{November 19}}}

1. 设$f(x)=\left\lbrace 
\begin{array}{l}
	\dfrac{x-\sin x}{x^{3}},x\neq 0\\
	a,x=0
\end{array}
\right. $处处连续,求$f''(0)$
\begin{itemize}
	\item A. $0$
	\item B. 不存在
	\item C. $\dfrac{1}{60}$
	\item D. $-\dfrac{a}{10}$
\end{itemize}
\myspace{1}
\begin{solution}
	
\end{solution}
\myspace{1}

\hl{\textbf{\textit{November 20}}}

1.设方程$a^{x}=bx(a>1)$有两个不同的实根,求常数$a,b$应满足的关系式
\myspace{1}
\begin{solution}
	
\end{solution}
\myspace{1}

\hl{\textbf{\textit{November 21}}}

1.设$y(x)$满足$y''+2ay'+b^{2}y=0(a>b>0),y(0)=1,y'(0)=1$,求$\int_{0}^{+\infty}y(x)dx$
\myspace{1}
\begin{solution}
	
\end{solution}
\myspace{1}

2. 设$f(x)$在$[0,+\infty]$上二阶可导,且$f(0)=0,f''(x)<0$,则当$0<a<x<b$时,下面那个选项正确:
\begin{itemize}
	\item A. $af(x)>xf(a)$
	\item B. $bf(x)>xf(b)$
	\item C. $xf(x)<bf(b)$
	\item D. $xf(x)>af(a)$
\end{itemize}
\myspace{1}
\begin{solution}
	
\end{solution}
\myspace{1}

\section{Week \Rmnum{4}}
\hl{\textbf{\textit{November 22}}}

1. 设$f(x)$二阶可导,且$f(1)=6,f'(1)=0$,且当$x\geq 1$,$x^{2}f''(x)-3xf'(x)-5f(x)\geq 0$,证明:当$x\geq 1$时,$f(x)\geq x^{5}+\dfrac{5}{x}$
\myspace{1}
\begin{solution}
	
\end{solution}
\myspace{1}

\hl{\textbf{\textit{November 23}}}

1.设$f(x)=\int_{0}^{x}t|x-t|dt-\dfrac{x^{2}}{6}$,求:

(1). 函数$f(x)$的极值和曲线$y=f(x)$的凹凸区间和拐点

(2). 曲线$y=f(x)$与$x$轴围成的区域的面积及绕$y$轴旋转所得旋转体的体积
\myspace{1}
\begin{solution}
	
\end{solution}
\myspace{1}

\hl{\textbf{\textit{November 24}}}

1. 曲线 $y=e^{\frac{1}{x}}\sqrt{1+x^{2}}$ 的渐近线有多少条?
\myspace{1}
\begin{solution}
	
\end{solution}
\myspace{1}

\hl{\textbf{\textit{November 25}}}

1. 设 $f(x)$ 在 $[-2,2]$ 上二阶可导,且 $|f(x)|\leq 1$,又 $[f(0)]^{2}+[f'(0)]^{2}=4$,证明在 $(-2,2)$ 内至少存在一点 $\xi$,使得 $f''(\xi)+f(\xi)=0$
\myspace{1}
\begin{solution}
	
\end{solution}
\myspace{1}

\hl{\textbf{\textit{November 26}}}

1. 设 $f(x)$ 在 $[a,b]$ 上有 $n+1$ 阶导数,且 $f^{(k)}(a)=f^{(k)}(b)=0,k=0,1,2,\cdots,n$,证明:存在 $\xi\in(a,b),s.t.\ f^{(n+1)}(\xi)=f(\xi)$
\myspace{1}
\begin{solution}
	
\end{solution}
\myspace{1}

\hl{\textbf{\textit{November 27}}}

1. 设 $f(x)$ 在 $[0,1]$ 上连续,在 $(0,1)$ 内可导,且 $f(0)=0,f(1)=1$,试证明:

(1). 在 $(0,1)$ 内存在 $\xi,\eta$,使得 $[1+\eta f(\eta)] f'(\xi)= f'(\eta)+f^{2}(\eta)$

(2). 存在 $\xi$ 和 $\eta$,满足 $0<\xi<\eta<1$,使得 $f'(xi)+f'(\eta)=2$
\myspace{1}
\begin{solution}
	
\end{solution}
\myspace{1}

\hl{\textbf{\textit{November 28}}}

1. 设函数 $f(x),g(x)$ 在 $[0,1]$ 上二阶可导,且 $f(1)>g(1),f(0)>g(0)$,$\int_{0}^{1}f(x)dx=\int_{0}^{1}g(x)dx$,试证明:至少存在一点 $\xi\in(0,1)$,使得 $f''(\xi)>g''(\xi)$
\myspace{1}
\begin{solution}
	
\end{solution}
\myspace{1}

\hl{\textbf{\textit{November 29}}}

1. 设 $\alpha$ 为正整数,且反常积分 $\int_{0}^{+\infty}\dfrac{\ln(1+x^{2})}{x^{\alpha}}dx$收敛,求 $\int_{0}^{+\infty}\dfrac{\ln(1+x^{2})}{x^{\alpha}}dx$
\myspace{1}
\begin{solution}
	
\end{solution}
\myspace{1}

\hl{\textbf{\textit{November 30}}}

1. 设 $f(x)$ 在 $[0,+\infty)$ 连续且单调,$f(x+2)-f(x)=4(x+2),f(0)=1$,$\int_{1}^{9}f^{-1}(x)dx=\dfrac{28}{3}$,其中 $f^{-1}(x)$ 为 $f(x)$ 的反函数,求$\int_{1}^{3}f(x)dx$
\myspace{1}
\begin{solution}
	
\end{solution}
\myspace{1}