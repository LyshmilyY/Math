\chapterimage{chap23.jpg}
\chapter{January}
\section{Week \Rmnum{1}}

\hl{\textbf{\textit{January 1}}}

1. 已知 $f(x+1)$ 的定义域为$[0,a]$,($a>0$),求 $f(x)$ 定义域
\myspace{1}
\begin{solution}
	$f(x+1)$ 的定义域为$[0,a]$,则 $f(x)$ 的定义域为$[-1,a-1]$
\end{solution}
2. 已知 $f(x)=e^{x^{2}},f[\varphi(x)]=1-x$ 且 $\varphi(x)\geq 0$,求 $\varphi(x)$ 并求出定义域
\myspace{1}

3. 设 $g(x)=\left\lbrace
\begin{aligned}
	2-x,x\leq 0\\ x+2,x>0
\end{aligned}
\right. $,$f(x)=\left\lbrace
\begin{aligned}
	 x^{2},x< 0\\ -x,x\geq 0
\end{aligned}
\right.$,求 $g[f(x)]$
\myspace{1}

4. 设函数 $f(x)=\left\lbrace
\begin{aligned}
	1-2x^{2},x<-1\\
	x^{3},-1\leq x\leq 2\\
	12x-16,x>2
\end{aligned}
\right. $,求 $f(x)$ 的反函数 $g(x)$ 的表达式
\myspace{1}

5. 证明:定义在 $[-a,a]$ 上的任意一个函数 $f(x)$ 都可以表示为一个奇函数和一个偶函数之和
\myspace{1}

6. 判断函数 $f(x)=x\tan x\cdot e^{\sin x}$的奇偶性、单调性、周期性和有界性
\myspace{1}

7. 函数 $f(x)=\dfrac{|x|\sin(x-2)}{x(x-1)(x-2)^{2}}$ 在下列哪个区间内有界

\begin{itemize}
	\item A. $(-1,0)$
	\item B. $(0,1)$
	\item C. $(1,2)$
	\item D. $(2,3)$
\end{itemize}
\myspace{1}

8. 求$\lim\limits_{n\to\infty}\left[\sqrt{1+2+\cdots+n}-\sqrt{1+2+\cdots+(n-1)}\right]$
\myspace{1}

9. 求极限 $\lim\limits_{x\to-\infty}\dfrac{\sqrt{4x^{2}+x-1}+x+1}{\sqrt{x^{2}+\sin x}}$
\myspace{1}

10. 求极限 $\lim\limits_{x\to 0}\left(\dfrac{2+e^{\frac{1}{x}}}{1+e^{\frac{4}{x}}}+\dfrac{\sin x}{|x|}\right)$
\myspace{1}

\hl{\textbf{\textit{January 2}}}

1. 求极限 $\lim\limits_{x\to+\infty}(\sqrt{x^{2}+x}-\sqrt{x^{2}-x})$
\myspace{1}

2. 求极限 $\lim\limits_{x\to 0}\dfrac{\sqrt{1+\tan x}-\sqrt{1+\sin x}}{x(1-\cos x)}$
\myspace{1}

3. 已知 $\lim\limits_{x\to 0}\dfrac{e^{x^{2}}-\cos 2x}{ax^{b}}=1$,求 $a,b$
\myspace{1}

4. 已知 $\lim\limits_{x\to x_{0}}\varphi(x)=0$,下列结论正确的个数为
\begin{itemize}
	\item A. $\lim\limits_{x\to x_{0}}\dfrac{\sin\varphi(x)}{\varphi(x)}=1$
	\item B. $\lim\limits_{x\to x_{0}}[1+\varphi(x)]^{\frac{1}{\varphi}}=e$
	\item C. 当$x\to x_{0}$时,$\sin \varphi(x)\sim \varphi(x)$
	\item D. 若$\lim\limits_{u\to 0}f(u)=A$,则 $\lim\limits_{x\to x_{0}}f[\varphi(x)]=A$
\end{itemize}
\myspace{1}

5. 求极限 $\lim\limits_{x\to 0}\dfrac{\arcsin x-\arctan x}{\sin x-\tan x}$
\myspace{1}

6. 求极限 $\lim\limits_{x\to 0}\dfrac{\ln\frac{x}{\ln(1+x)}}{x}$
\myspace{1}

7. 求极限 $\lim\limits_{x\to +\infty}\dfrac{e^{x}}{(1+\frac{1}{x})^{x^{2}}}$
\myspace{1}

8. 求极限 $\lim\limits_{x\to 0}\dfrac{\cos x-\cos(\sin x)}{x^{4}}$
\myspace{1}

9. 求极限 $\lim\limits_{x\to +\infty}x^{2}[\arctan(x+1)-\arctan x]$
\myspace{1}

10. 求极限 $\lim\limits_{n\to \infty}n^{2}[\arctan\dfrac{a}{n}-\arctan \dfrac{a}{n+1}]$
\myspace{1}

\hl{\textbf{\textit{January 3}}}

1. 求极限 $\lim\limits_{x\to +\infty}[\sin\sqrt{x+1}-\sin\sqrt{x}]$
\myspace{1}

2. 求极限 $\lim\limits_{x\to 0}[\dfrac{1}{\ln(1+x^{2})}-\dfrac{1}{\ln(1+\tan^{2}x)}]$
\myspace{1}

3. 求极限 $\lim\limits_{x\to 0}[\dfrac{1}{\ln(1+x^{2})}-\dfrac{1}{\ln(1+\sin^{2}x)}]$
\myspace{1}

4. 求极限 $\lim\limits_{x\to 0}(x+2^{x})^{\frac{2}{x}}$
\myspace{1}

5. 若 $\lim\limits_{x\to 0}\left(\dfrac{1-\tan x}{1+\tan x}^{\frac{1}{\sin kx}} \right)=e$,求 $k$
\myspace{1}

6.  若 $\lim\limits_{x\to 0}(e^{x}+ax^{2}+bx)^{\frac{1}{x^{2}}}=1$,求 $a,b$
\myspace{1}

7. 求极限 $\lim\limits_{x\to 0}\left(\dfrac{\arctan x}{x}\right)^{\frac{1}{1-\cos x}}$
\myspace{1}

8. 求极限 $\lim\limits_{n\to \infty}(n\tan\frac{1}{n})^{n^{2}}$
\myspace{1}

9. 求极限 $\lim\limits_{n\to \infty}\tan^{n}\left( \dfrac{\pi}{4}+\dfrac{1}{n}\right)$
\myspace{1}

10. 求极限 $\lim\limits_{x\to 0}\left( \dfrac{\ln(1+x)}{x}\right)^{\frac{1}{e^{x}-1}}$
\myspace{1}

\hl{\textbf{\textit{January 4}}}

1. 求极限 $\lim\limits_{x\to 0}\left( \dfrac{(1+x)^{\frac{1}{x}}}{e}\right)^{\frac{1}{x}}$
\myspace{1}

2. 求极限 $\lim\limits_{x\to 0}(\cos 2x+2x\sin x)^{\frac{1}{x^{4}}}$
\myspace{1}

3. 求极限 $\lim\limits_{x\to \frac{\pi}{4}}\left( \tan x\right) ^{\frac{1}{\cos x-\sin x}}$
\myspace{1}

4. 求极限 $\lim\limits_{x\to 0}\left(\dfrac{(e^{x}+e^{2x}+\cdots +e^{nx})}{n} \right)^{\frac{1}{x}} $
\myspace{1}

5. 求极限 $\lim\limits_{x\to \infty}\left(\dfrac{x^{n}}{(x+1)(x+2)\cdots(x+n)} \right)^{x} $
\myspace{1}

6. 求极限 $\lim\limits_{x\to \infty}\left(\sin\frac{1}{x}+\cos\frac{1}{x} \right)^{x} $
\myspace{1}

7. 求极限 $\lim\limits_{n\to \infty}n\left[e\left(1+\dfrac{1}{n} \right)^{-n}-1 \right]$
\myspace{1}

8. 设 $a>0,a\neq 1$,且 $\lim\limits_{x\to +\infty}x^{p}(a^{\frac{1}{x}}-a^{\frac{1}{x+1}})=\ln a$,求 $p$
\myspace{1}

9. 求极限 $\lim\limits_{x\to 0}\dfrac{(1-\sqrt{\cos x})(1-\sqrt[3]{\cos x})\cdots(1-\sqrt[n]{\cos x})}{(1-\cos x)^{n-1}}$
\myspace{1} 

10. 求极限 $\lim\limits_{x\to 0}\dfrac{\ln(\sin^{2}x+e^{x})-x}{\ln(x^{2}+e^{2x})-2x}$
\myspace{1} 

\hl{\textbf{\textit{January 5}}}

1. 已知极限 $\lim\limits_{x\to 0}\dfrac{x-\tan x}{x^{k}}=c$,其中 $k,c$ 为常数,且 $c\neq 0$,求 $k,c$
\myspace{1}

2. 若 $\lim\limits_{x\to 0}\left(\dfrac{\sin x^{4}}{x^{4}}-\dfrac{f(x)}{x^{3}}\right)=2$,则当 $x\to 0$ 时,$f(x)$ 是 x 的:
\begin{itemize}
	\item A. 等价无穷小
	\item B. 同阶但非等价无穷小
	\item C. 高阶无穷小
	\item D. 低阶无穷小
\end{itemize}
\myspace{1}

3. 当 $x\to 0$ 时,$\alpha(x)=k^{2}$ 与 $\beta(x)=\sqrt{1+x\arcsin x}-\sqrt{\cos x}$ 时等价无穷小,求 $k$
\myspace{1}

4. 当 $x\to 0^{+}$ 时,与 $\sqrt{x}$ 等价的无穷小量为:
\begin{itemize}
	\item A. $1-e^{\sqrt{x}}$
	\item B. $\ln\dfrac{1+x}{1-\sqrt{x}}$
	\item C. $\sqrt{1+\sqrt{x}}-1$
	\item D. $1-\cos \sqrt{x}$
\end{itemize}
\myspace{1}

5. 设 $\alpha_{1}=x(\cos\sqrt{x}-1),\alpha_{2}=\sqrt{x}\ln(1+\sqrt[3]{x}),\alpha_{3}=\sqrt[3]{x+1}-1$,当 $x\to 0^{+}$时,以上 $3$ 个无穷小量从低阶到高阶的排序为:
\begin{itemize}
	\item A. $\alpha_{1},\alpha_{2},\alpha_{3}$
	\item B. $\alpha_{2},\alpha_{3},\alpha_{1}$
	\item C. $\alpha_{2},\alpha_{1},\alpha_{3}$
	\item D. $\alpha_{3},\alpha_{2},\alpha_{1}$
\end{itemize}
\myspace{1}

6. 函数 $f(x)=\dfrac{(e^{\frac{1}{x}+e})\tan x}{x(e^{\frac{1}{x}}-e)}$ 在 $[-\pi,\pi]$上的第一类间断点是:
\begin{itemize}
	\item A. $0$
	\item B. $1$
	\item C. $-\dfrac{\pi}{2}$
	\item D. $\dfrac{\pi}{2}$
\end{itemize}
\myspace{1}

7. 设函数 $f(x)=\dfrac{\ln|x}{|x-1|}\sin x$,则$f(x)$ 有
\begin{itemize}
	\item A. $1$ 个可去间断点,$1$ 个跳跃间断点
	\item B. $1$ 个可去间断点,$1$ 个无穷间断点
	\item C. $2$ 个跳跃间断点
	\item D. $2$ 个无穷间断点
\end{itemize}
\myspace{1}

8. 函数 $f(x)=\dfrac{x^{2}-x}{x^{2}-1}\sqrt{1+\dfrac{1}{x^{2}}}$ 的无穷间断点的个数为:
\begin{itemize}
	\item A. $0$
	\item B. $1$
	\item C. $2$
	\item D. $3$
\end{itemize}
\myspace{1}

9. 函数 $f(x)=\dfrac{|x|^{x}-1}{x(x+1)\ln|x|}$ 的可去间断点的个数为:
\begin{itemize}
	\item A. $0$
	\item B. $1$
	\item C. $2$
	\item D. $3$
\end{itemize}
\myspace{1}

10. 设函数 $f(x)=\lim\limits_{n\to \infty}\dfrac{1+x}{1+x^{2n}}$,讨论函数的间断点,其结论为:
\begin{itemize}
	\item A. 不存在间断点
	\item B. 存在间断点 $x=1$
	\item C. 存在间断点 $x=0$
	\item D. 存在间断点 $x=-1$
\end{itemize}
\myspace{1}

\hl{\textbf{\textit{January 6}}}

1. 设函数 $f(x)$ 在 $x=0$ 处连续,且 $\lim\limits_{h\to 0}\dfrac{f(h^{2})}{h^{2}}=1$,则:
\begin{itemize}
	\item A. $f(0)=0$ 且 $f_{-}^{'}(0)$ 存在
	\item B. $f(0)=1$ 且 $f_{-}^{'}(0)$ 存在
	\item C. $f(0)=0$ 且 $f_{+}^{'}(0)$ 存在
	\item D. $f(0)=0$ 且 $f_{+}^{'}(0)$ 存在
\end{itemize}
\myspace{1}

2. 设函数 $f(x)$ 在区间 $(-\delta,\delta)$内有定义,若当 $x\in(-\delta,\delta)$时,恒有$|f(x)|\leq x^{2}$,则$x=0$ 必是$f(x)$的:
\begin{itemize}
	\item A. 间断点
	\item B. 连续而不可导的点
	\item C. 可导的点,且 $f'(0)=0$
	\item D. 可导的点,且 $f'(0)\neq 0$
\end{itemize}
\myspace{1}

3. 设函数 $f(x)=\begin{cases}
	x^{\alpha}\cos\dfrac{1}{x^{\beta}}, & x>0\\
	0,& x\leq 0
\end{cases}(\alpha>0,\beta>0)$,若 $f'(x)$ 在 $x=0$ 处连续,则:
\begin{itemize}
	\item A. $\alpha-\beta>1$
	\item B. $0<\alpha-\beta\leq 1$
	\item C. $\alpha-\beta>2$
	\item D. $0<\alpha-\beta\leq 2$
\end{itemize}
\myspace{1}

4. 曲线 $x+y+e^{2xy}=0$ 在点 $(0,-1)$ 处的切线方程
\myspace{1}

5.
(1). 设函数 $u(x),v(x)$ 可导,利用导数定义证明:$[u(x)v(x)]'=u'(x)v(x)+u(x)v'(x)$

(2). 设函数$u_{1}(x),u_{2}(x),\cdots,u_{n}(x)$可导,$f(x)=u_{1}(x)u_{2}(x)\cdots u_{n}(x)$,写出$f(x)$ 的求导公式
\myspace{1}

6. 设函数 $f(x)=\begin{cases}
	\ln\sqrt{x},x\geq 1\\2x-1,x<1
\end{cases}$,$y=f(f(x))$,则 $\dfrac{dy}{dx}|_{x=0}$
\myspace{1}

7. 设 $y=y(x)$ 是由方程 $xy+e^{y}=x+1$确定的隐函数,则 $\dfrac{d^{2}y}{dx^{2}}|_{x=0}$
\myspace{1}

8. 设函数 $y=y(x)$ 由参数方程$\begin{cases}
	x=t-\ln(1+t)\\y=t^{3}+t^{2}
\end{cases}$所确定,则 $\dfrac{d^{2}y}{dx^{2}}$
\myspace{1}

9. 设 $y=x^{2}2^{x}$,求 $y^{(n)}$
\myspace{1}

10. 设 $y=\dfrac{1}{x^{2}-1}$,求 $y^{(n)}$
\myspace{1}

\hl{\textbf{\textit{January 7}}}

1. 已知函数 $f(x)$ 具有任意阶导数,且 $f'(x)=[f(x)]^{2}$,则当 $n$ 为大于 $2$ 的正整数时,$f(x)$ 的 $n$ 阶导数 $f^{(n)}(x)$ 是:
\begin{itemize}
	\item A. $n![f(x)]^{n+1}$
	\item B. $n[f(x)]^{n+1}$
	\item C. $[f(x)^{2n}]$
	\item D. $n![f(x)]^{2n}$
\end{itemize}
\myspace{1}

2. 设 $f(x)$ 在 $(-\infty,+\infty)$ 内可导,且对任意 $x_{1},x_{2}$,当 $x_{1}>x_{2}$ 时,都有 $f(x_{1})>f(x_{2})$,则:
\begin{itemize}
	\item A. 对任意 $x$,$f'(x)>0$
	\item B. 对任意 $x$,$f'(-x)\leq 0$
	\item C. 函数 $f(-x)$ 单调增加
	\item D. 函数 $-f(-x)$ 单调增加
\end{itemize}
\myspace{1}

3. 设 $f(x),g(x)$是恒大于零的可导函数,且 $f'(x)g(x)-f(x)g'(x)<0$,当 $a<x<b$ 时,有:
\begin{itemize}
	\item A. $f(x)g(b)>f(b)g(x)$
	\item B. $f(x)g(a)>f(a)g(x)$
	\item C. $f(x)g(x)>f(b)g(b)$
	\item D. $f(x)g(x)>f(a)g(a)$
\end{itemize}
\myspace{1}

4. 设 $\lim\limits_{x\to a}\dfrac{f(x)-f(a)}{(x-a)^{n}}=-1$,其中 $n$ 为 大于 $1$ 的整数,则在点 $x=a$ 处:
\begin{itemize}
	\item A. $f(x)$ 的导数存在,且 $f'(a)\neq 0$
	\item B. $f(x)$ 取得极大值
	\item C. $f(x)$ 取得极小值
	\item D. $f(x)$ 是否取得极值与 $n$ 的取值有关
\end{itemize}
\myspace{1}

5. 设 $f(x)$ 的导数在 $x=a$处连续,又 $\lim\limits_{x\to a}\dfrac{f'(x)}{x-a}=-1$,则:
\begin{itemize}
	\item A. $x=a$ 是 $f(x)$的极小值点
	\item B. $x=a$ 是 $f(x)$的极大值点
	\item C. $(a,f(a))$ 是曲线 $y=f(x)$ 的拐点
	\item D. $x=a$ 不是 $f(x)$ 的极值点,$(a,f(a))$也不是曲线 $y=f(x)$ 的拐点
\end{itemize}
\myspace{1}

6. 曲线 $y=(x-5)x^{\frac{2}{3}}$的拐点坐标为:
\myspace{1}

7. 已知函数 $y=f(x)$ 对一切 $x$ 满足 $xf''(x)+3x[f'(x)]^{2}=1-e^{-x}$,若 $f'(x_{0})=0(x_{0}\neq 0)$,则:
\begin{itemize}
	\item A. $f(x_{0})$ 是 $f(x)$ 的极大值
	\item B. $f(x_{0})$ 是 $f(x)$ 的极小值
	\item C. $(x_{0},f(x_{0}))$ 是曲线 $y=f(x)$ 的拐点
	\item D. $f(x_{0})$ 不是 $f(x)$ 的极值,$(x_{0},f(x_{0}))$ 也不是曲线 $y=f(x)$ 的拐点
\end{itemize}
\myspace{1}

8. 设函数 $f(x)$ 满足关系式 $f''(x)+[f'(x)]^{2}=x$ 且 $f'(0)=0$,则:
\begin{itemize}
	\item A. $f(0)$ 是 $f(x)$ 的极大值
	\item B. $f(0)$ 是 $f(x)$ 的极小值
	\item C. $(0,f(0))$ 是曲线 $y=f(x)$ 的拐点
	\item D. $f(0)$ 不是 $f(x)$ 的极值,$(0,f(0))$ 也不是曲线 $y=f(x)$ 的拐点
\end{itemize}
\myspace{1}

9. 设函数 $f(x)$ 在 $(-\infty,+\infty)$ 上连续,其导函数图形如图所示,则:
\begin{itemize}
	\item A. 函数 $f(x)$ 有 $2$ 个极值点,曲线 $y=f(x)$ 有 $2$ 个拐点
	\item B. 函数 $f(x)$ 有 $2$ 个极值点,曲线 $y=f(x)$ 有 $3$ 个拐点
	\item C. 函数 $f(x)$ 有 $3$ 个极值点,曲线 $y=f(x)$ 有 $1$ 个拐点
	\item D. 函数 $f(x)$ 有 $3$ 个极值点,曲线 $y=f(x)$ 有 $2$ 个拐点
\end{itemize}
\myspace{1}

10. 曲线 $y=x\ln\left( e+\dfrac{1}{x}\right)(x>0) $ 的渐近线方程为:
\myspace{1}

\section{Week \Rmnum{2}}
\hl{\textbf{\textit{January 8}}}

1. 曲线 $y=\dfrac{x^{2}+x}{x^{2}-1}$ 渐近线的条数为:
\begin{itemize}
	\item A. $0$
	\item B. $1$
	\item C. $2$
	\item D. $3$
\end{itemize}
\myspace{1}

2. 设函数 $y=\dfrac{x^{3}+4}{x^{2}}$,求

(1). 函数的增减区间及极值

(2). 函数图像的凹凸区间及拐点

(3). 渐近线

(4). 作出其图形
\myspace{1}

3. 在区间 $(-\infty,+\infty)$ 内,方程 $|x|^{\frac{1}{4}}+|x|^{\frac{1}{2}}-\cos x=0$: 
\begin{itemize}
	\item A. 无实根
	\item B. 有且仅有一个实根
	\item C. 有且仅有两个实根
	\item D. 有无穷多个实根
\end{itemize}
\myspace{1}

4. 函数 $f(x)=\ln|(x-1)(x-2)(x-3)|$ 的驻点个数为:
\begin{itemize}
	\item A. $0$
	\item B. $1$
	\item C. $2$
	\item D. $3$
\end{itemize}
\myspace{1}

5. 设 $f(x)=x^{2}(1-x)^{2}$,则方程 $f''(x)=0$在 $(0,1)$ 上:
\begin{itemize}
	\item A. 无实根
	\item B. 有且仅有一个实根
	\item C. 有且仅有两个实根
	\item D. 有且仅有三个实根
\end{itemize}
\myspace{1}

6. 设常数 $k>0$,设函数 $f(x)=\ln x-\dfrac{x}{e}+k$ 在 $(0,+\infty)$ 内零点个数为:
\begin{itemize}
	\item A. $3$
	\item B. $2$
	\item C. $1$
	\item D. $0$
\end{itemize}
\myspace{1}

7. 证明:当 $x>0$ 时,有不等式 $\ln(1+\dfrac{1}{x})>\dfrac{1}{1+x}$
\myspace{1}

8. 证明:当 $x>0$ 时,有不等式 $\arctan x+\dfrac{1}{x}>\dfrac{\pi}{2}$
\myspace{1}

9. 设 $p,q$ 是大于 $1$ 的常数,并且 $\dfrac{1}{p}+\dfrac{1}{q}=1$,证明:对于任意的 $x>0$,有 $\dfrac{1}{p}x^{p}+\dfrac{1}{q}\geq x$
\myspace{1}

10. 设函数 $f(x)$ 在 $[0,3]$ 上连续,在 $(0,3)$内可导,且 $f(0)+f(1)+f(2)=3,f(3)=1$,试证明:必存在$\xi\in(0,3),s.t.\ f'(\xi)=0$
\myspace{1}

\hl{\textbf{\textit{January 8}}}

1. 设 $f(x)$在区间 $[a,b]$ 上具有二阶导数,且 $f(a)=f(b)=0$,$f'(a)f'(b)>0$,试证明:存在 $\xi\in(a,b)$ 和 $\eta\in(a,b)$,$s.t.\ f(\xi)=0$ 且 $f''(\eta)=0$
\myspace{1}

2. 设 $f(x)$ 在 $[a,b]$ 上连续,在 $(a,b)$ 内可导,且 $f(a)=f(b)=0$,试证明:

(1). $\exists \xi\in(a,b),s.t.\ f'(\xi)+f(\xi)=0$

(2). $\exists \eta\in(a,b),s.t.\ f'(\eta)-f(\eta)=0$

(3). $\exists \zeta\in(a,b),s.t.\ f'(\zeta)+\lambda f(\zeta)=0$
\myspace{1}

3. 设 $f(x)$ 在 $[0,1]$ 上连续,在 $(0,1)$ 内可导,且 $f(1)=0$,试证明:$\exists \xi\in(0,1),s.t.\ \xi f'(\xi)=-f(\xi)$
\myspace{1}

4. 设函数 $f(x)$ 在闭区间 $[0,1]$ 上连续,在开区间$(0,1)$ 内可导,且$f(0)=0,f(1)=\dfrac{1}{3}$,证明:存在 $\xi\in\left(0,\dfrac{1}{2}\right),\eta\in\left( \dfrac{1}{2},1\right),s.t.\ f'(\xi)+f'(\eta)=\xi^{2}+\eta^{2}$
\myspace{1}

5. 设 $f(x)$ 在区间 $[a,b]$ 上连续,在 $(a,b)$ 内可导,且 $a,b$ 同号,证明:存在$\xi,\eta\in(a,b),s.t.\ abf'(\xi)=\eta^{2}f'(\eta)$
\myspace{1}

6. 求下列的不定积分

(1). $\int \dfrac{1}{\cos x}dx$

(2). $\int \dfrac{1}{\sin x}dx$
\myspace{1}

7. 求下列的不定积分

(1). $\int \dfrac{x+1}{x(1+xe^{x})}dx$

(2). $\int (1+\ln x)(\ln x+\ln\ln x)dx$
\myspace{1}

8. 求下列的不定积分

(1). $\int \dfrac{1+1}{1+x^{3}}dx$

(2). $\int \dfrac{1-x}{1+x^{3}}dx$
\myspace{1}

9. 求下列的不定积分

(1). $\int \dfrac{dx}{1+x^{3}}$

(2). $\int \dfrac{x}{1+x^{3}}dx$
\myspace{1}

10. 已知 $f(x)$ 的一个原函数为 $\ln^{2}x$,求 $\int xf'(x)dx$
\myspace{1}

\hl{\textbf{\textit{January 9}}}

1. 设 $f(\ln x)=\dfrac{\ln(1+x)}{x}$,求 $\int f(x)dx$
\myspace{1}

2. 计算不定积分 $\int \max(1,x^{2})dx$
\myspace{1}

3. 设 $M=\int_{-\frac{\pi}{2}}^{\frac{\pi}{2}}\dfrac{\sin x}{1+x^{2}}\cos^{4}xdx,N=\int_{-\frac{\pi}{2}}^{\frac{\pi}{2}}(\sin^{3}x+\cos^{4}x)dx,P=\int_{-\frac{\pi}{2}}^{\frac{\pi}{2}}(x^{2}\sin^{3}x-\cos^{4}x)dx$,则有: 
\begin{itemize}
	\item A. $N<P<M$
	\item B. $M<P<N$
	\item C. $N<M<P$
	\item D. $P<M<N$
\end{itemize}
\myspace{1}

4. 设 $M=\int_{-\frac{\pi}{2}}^{\frac{\pi}{2}}\dfrac{(1+x)^{2}}{1+x^{2}}dx,N=\int_{-\frac{\pi}{2}}^{\frac{\pi}{2}}\dfrac{1+x}{e^{x}}dx,K=\int_{-\frac{\pi}{2}}^{\frac{\pi}{2}}(1+\sqrt{\cos x})dx$,则有: 
\begin{itemize}
	\item A. $M>N>K$
	\item B. $M>K>N$
	\item C. $K>M>N$
	\item D. $K>N>M$
\end{itemize}
\myspace{1}

5. 设 $I_{1}=\int_{0}^{\frac{\pi}{4}}\dfrac{\tan x}{x}dx,I_{2}=\int_{0}^{\frac{\pi}{4}}\dfrac{x}{\tan x}dx$,则:
\begin{itemize}
	\item A. $I_{1}>I_{2}>1$
	\item B. $1>I_{1}>I_{2}$
	\item C. $I_{2}>I_{1}>1$
	\item D. $1>I_{2}>I_{1}$
\end{itemize}
\myspace{1}

6. 求定积分 $\int_{-2}^{2}[\ln(x+\sqrt{1+x^{2}})+\sqrt{1-\dfrac{x^{2}}{4}}]dx$
\myspace{1}

7. 求定积分 $\int_{-\pi}^{\pi}|x|[x^{3}+\sin^{2}x]\cos^{2}xdx$
\myspace{1}

8. 求定积分 $\int_{0}^{\pi}\sqrt{1-\sin x}dx$
\myspace{1}

9. 求定积分 $\int_{\sqrt{e}}^{e^{\frac{3}{4}}}\dfrac{dx}{x\sqrt{\ln x(1-\ln x)}}$
\myspace{1}

10. 求定积分 $\int_{0}^{1}\dfrac{\arcsin \sqrt{x}}{\sqrt{x(1-x)}}dx$
\myspace{1}

\hl{\textbf{\textit{January 10}}}

1. 求定积分 $\int_{0}^{1}\dfrac{\ln(1+x)}{(2-x)^{2}}dx$
\myspace{1}

2. 已知函数 $f(x)=\int_{1}^{x}\sqrt{1+t^{4}}dt$,则 $\int_{0}^{1}x^{2}f(x)dx$
\myspace{1}

3. 已知 $f(x)$ 连续,$\int_{0}^{x}tf(x-t)dt=1-\cos x$,求 $\int_{0}^{\frac{\pi}{2}}f(x)dx$
\myspace{1}

4. 设 $f(x)=\begin{cases}
	x^{2},0\leq x\leq 1\\2-x, 1<x\leq 2 
\end{cases}$,记 $F(x)=\int_{0}^{x}f(t)dt(0\leq x\leq 2)$,则有:
\begin{itemize}
	\item A. $F(x)=\begin{cases}
		\dfrac{x^{3}}{3},0\leq x\leq 1\\
		2x-\dfrac{x^{2}}{2}, 1<x\leq 2
	\end{cases}$
	\item B. $F(x)=\begin{cases}
		\dfrac{x^{3}}{3},0\leq x\leq 1\\
		\dfrac{1}{3}+2x-\dfrac{x^{2}}{2}, 1<x\leq 2
	\end{cases}$
	\item C. $F(x)=\begin{cases}
		\dfrac{x^{3}}{3},0\leq x\leq 1\\
		-\dfrac{7}{6}+2x-\dfrac{x^{2}}{2}, 1<x\leq 2
	\end{cases}$
	\item D. $F(x)=\begin{cases}
		\dfrac{x^{3}}{3},0\leq x\leq 1\\
		\dfrac{x^{3}}{3}+2x-\dfrac{x^{2}}{2}, 1<x\leq 2
	\end{cases}$
\end{itemize}
\myspace{1}

5. 设 $x\geq -1$,求 $\int_{-1}^{x}(1-|t|)dt$
\myspace{1}

6. 设 $x=x(t)$ 由方程 $\sin t-\int_{1}^{x-t}e^{-u^{2}}du=0$ 所确定,试求 $\dfrac{d^{2}x}{dt^{2}}|_{t=0}$
\myspace{1}

7. 设函数 $f(x)=\int_{0}^{1}|t(t-x)|dt(0<x<1)$,求 $f(x)$ 的极值、单调区间及曲线 $y=f(x)$ 的凹凸区间
\myspace{1}

8. 下列反常积分中发散的是:
\begin{itemize}
	\item A. $\int_{0}^{+\infty}xe^{-x}dx$
	\item B. $\int_{0}^{+\infty}xe^{-x^{2}}dx$
	\item C. $\int_{0}^{+\infty}\dfrac{\arctan x}{1+x^{2}}dx$
	\item D. $\int_{0}^{+\infty}\dfrac{x}{1+x^{2}}dx$
\end{itemize}
\myspace{1}

9. 求 $I=\int_{5}^{+\infty}\dfrac{dx}{x^{2}-4x+3}$
\myspace{1}

10. 求 $I=\int_{1}^{+\infty}\dfrac{dx}{e^{1+x}+e^{3-x}}$
\myspace{1}

\hl{\textbf{\textit{January 11}}}

1. 求 $\int_{0}^{+\infty}\dfrac{\ln(1+x)}{(1+x)^{2}}dx$
\myspace{1}

2. 已知抛物线通过 $x$ 轴上的两点 $A(1,0),B(3,0)$

(1). 求证: 两坐标轴与该抛物线所围图形的面积等于 $x$ 轴与该抛物线所围图形的面积

(2). 计算上述两平面图形绕 $x$ 轴旋转一周所产生的两个旋转体体积之比
\myspace{1}

3. 求心形线 $r=a(1+\cos\theta)\ (a>0)$ 所围图形的面积
\myspace{1}

4. 已知平面区域 $D=\{(x,y)|0\leq y\leq \dfrac{1}{x\sqrt{1+x^{2}}},x\geq 1\}$

(1). 求 $D$ 的面积
(2). 求 $D$ 绕 $x$ 轴旋转所成旋转体的体积
\myspace{1}

5. 某水库的闸门形状为等腰梯形, 它的两条底边各长 $10m$ 和 $6m$, 高为 $20m$,较长的底边与水面相齐,求闸门的一侧所受水的压力
\myspace{1}

6. 一个半径为 $R(m)$ 的球形贮水箱盛满了水,如果把箱中的水从顶部全部抽出,需要作的功
\myspace{1}

7. 方程 $(xy^{2}+x)dx+(y-x^{2}y)dy=0$ 的通解
\myspace{1}

8. 方程 $y'=1+x+y^{2}+xy^{2}$ 的通解
\myspace{1}

9. 方程 $(y+\sqrt{x^{2}+y^{2}})dx-xdy=0$ 满足条件 $y|_{x=1}=0$ 的特解
\myspace{1}

10. 方程 $\dfrac{dy}{dx}=\dfrac{y}{x+y^{4}}$ 的通解
\myspace{1}

\hl{\textbf{\textit{January 12}}}

1. 求不定积分 $\int\dfrac{x\ln x+x\ln^{2}x}{2+x\ln x}dx$
\myspace{1}

2. 求不定积分 $\int\dfrac{\sin 2x\sin^{2}x}{2+\cos^{4} x}dx$
\myspace{1}

3. 求不定积分 $\int\dfrac{\sin x}{\sin x+\cos x}dx$
\myspace{1}

4. 求不定积分 $\int\dfrac{\cos 2x}{\cos^{2} x(1+\sin^{2} x)}dx$
\myspace{1}

5. 求不定积分 $\int\dfrac{1}{\cos^{3} x}dx$
\myspace{1}

6. 求不定积分 $\int\dfrac{2+x}{(1+x^{2})^{2}}dx$
\myspace{1}

7. 求不定积分 $\int\dfrac{\arcsin\sqrt{x}+\ln x}{\sqrt{x}}dx$
\myspace{1}

8. 求不定积分 $\int e^{x}\arcsin\sqrt{1-e^{2x}}dx$
\myspace{1}

9. 已知 $f(\ln x)=\dfrac{\ln(1+x)}{x}$, 求 $\int f(x)dx$
\myspace{1}

10. 求定积分 $\int_{0}^{\frac{\pi}{2}}x\sin^{2}x\cos^{2}xdx$
\myspace{1}

\hl{\textbf{\textit{January 13}}}

1. 求定积分 $\int_{0}^{\pi^{2}}\sqrt{x}\cos\sqrt{x}dx$
\myspace{1}

2. 求定积分 $\int_{0}^{2}x\sqrt{2x-x^{2}}dx$
\myspace{1}

3. 求定积分 $\int_{0}^{1}x(1-x^{4})^{\frac{3}{2}}dx$
\myspace{1} 

4. 求定积分 $\int_{1}^{2}(x-1)^{2}(x-2)^{2}dx$
\myspace{1}

5. 求定积分 $\int_{-2}^{2}x\ln(1+e^{x})dx$
\myspace{1}

6. 设 $D$ 是由曲线 $xy+1=0$ 与直线 $y+x=0$ 及 $y=2$ 围成的有界区域,求 $D$ 的面积
\myspace{1}

7. 设 $D$ 是由曲线 $y=x^{2}$ 与 $y=x$ 围成的有界区域,求区域 $D$ 分别绕直线 $y=0,x=0,x=1,x=2$ 旋转所得旋转体的体积
\myspace{1}

8. 方程 $y''+4y'+4y=e^{-2x}$ 满足条件 $y(0)=1,y'(0)=0$ 的特解
\myspace{1}

9. 具有特解 $y_{1}=e^{-x},y_{2}=2xe^{-x},y_{3}=3e^{x}$ 的三阶常系数线性齐次方程为:
\begin{itemize}
	\item A. $y'''-y''-y'+y=0$
	\item B. $y'''+y''-y'-y=0$
	\item C. $y'''-6y''+11y'-6y=0$
	\item D. $y'''-2y''-y'+2y=0$
\end{itemize}
\myspace{1}

10. 方程 $y''-2y'=xe^{2x}$ 的特解形式为:
\begin{itemize}
	\item A. $y=axe^{2x}$
	\item B. $y=(ax+b)e^{2x}$
	\item C. $y=x(ax+b)e^{2x}$
	\item D. $y=x^{2}(ax+b)e^{2x}$
\end{itemize}
\myspace{1}

\hl{\textbf{\textit{January 14}}}

1. 方程 $y''+y=e^{x}+1+\sin x$ 的特解形式为:
\begin{itemize}
	\item A. $ae^{x}+b+c\sin x$
	\item B. $ae^{x}+b+c\cos x+d\sin x$
	\item C. $ae^{x}+b+x(c\cos x+d\sin x)$
	\item D. $y=ae^{x}+b+cx\sin x$
\end{itemize}
\myspace{1}

2. 设函数 $f(x)$ 具有一阶连续导数, 且满足 $f(x)=\int_{0}^{x}(x^{2}-t^{2})f'(t)dt+x^{2}$, 求 $f(x)$ 表达式
\myspace{1}

3. 设 $L$ 是一条平面曲线, 其上任意一点 $P(x,y)(x>0)$ 到坐标原点的距离恒等于该点处切线在 $y$ 轴上的截距, 且 $L$ 经过 $(\dfrac{1}{2},0)$, 求曲线 $L$ 的渐近线方程为
\myspace{1}

4. 二元函数 $f(x,y)=
\begin{cases}
	\dfrac{x^{2}y}{x^{2}+y^{2}},&(x,y)\neq (0,0)\\
	0,&(x,y)=(0,0)
\end{cases}$ 在点 $(0,0)$ 处:
\begin{itemize}
	\item A. 不连续
	\item B. 两个偏导数都不存在
	\item C. 偏导数存在但不可微
	\item D. 可微
\end{itemize}
\myspace{1}

5. 二元函数 $f(x)$ 在点 $(x_{0},y_{0})$ 处两个偏导数 $f'_{x}(x_{0},y_{0}),f'_{y}(x_{0},y_{0})$ 存在, 是 $f(x)$ 在点 $(x_{0},y_{0})$ 处连续的:
\begin{itemize}
	\item A. 充分不必要条件
	\item B. 必要不充分条件
	\item C. 充分必要条件
	\item D. 不充分不必要条件
\end{itemize}
\myspace{1}

6. 已知 $f(x,y)=\sin\sqrt{x^{4}+y^{4}}$, 则:
\begin{itemize}
	\item A. $f'_{x}(0,0),f'_{y}(0,0)$ 都存在
	\item B. $f'_{x}(0,0)$ 不存在, $f'_{y}(0,0)$ 存在
	\item C. $f'_{x}(0,0)$ 存在, $f'_{y}(0,0)$ 不存在
	\item D. $f'_{x}(0,0),f'_{y}(0,0)$ 都不存在
\end{itemize}

7. 设 $f(x,y)=\dfrac{2x+y^{2}}{1+y^{2}\sqrt{1+x^{2}+y^{2}}}$, 则 $d f(0,0)$
\myspace{1}

8. 已知 $dF(x,y)=xye^{x}dx+(f(x)+y^{2})dy$, 且 $f(x)$ 有连续一阶导数, $f(x)=0$, 求 $F(x,y)$
\myspace{1}

9. 设函数 $f(x,y)$ 可微, 且对于任意 $x,y$ 都有 $\dfrac{\partial f(x,y)}{\partial x}>0,\dfrac{\partial f(x,y)}{\partial y}<0$, 则下列结论正确的是:
\begin{itemize}
	\item A. $f(1,1)>f(0,0)$
	\item B. $f(-1,1)>f(0,0)$
	\item C. $f(-1,-1)>f(0,0)$
	\item D. $f(1,-1)>f(0,0)$
\end{itemize}
\myspace{1}

10. 设 $z=(x+e^{y})^{x}$, 求 $\dfrac{\partial z}{\partial x}_{(1,0)}$
\myspace{1}

\hl{\textbf{\textit{January 15}}}

1. 设函数 $z=z(x,y)$ 由方程 $(x+1)z+y\ln z-\arctan(2xy)=1$ 确定, 求$\dfrac{\partial z}{\partial x}_{(0,2)}$
\myspace{1}

2. 设 $f(x,y,z)=e^{x}+y^{2}z$, 其中 $z=z(x,y)$ 是由方程 $x+y+z+xyz=0$ 所确定的隐函数, 求 $f'_{x}(0,1,-1)$
\myspace{1}

3. 设 $z=xyf(\dfrac{y}{x})$, 其中 $f(u)$ 可导, 求 $xz'_{x}+yz'_{y}$
\myspace{1}

4. 设 $z=e^{xy}+f(x+y,xy)$, 求 $\dfrac{\partial^{2} z}{\partial x\partial y}$, 其中 $f(u,v)$ 有二阶连续偏导数
\myspace{1}

5. 已知 $z=f(x,y)$ 在 $(x_{0},y_{0})$ 处取得极小值, 则:
\begin{itemize}
	\item A. $f'_{x}(x_{0},y_{0})=f'_{y}(x_{0},y_{0})=0$
	\item B. $f''_{xx}(x_{0},y_{0})f''_{yy}(x_{0},y_{0})-(f''_{xy}(x_{0},y_{0}))^{2}>0$, 且 $f''_{xx}(x_{0},y_{0})>0$
	\item C. $f(x_{0},y)$ 在 $y_{0}$ 处取得极大值
	\item D. $f(x,y_{0})$ 在 $x_{0}$ 处取得极小值
\end{itemize}
\myspace{1}

6. 设函数 $f(x),g(x)$ 均有二阶连续导数,满足 $f(0)>0,g(0)<0$, 且 $f'(0)=g'(0)=0$, 则函数 $z=f(x)g(y)$ 在点 $(0,0)$ 处取得极小值的一个充分条件是:
\begin{itemize}
	\item A. $f''(0)<0,g''(0)>0$
	\item B. $f''(0)<0,g''(0)<0$
	\item C. $f''(0)>0,g''(0)>0$
	\item D. $f''(0)>0,g''(0)<0$
\end{itemize}
\myspace{1}

7. 已知函数 $z=f(x,y)$ 的全微分 $dz=(ay-x^{2})dx+(ax-y^{2})dy,(a>0)$, 则函数 $f(x,y)$:
\begin{itemize}
	\item A. 无极值点
	\item B. 点 $(a,a)$ 为极小值点
	\item C. 点 $(a,a)$ 为极大值点
	\item D. 是否有极值点与 $a$ 的取值有关
\end{itemize}
\myspace{1}

8. 设函数 $z=f(xy,yg(x))$, 其中 $f$ 函数具有二阶连续偏导数, 函数 $g(x)$ 可导且在 $x=1$ 处取得极值 $g(1)=1$, 求 $\dfrac{\partial^{2}z}{\partial x\partial y}|_{(1,1)}$
\myspace{1}

9. 求函数 $f(x,y)=xe^{-\frac{x^{2}+y^{2}}{2}}$ 的极值
\myspace{1}

10. 求 $f(x,y)=x^{2}-y^{2}+2$ 在椭圆域 $D=\left\lbrace (x,y)|x^{2}+\dfrac{y^{2}}{4}\leq 1\right\rbrace$ 上的最大值和最小值
\myspace{1}

\hl{\textbf{\textit{January 16}}}

1. 交换二次积分的积分次序($a>0$)

(1). $\int_{0}^{2}dx\int_{\frac{x^{2}}{4}}^{3-x}f(x,y)dy$

(2). $\int_{0}^{a}dy\int_{0}^{\sqrt{ay}}f(x,y)dx+\int_{a}^{2a}dy\int_{0}^{2a-y}dx$
\myspace{1}

2. 设 $f(x)$ 是连续函数, 则 $\int_{0}^{1}dy\int_{-\sqrt{1-y^{2}}}^{1-y}f(x,y)dx$ 等价于:
\begin{enumerate}
	\item A. $\int_{0}^{1}dx\int_{0}^{x-1}f(x,y)dy+\int_{-1}^{0}dx\int_{0}^{\sqrt{1-x^{2}}}f(x,y)dy$
	\item B. $\int_{0}^{1}dx\int_{0}^{1-x}f(x,y)dy+\int_{-1}^{0}dx\int_{-\sqrt{1-x^{2}}}^{0}f(x,y)dy$
	\item C. $\int_{0}^{\frac{\pi}{2}}d\theta\int_{0}^{\frac{1}{\cos\theta+\sin\theta}}f(r\cos\theta,r\sin\theta)dr-\int_{\frac{\pi}{2}}^{\pi}d\theta\int_{0}^{1}f(r\cos\theta,r\sin\theta)dr$
	\item D. $\int_{0}^{\frac{\pi}{2}}d\theta\int_{0}^{\frac{1}{\cos\theta+\sin\theta}}f(r\cos\theta,r\sin\theta)rdr-\int_{\frac{\pi}{2}}^{\pi}d\theta\int_{0}^{1}f(r\cos\theta,r\sin\theta)rdr$
\end{enumerate}
\myspace{1}

3. 设函数 $f(t)$ 连续,则二次积分 $\int_{0}^{\frac{\pi}{2}}d\theta\int_{2\cos\theta}^{2}f(r^{2})rdr$
\begin{enumerate}
	\item A. $\int_{0}^{2}dx\int_{\sqrt{2x-x^{2}}}^{\sqrt{4-x^{2}}}\sqrt{x^{2}+y^{2}}f(x^{2}+y^{2})dy$
	\item B. $\int_{0}^{2}dx\int_{\sqrt{2x-x^{2}}}^{\sqrt{4-x^{2}}}f(x^{2}+y^{2})dy$
	\item C. $\int_{0}^{2}dy\int_{1+\sqrt{1-y^{2}}}^{\sqrt{4-y^{2}}}\sqrt{x^{2}+y^{2}}f(x^{2}+y^{2})dx$
	\item D. $\int_{0}^{2}dy\int_{1+\sqrt{1-y^{2}}}^{\sqrt{4-y^{2}}}f(x^{2}+y^{2})dx$
\end{enumerate}
\myspace{1}

4. 计算二重积分

(1). $\iint\limits_{x^{2}+y^{2}\leq 1}(2x+3y)^{2}d\sigma$

(2). $\int_{\frac{1}{4}}^{\frac{1}{2}}dy\int_{\frac{1}{2}}^{\sqrt{y}}e^{\frac{y}{x}}dx+\int_{\frac{1}{2}}^{1}dy\int_{y}^{\sqrt{y}}e^{\frac{y}{x}}dx$

(3). $\int_{0}^{1}dy\int_{y}^{1}\sqrt{x^{2}-y^{2}}dx$

(4). $\int_{0}^{1}dy\int_{y}^{1}\left(\frac{e^{x^{2}}}{x}-e^{y^{3}}\right)dx$

(5). $\iint\limits_{D}\left(xy^{5}-1\right)dxdy,D=\left\{(x,y)|-\frac{\pi}{2}\leq x \leq \frac{\pi}{2},\sin x\leq y \leq 1 \right\}$

(6). $\iint\limits_{D}x^{2}ydxdy$, 其中 $D$ 是由双曲线 $x^{2}-y^{2}=1$以及直线 $y=0,y=1$ 所围成的平面区域

(7). $\iint\limits_{D}\sqrt{x^{2}+y^{2}}dxdy,D=\left\{(x,y)|0\leq y \leq x, x^{2}+y^{2}\leq 2x\right\}$

(8). $\iint\limits_{D}r^{2}\sin\theta\sqrt{1-r^{2}\cos 2\theta}drd\theta,D=\left\{(r,\theta)|0\leq r\leq \sec\theta,0\leq \theta \leq \frac{\pi}{4}\right\}$

\myspace{1}

\section{Week \Rmnum{3}}

\section{Week \Rmnum{4}}

