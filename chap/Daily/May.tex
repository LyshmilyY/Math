\chapterimage{chap31.jpg}
\chapter{May}
\section{Week \Rmnum{1}}
\hl{\textbf{\textit{May 1}}}

1. 二阶常系数非齐次线性微分方程 $y''-4y'+3y=2e^{2x}$ 的通解为
\myspace{1}
\begin{solution}
	
	原微分方程对应的特征方程为: $r^2-4r+3=0\Rightarrow r_{1}=1,\ r_{2}=3$
	
	齐次微分方程的通解为: $y=C_{1}e^{x}+C_{2}e^{3x}$
	
	我们设方程的特解为: $y^{*}=Ae^{2x}\Rightarrow A(4-8+3)e^{2x}=2e^{2x}\Rightarrow A=-2$
	
	我们得到原微分方程的通解为: 
	$$y=-2e^{2x}+C_{1}e^{x}+C_{2}e^{3x},C_{1},C_{2}\in\mathbb{R}$$
\end{solution}
\myspace{1}

2.求微分方程 $y''-2y'-e^{2x}=0$ 满足条件 $y(0)=1,y'(0)=1$的解.
\myspace{1}
\begin{solution}
	
	原微分方程对应的特征方程为: $r^2-2r=0\Rightarrow r_{1}=0,\ r_{2}=2$
	
	我们得到齐次微分方程的通解: $y=C_{1}e^{2x}+C_{2}$
	
	我们设微分方程的特解为: $$y^{*}=Axe^{2x}\Rightarrow A(4+4x-2-4x)e^{2x}=e^{2x}\Rightarrow A=\frac{1}{2}$$
	
	我们得到方程的解为: $y=C_{1}e^{2x}+C_{2}+\dfrac{1}{2}xe^{2x}$
	
	又因为$y(0)=1,y'(0)=1\Rightarrow$,我们得到: 
	$$\left\lbrace 
	\begin{array}{l}
		C_{1}+C_{2}=1\\
		2C_{1}+\dfrac{1}{2}=1
	\end{array}
	\right. \Rightarrow \left\lbrace 
	\begin{array}{l}
		C_{1}=\dfrac{1}{4}\\
		C_{2}=\dfrac{3}{4}
	\end{array}
	\right. $$
	
	原微分方程的解: $y=\dfrac{1}{4}e^{2x}+\dfrac{1}{2}xe^{2x}+\dfrac{3}{4}$
\end{solution}
\myspace{1}

\hl{\textbf{\textit{May 2}}}

1. 设$f(x)$ 在 $(0,+\infty)$ 连续,$f(x)>0$,$\lim\limits_{x\rightarrow  +\infty}f(x)=1$,$\lim\limits_{h\rightarrow 0}\left[\dfrac{f(x+hx)}{f(x)} \right]^{\frac{1}{h}}=e^{\frac{1}{x}} $,求$f(x)$
\myspace{1}
\begin{solution}
	
	我们得到: 
	$$\lim\limits_{h\rightarrow 0}e^{\frac{1}{h}\ln(1+\frac{f(x+hx)-f(x)}{f(x)})}=\lim\limits_{h\rightarrow 0}e^{\frac{1}{h}\frac{f(x+hx)-f(x)}{f(x)}}=e^{\frac{1}{x}}$$
	
	我们进而得到: 
	$$\lim\limits_{h\rightarrow 0}\frac{x}{f(x)}\frac{f(x+x)-f(x)}{hx}=\frac{1}{x}\Rightarrow \frac{f'(x)}{f(x)}=\frac{1}{x^2}$$
	
	即: $\ln f(x)=-\dfrac{1}{x}+C$,又因为$\lim\limits_{x\rightarrow  +\infty}f(x)=1$,两边同时取$x\rightarrow +\infty$ 的极限: 
	$$\lim\limits_{x\rightarrow +\infty}\ln f(x)=\lim\limits_{x\rightarrow +\infty}(-\frac{1}{x}+C)\Rightarrow C=0$$
	
	我们得到: $f(x)=e^{-\frac{1}{x}}$
\end{solution}
\myspace{1}

2. $\sum\limits_{n=1}^{+\infty}\dfrac{4^n}{5^n-3^n}$和$\sum\limits_{n=1}^{+\infty}\dfrac{1}{\sqrt{n^3-1}}$敛散性
\myspace{1}
\begin{solution}
	
	采取抓大头的方式,利用级数判别方法中的比较法的极限形式
	
	(i).$\lim\limits_{n\rightarrow+\infty}\dfrac{(\frac{4^n}{5^n-3^n})^n}{(\dfrac{4}{5})^n}=1\Rightarrow \text{原级数和级数}\sum\limits_{n=0}^{+\infty}(\dfrac{4}{5})^n\text{同敛散性}$,原级数收敛
	
	(ii).$\lim\limits_{n\rightarrow+\infty}\dfrac{\frac{1}{\sqrt{n^3-1}}}{\dfrac{1}{\sqrt{n^3}}}=1\Rightarrow \text{原级数和级数}\sum\limits_{n=0}^{+\infty}\dfrac{1}{\sqrt{n^3}}\text{同敛散性}$,原级数收敛
\end{solution}
\myspace{1}

\hl{\textbf{\textit{May 3}}}

1. 设函数$f(x)$连续,且对于任意的$x\in(-\infty,+\infty)$,恒有$f(x+1)=-f(x)$,下面结论不正确的是: 
\begin{itemize}
	\item A. $f(x)\text{是以2为周期的函数}$ 
	\item B. $\int_{0}^{x}[f(t)-f(-t)]dt\text{是以2为周期的函数}$ 
	\item C. $\int_{0}^{x}f(t)dt-\dfrac{x}{2}\int_{0}^{2}f(t)dt\text{是以2为周期的函数}$ 
	\item \hl{D}. $f'(x)\text{是以2为周期的函数}$ 
\end{itemize}
\myspace{1}
\begin{solution}
	
	我们得到: $\left\lbrace 
	\begin{array}{l}
		f(x+1)+f(x)=0\\
		f(x+2)+f(x+1)=0
	\end{array}
	\right. \Rightarrow f(x+2)=f(x)$
	
	我们得到: $f(x)$是以$2$为周期的周期函数,$A$正确
	
	我们由原函数和导函数周期性关系得到: 
	$$f(x)\text{周期函数},\int_{0}^{x}f(t)dt\text{周期函数}\Leftrightarrow \int_{0}^{T}f(x)dx=0$$
	
	对于$B$选项,我们发现$g(x)=f(x)-f(-x),g(x)\text{是奇函数}$
	
	$g(x)\text{是周期函数}$,且$\int_{-1}^{1}g(x)dx=0\Rightarrow \int_{0}^{x}g(t)dt\text{是周期函数}$
	
	对于$C$选项,我们得到: $\int_{0}^{x}[f(t)-\dfrac{\int_{0}^{2}f(x)dx}{2}]dt$
	
	我们只需要证明: 
	$$\int_{0}^{2}[f(t)-\frac{\int_{0}^{2}f(x)dx}{2}]dt=0$$
	
	不妨设$\int_{0}^{2}f(x)dx=A$,我们有: 
	$$\text{左边}=\int_{0}^{2}[f(t)-\frac{A}{2}]dt=\int_{0}^{2}f(x)dx-A=0\Rightarrow \text{原命题得证}$$
	
	对于$D$选项,我们未知$f(x)$是否可导,故此题答案选$D$
\end{solution}
\myspace{1}

2. 求微分方程: $y''+y=4\sin x$的通解
\myspace{1}
\begin{solution}
	
	特征方程为: $r^2+1=0\Rightarrow r_{1}=i,\ r_{2}=-i$
	
	齐次微分方程的通解为: $y=A\sin x+B\cos x$
	
	非齐次微分方程特解: $y^{*}=x(C_{1}\sin x+C_{2}\cos x)\Rightarrow \left\lbrace 
	\begin{array}{l}
		C_{1}=0\\
		C_{2}=-2
	\end{array}
	\right. $
	
	原微分方程的通解为: $y=-2x\cos x+A\sin x+B\cos x,A\text{、}B\in \mathbb{R}$
\end{solution}
\myspace{1}

\hl{\textbf{\textit{May 4}}}

1. 求微分方程: $y''+y=\sin x+x\cos 2x$的通解
\myspace{1}
\begin{solution}
	
	特征方程为: $r^2+1=0\Rightarrow r_{1}=i,\ r_{2}=-i$
	
	齐次微分方程的通解为: $y=A\sin x+B\cos x$
	
	非齐次微分方程特解: 
	$$y_{1}^{*}=x(C_{1}\sin x+C_{2}\cos x)\Rightarrow \left\lbrace 
	\begin{array}{l}
		C_{1}=0\\
		C_{2}=-\frac{1}{2}
	\end{array}
	\right. \Rightarrow y_{1}^{*}=-\frac{x}{2}\cos x$$
	$$y_{1}^{*}=(Ax+B)\sin 2x+(Cx+D)\cos 2x)\Rightarrow \left\lbrace 
	\begin{array}{l}
		C=-\frac{1}{3}\\
		A=D=0\\
		B=\frac{4}{9}
	\end{array}
	\right.\Rightarrow y_{2}^{*}=-\frac{x}{3}\cos 2x+\frac{4}{9}\sin 2x $$
	
	原微分方程的通解为: $y=y_{1}^{*}+y_{2}^{*}+A\sin x+B\cos x,A\text{、}B\in \mathbb{R}$
\end{solution}
\myspace{1}

2. $\lim\limits_{x\rightarrow 0 }\int_{0}^{2x}du\int_{0}^{\sqrt{2ux-u^2}}\dfrac{cos(t-u)^2}{\ln(1+x|x|)}dt$
\myspace{1}
\begin{solution}
	
	我们不难发现此题需要分左右极限分别来求: 
	
	$$I_{\text{左}}=\lim\limits_{x\rightarrow 0 }\frac{-\iint\limits_{D_{1}}\cos(t-u)^2d\sigma}{-x^2}$$
	
	由二元函数的积分中值定理: 
	
	$$\iint\limits_{D_{1}}\cos(t-u)^2d\sigma=S_{D_{1}}f(\varepsilon_{1},\varphi_{1}),(\varepsilon_{1},\varphi_{1})\text{在以}(0,x)\text{为半径},x
	\text{为半径的圆内}$$
	$$I_{\text{左}}=\lim\limits_{x\rightarrow 0 }\frac{\frac{x^2\pi}{2}}{-x^2}\cos(\varepsilon_{1}-\varphi_{1})^2=\frac{\pi}{2}$$
	
	$$I_{\text{右}}=\lim\limits_{x\rightarrow 0 }\frac{\iint\limits_{D_{2}}\cos(t-u)^2d\sigma}{x^2}$$
	
	由二元函数的积分中值定理: 
	
	$$\iint\limits_{D_{2}}\cos(t-u)^2d\sigma=S_{D_{2}}f(\varepsilon_{2},\varphi_{2}),(\varepsilon_{2},\varphi_{2})\text{在以}(0,x)\text{为半径},x
	\text{为半径的圆内}$$
	$$I_{\text{右}}=\lim\limits_{x\rightarrow 0 }\frac{\frac{x^2\pi}{2}}{x^2}\cos(\varepsilon_{2}-\varphi_{2})^2=\frac{\pi}{2}$$
	
	综上所述: $I=\frac{\pi}{2}$
\end{solution}
\myspace{1}

\hl{\textbf{\textit{May 5}}}

1. 下列函数中原函数必为周期函数的是: 
\begin{itemize}
	\item A. $|\sin x|$ 
	\item B. $\sin^4 x$ 
	\item C. $\dfrac{1}{1+\sin^2x}$ 
	\item \hl{D}. $\dfrac{\sin x}{1+\sin^4 x}$ 
\end{itemize}
\myspace{1}
\begin{solution}
	
	原函数为周期函数,只需满足函数为周期函数,且$\int_{0}^{T}f(t)dt=0$
	
	对于四个选项: $A,B,C\text{对应的}f(t)\text{在一个周期中函数值大于0},\int_{0}^{T}f(t)dt>0$
	
	此题答案选$D$
\end{solution}
\myspace{1}

2. 设$y=\dfrac{1}{2}e^{2x}+(x-\frac{1}{3})e^{x}$是二阶常系数线性微分方程$y''+ay'+by=ce^x$的一个特解,则: 
\begin{itemize}
	\item \hl{A}. $a=-3,\ b=2,\ c=-1$ 
	\item B. $a=3,\ b=2,\ c=-1$ 
	\item C. $a=-3,\ b=2,\ c=1$ 
	\item D. $a=3,\ b=2,\ c=1$ 
\end{itemize}
\myspace{1}
\begin{solution}
	
	我们可以得到特征方程的两根: $r_{1}=1,\ r_{2}=2\Rightarrow \text{特征方程为: }r^2-3r+2=0\Rightarrow a=-3,\ b=2$
	
	我们将特解$y{*}=xe^{x}$代入微分方程,得到: $c=-1$,故此题选$A$
\end{solution}
\myspace{1}

\hl{\textbf{\textit{May 6}}}

1. 设$F(a)=\int_{0}^{\frac{\pi}{2}}|\sin x-a\cos x|dx$,求$F(a)$的最小值
\myspace{1}
\begin{solution}
	
	$$F(a)=\sqrt{a^2+1}\int_{0}^{\frac{\pi}{2}}|\sin(x+\varphi)|dx,\text{其中}\left\lbrace 
	\begin{array}{l}
		\cos \varphi=\frac{1}{\sqrt{a^2+1}}\\
		\sin \varphi=-\frac{a}{\sqrt{a^2+1}}
	\end{array}
	\right. $$
	
	(1).$a>0,\ \varphi\in(-\frac{\pi}{2},0)$
	
	$$\frac{F(a)}{\sqrt{a^2+1}}=\int_{\varphi}^{\frac{\pi}{2}+\varphi}|\sin t|dt=2\int_{0}^{-\varphi}\sin tdt+\int_{-\varphi}^{\frac{\pi}{2}+\varphi}\sin tdt=2-\cos \varphi+\sin \varphi=2-\frac{a+1}{\sqrt{a^2+1}}$$
	\begin{eqnarray*}
		F(a)&=&2\sqrt{a^2+1}-(a+1)\\ F'(a)&=&\frac{2a-\sqrt{a^2+1}}{\sqrt{a^2+1}}\\
		\text{当}x&=&\frac{\sqrt{3}}{3},F(a)_{min}=\sqrt{3}-1
	\end{eqnarray*}
	
	(2).$a<0,\ \varphi\in(0,\frac{\pi}{2})$
	
	$$\frac{F(a)}{\sqrt{a^2+1}}=\int_{\varphi}^{\frac{\pi}{2}+\varphi}|\sin t|dt=\cos \varphi+\sin \varphi=\frac{1-a}{\sqrt{a^2+1}}$$
	\begin{eqnarray*}
		F(a)&=&1-a,\ F(a)\geq 1,\ F(a)_{min}=1
	\end{eqnarray*}
	
	综上,$F(a)_{min}=\sqrt{3}-1$
\end{solution}
\myspace{1}

2. 设 $f(x)$连续且以$T$为周期,则下列函数以$T$为周期的是: 
\begin{itemize}
	\item A. $\int_{0}^{x}f(t)dt$ 
	\item B. $\int_{-x}^{0}f(t)dt$ 
	\item \hl{C}. $\int_{0}^{x}f(t)dt-\int_{-x}^{0}f(t)dt$ 
	\item D. $\int_{0}^{x}f(t)dt+\int_{-x}^{0}f(t)dt$ 
\end{itemize}
\myspace{1}
\begin{solution}
	
	我们知道如果原函数也为周期函数,那么必有: $\int_{0}^{T}f(t)dt=0$
	
	对于$A,B$选项,我们自然可以排除掉,没有任何附加条件
	
	对于$C,D$选项
	$$C\Rightarrow \int_{0}^{x}[f(t)-f(-t)]dt\text{奇函数}\Rightarrow \int_{-\frac{T}{2}}^{\frac{T}{2}}[f(t)-f(-t)]dt=0$$
	$$D\Rightarrow \int_{0}^{x}[f(t)+f(-t)]dt\text{偶函数}$$
	
	故此题答案选$C$
\end{solution}
\myspace{1}

\hl{\textbf{\textit{May 7}}}

1. $\int_{0}^{+\infty}\dfrac{dx}{(1+x^2)(1+x^3)}$
\myspace{1}
\begin{solution}
	
	令$x=\dfrac{1}{t},t\in(0,+\infty),dx=-\dfrac{1}{t^2}dt$,原积分等价于: 
	$$I=\int_{0}^{+\infty}\dfrac{t^3}{(1+t^2)(1+t^3)}dt=\int_{0}^{+\infty}\dfrac{x^3}{(1+x^2)(1+x^3)}dx$$
	$$2I=\int_{0}^{+\infty}\dfrac{1+x^3}{(1+x^2)(1+x^3)}dx=\int_{0}^{+\infty}\dfrac{1}{1+x^2}dx=\frac{\pi}{2}$$
	$$I=\int_{0}^{+\infty}\dfrac{dx}{(1+x^2)(1+x^3)}=\frac{\pi}{4}$$
\end{solution}
\myspace{1}

2. 设连续函数$f(x)$满足$f(x)+2\int_{0}^{x}f(t)dt=x^2$,求$f(x)$
\myspace{1}
\begin{solution}
	
	令$F(x)=\int_{0}^{x}f(t)dt,F'(x)=f(x)$,原微分方程等价于: 
	$$F'(x)+2F(x)=x^2\Rightarrow [e^{2x}F(x)]'=x^2e^{2x}$$
	$$e^{2x}F(x)=\int x^2e^{2x}dx\Rightarrow e^{2x}F(x)=\frac{x^2e^{2x}}{2}-\frac{xe^{2x}}{2}+\frac{e^{2x}}{4}+C$$
	我们得到: $$F(x)=\frac{1}{2}x^2-\frac{1}{2}x+\frac{1}{4}+Ce^{-2x}$$
	
	$f(x)=F'(x)=x-\frac{1}{2}-2Ce^{-2x},f(0)=-\frac{1}{2}-2C=0\Rightarrow C=-\frac{1}{4}$
	
	$$f(x)=\frac{1}{2}e^{-2x}+x-\frac{1}{2}$$
	
\end{solution}
\myspace{1}

\section{Week \Rmnum{2}}
\hl{\textbf{\textit{May 8}}}

1. 设连续函数$f(x)$满足$x\int_{0}^{1}f(tx)dt=f(x)+x$,求$f(x)$
\myspace{1}
\begin{solution}
	
	对于$x\int_{0}^{1}f(tx)dt$,令$u=tx,t=\dfrac{u}{x},dt=\dfrac{du}{x}$,原微分方程为: 
	$$\int_{0}^{x}f(u)du=f(x)+x$$
	
	令$F(x)=\int_{0}^{x}f(u)du=f(x)+x,F'(x)=f(x)$,原微分方程等价于: 
	$$F(x)-F'(x)=x\Rightarrow [e^{-x}F(x)]'=-xe^{-x}$$
	$$e^{-x}F(x)=\int(-xe^{-x})dx=(x+1)e^{-x}+C\Rightarrow F(x)=Ce^{x}+x+1$$
	$$f(x)=F'(x)=Ce^{x}+1,f(0)=0\Rightarrow C=-1$$
	$$f(x)=1-e^{x}$$
\end{solution}
\myspace{1}

2. 求抛物线$y=x^2$和直线$x-y-2=0$之间的最短距离
\myspace{1}
\begin{solution}
	
	方法一(不太严谨): 
	
	设抛物线上任意一点坐标$P(x_{0},y_{0})$,点$P$到直线$x-y-2=0$的距离为: 
	$$d=\dfrac{|x_{0}-x_{0}^2-2|}{\sqrt{2}}=\dfrac{(x_{0}-\frac{1}{2})^2+\frac{7}{4}}{\sqrt{2}}$$
	$$d_{min}=\dfrac{7\sqrt{2}}{8}$$
	
	方法二: 
	
	我们设抛物线上一点$P$坐标为$(x,y)$,直线$x-y-2=0$上一点$Q$坐标$(\alpha,\beta)$,我们得到$|PQ|^2=(x-\alpha)^2+(y-\beta)^2$.
	
	我们令$f(x,y,\alpha,\beta,\lambda,\mu)=(x-\alpha)^2+(y-\beta)^2+\lambda(y-x^2)+\mu(\alpha-\beta-2)$
	
	我们令: $$\left\lbrace 
	\begin{array}{l}
		f'_{x}=2(x-\alpha)-2\lambda x=0\\
		f'_{y}=2(y-\beta)+\lambda=0\\
		f'_{\alpha}=-2(x-\alpha)+\mu=0\\
		f'_{\beta}=-2(y-\beta)-\mu=0\\
		f'_{\lambda}=y-x^2=0\\
		f'_{\mu}=\alpha-\beta-2=0
	\end{array}
	\right. $$
	我们解得: $$\left\lbrace 
	\begin{array}{l}
		x=\dfrac{1}{2}\\
		y=\dfrac{1}{4}\\
		\alpha=\dfrac{11}{8}\\
		\beta=-\dfrac{5}{8}\\
		\lambda=-\dfrac{7}{4}\\
		\mu=-\dfrac{7}{4}
	\end{array}
	\right. $$
	$$f(x,y,\alpha,\beta,\lambda,\mu)_{min}=f(\dfrac{1}{2},\dfrac{1}{4},\dfrac{11}{8},-\dfrac{5}{8},-\dfrac{7}{4},-\dfrac{7}{4})=\dfrac{7\sqrt{2}}{8}$$
\end{solution}
\myspace{1}

\hl{\textbf{\textit{May 9}}}

1. 设$f(x)$二阶可导,$f(-x)=-f(x),f(x+1)=f(x),x\in(-\infty,+\infty)$,且$\lim\limits_{x\rightarrow 1}\dfrac{f(x)}{\sin(x-1)}=1$,则
\begin{itemize}
	\item A. $f''(99)\leq f'(100)\leq f(101)$ 
	\item \hl{B}. $f(99)=f(100)<f'(101)$ 
	\item C. $f'(99)\leq f(100)<f''(101)$ 
	\item D. $f(99)< f'(100)=f''(100)$ 
\end{itemize}
\myspace{1}
\begin{solution}
	
	由题意知道: $$f(x),f'(x),f''(x)\text{周期为}1,f(x),f''(x)\text{是奇函数},\text{且}f(1)=0$$
	$$\lim\limits_{x\rightarrow 1}\frac{f(x)}{\sin(x-1)}=\lim\limits_{x\rightarrow 1}\frac{f(x)-f(1)}{x-1}=f'(1)=1$$
	
	我们从而得到: $f(n)=0,f'(n)=1,f''(n)=0$
	
	此题答案选$B$
\end{solution}
\myspace{1}

2. 设连续函数$f(x)$满足$\int_{0}^{x}f(x-t)dt=\int_{0}^{x}(x-t)f(t)dt+e^{-x}-1$,求$f(x)$
\myspace{1}
\begin{solution}
	
	原微分方程等价于: 
	$$\int_{0}^{x}f(t)dt=x\int_{0}^{x}f(t)dt-\int_{0}^{x}tf(t)dt+e^{-x}-1$$
	
	对微分方程左右两边对$x$求导,得到: 
	$$f(x)=\int_{0}^{x}f(t)dt-e^{-x}$$
	
	我们令$F(x)=\int_{0}^{x}f(t)dt,f(x)=F'(x)$,上式等价于: 
	$$F(x)-F'(x)=e^{-x}\Rightarrow [e^{-x}F(x)]'=-e^{-2x}$$
	$$F(x)=Ce^{x}+\frac{1}{2}e^{-x}$$
	$$f(x)=F'(x)=Ce^{x}-\frac{1}{2}e^{-x},f(0)=-1\Rightarrow C=-\frac{1}{2}$$
	$$f(x)=-\frac{1}{2}e^{x}-\frac{1}{2}e^{-x}$$
\end{solution}
\myspace{1}

\hl{\textbf{\textit{May 10}}}

1.设连续函数$f(x)$满足$f(x)=\sin x-\int_{0}^{x}(x-t)f(t)dt$,求$f(x)$
\myspace{1}
\begin{solution}
	
	对微分方程左右两边对$x$求导,得到: 
	$$f'(x)=\cos x-\int_{0}^{x}f(t)dt$$
	
	令$F(x)=\int_{0}^{x}f(t)dt,f(x)=F'(x)$,上式等价于: 
	$$F''+F=\cos x$$
	
	特征方程为$r^2+1=0,r_{1}=i,\ r_{2}=-i$,方程通解为$F(x)=C_{1}\cos x+C_{2}\sin x$
	
	特解设为$F(x)=x(A\cos x+B\sin x)$,代入得到: 
	$$A=0,\ B=\frac{1}{2}$$
	$$F(x)=C_{1}\cos x+C_{2}\sin x+\frac{x}{2}\sin x$$
	$$f(x)=F'(x)=C_{2}\cos x-C_{1}\sin x+\frac{1}{2}\sin x+\frac{x}{2}\cos x\Rightarrow \left\lbrace 
	\begin{array}{l}
		f(0)=0\\
		f'(0)=1
	\end{array}
	\right. \Rightarrow \left\lbrace 
	\begin{array}{l}
		C_{1}=0\\
		C_{2}=0
	\end{array}
	\right. $$
	$$f(x)=\frac{1}{2}\sin x+\frac{x}{2}\cos x$$
\end{solution}
\myspace{1}

2. $\int_{0}^{\frac{\pi}{4}}\ln(1+\tan x)dx$ \label{problem: 区间再现}
\myspace{1}
\begin{solution}
	
	典型的区间再现
	
	$$I=\int_{0}^{\frac{\pi}{4}}\ln(1+\tan x)dx=\int_{0}^{\frac{\pi}{4}}\ln(1+\tan(\frac{\pi}{4}-x))dx$$
	$$I=\int_{0}^{\frac{\pi}{4}}(\ln2-\ln(1+\tan x))dx$$
	$$2I=\int_{0}^{\frac{\pi}{4}}\ln 2dx=\frac{\ln 2}{4}\pi$$
	$$I=\int_{0}^{\frac{\pi}{4}}\ln(1+\tan x)dx=\frac{\ln 2}{8}\pi$$
\end{solution}
\myspace{1}

\hl{\textbf{\textit{May 11}}}

1. 设$f(x)$是连续的正值函数,且单调减少,证明: $\dfrac{\int_{0}^{1}xf^{2}(x)dx}{\int_{0}^{1}xf(x)dx}\leq \dfrac{\int_{0}^{1}f^{2}(x)dx}{\int_{0}^{1}f(x)dx}$
\myspace{1}
\begin{solution}
	
	原命题等价于: 
	$$\int_{0}^{1}xf^{2}(x)dx\int_{0}^{1}f(y)dy-\int_{0}^{1}xf(x)dx\int_{0}^{1}f^{2}(y)dy\leq 0$$
	$$I=\iint\limits_{D}[xf(x)f(y)(f(x)-f(y))]dxdy,D=\{0\leq x\leq 1;\ 0\leq y\leq 1 \}$$
	
	积分区域关于$y=x$对称,我们交换$x,y$位置,得到
	$$2I=\iint\limits_{D}[f(x)f(y)(f(x)-f(y))(x-y)]dxdy,D=\{0\leq x\leq 1;\ 0\leq y\leq 1 \}$$
	
	我们知道$f(x)$单调递减,且$f(x)>0$,我们得到: 
	$$f(x)f(y)>0,\ (x-y)(f(x)-f(y))\leq 0\Rightarrow I\leq 0,\text{证毕}$$
\end{solution}
\myspace{1}

2. $\iint\limits_{D}\dfrac{1}{\arcsin\sqrt{x^2+y^2}}dxdy$,其中$D=\{(x,y)|x^2+y^2\leq 1,\ x^2+y^2\geq y,\ x\geq 0,\ y\geq 0\}$
\myspace{1}
\begin{solution}
	
	利用极坐标公式进行代换,$x=r\cos \theta,\ y=r\sin \theta$,我们得到: 
	$$I=\iint\limits_{D_{1}}\dfrac{r}{\arcsin r}drd\theta,\theta\in(0,\frac{\pi}{2}),\ r\in(\sin\theta,1)$$
	
	上面的积分可以化为: 
	$$I=\int_{0}^{\frac{\pi}{2}}d\theta\int_{\sin\theta}^{1}\dfrac{r}{\arcsin r}dr=\int_{0}^{1}dr\int_{0}^{\arcsin r}\frac{r}{\arcsin r}d\theta=\frac{1}{2}$$
	$$\iint\limits_{D}\dfrac{1}{\arcsin\sqrt{x^2+y^2}}dxdy=\frac{1}{2}$$
\end{solution}
\myspace{1}

\hl{\textbf{\textit{May 12}}}

1. 设$f(x)$在$[0,1]$连续,在$(0,1)$内可导,且$\int_{0}^{1}f(x)dx=1$,证明: $\exists \xi\neq \eta\in(0,1),\text{s.t.}\ f(\xi)+3f(\eta)=4f(\xi)f(\eta)$
\myspace{1}
\begin{solution}
	
	我们令$F(x)=\int_{0}^{x}f(t)dt$,我们有: 
	
	$$F(0)=0,\ F(1)=1,F(x)\text{在}[0,1]\text{连续},\text{在}(0,1)\text{可导}$$
	
	原命题转化为证明: 
	$$\exists \xi\neq \eta\in(0,1),\text{s.t.}\ F'(\xi)+3F'(\eta)=4F'(\xi)F'(\eta)$$
	
	上面的式子我们进行一些变形: 
	$$\exists \xi\neq \eta\in(0,1),\text{s.t.}\ \frac{1}{F'(\eta)}+\frac{3}{F'(\xi)}=4$$
	
	$F(x)\text{在}[0,1]\text{连续},F(0)=0,F(1)=1$,$\exists c\in(0,1)\text{s.t.}\ F(c)=\dfrac{1}{4}$.
	
	由拉格朗日中值定理我们得到: 
	$$\exists\xi\in(0,c),\eta\in(c,1), \text{s.t.}\left\lbrace 
	\begin{array}{l}
		\dfrac{F(c)-F(0)}{c}=F'(\xi)\\
		\dfrac{F(1)-F(c)}{1-c}=F'(\eta)
	\end{array}
	\right. \Rightarrow \left\lbrace 
	\begin{array}{l}
		\dfrac{1}{F'(\xi)}=4c\\
		\dfrac{1}{F'(\eta)}=\frac{4}{3}(1-c)
	\end{array}
	\right. $$
	$$\frac{1}{F'(\xi)}+\frac{3}{F'(\eta)}=4c+4(1-c)=4$$
	$$\exists \xi\neq \eta\in(0,1),\text{s.t.}\ f(\xi)+3f(\eta)=4f(\xi)f(\eta),\text{证毕}$$
\end{solution}
\myspace{1}

2. 函数$\dfrac{|x|e^{\frac{1}{x-1}}\ln(x-1)^2}{x(x-1)(x-2)}$在下列哪个区间内无界: 
\begin{itemize}
	\item A. $(-\infty,0)$ 
	\item B. $(0,1)$ 
	\item \hl{C}. $(1,2)$ 
	\item D. $(2,+\infty)$ 
\end{itemize}
\myspace{1}
\begin{solution}
	$$\lim\limits_{x\rightarrow 0^{-}}\frac{-2x\ln(1-x)}{2ex}=0,\ \lim\limits_{x\rightarrow 0^{+}}\frac{2x\ln(1-x)}{2ex}=0$$
	$$\lim\limits_{x\rightarrow 1^{-}}\frac{2e^{\frac{1}{x-1}}\ln(1-x)}{1-x}=0,\ \lim\limits_{x\rightarrow 1^{+}}\frac{2e^{\frac{1}{x-1}}\ln(x-1)}{1-x}=\infty$$
	$$\lim\limits_{x\rightarrow 2}\frac{2e\ln(x-1)}{x-2}=2e$$
	$$\lim\limits_{x\rightarrow +\infty}\frac{2\ln(x-1)}{(x-1)(x-2)}=0,\ \lim\limits_{x\rightarrow -\infty}\frac{-2\ln(1-)}{(x-1)(x-2)}=0$$
	
	答案:$C$
\end{solution}
\myspace{1}

\hl{\textbf{\textit{May 13}}}

1. 计算极限$\lim\limits_{\substack{x\rightarrow 0\\ y\rightarrow 0}}\dfrac{\sqrt{1+x^2+y^2}-1}{x^2+y^2}$和$\lim\limits_{\substack{x\rightarrow +\infty\\ y\rightarrow 1}}\left( 1+\dfrac{1}{xy}\right)^{\frac{y^2}{\sin\frac{2}{x}}} $
\myspace{1}
\begin{solution}
	$$I_{1}=\lim\limits_{\substack{x\rightarrow 0\\ y\rightarrow 0}}\dfrac{1+x^2+y^2-1}{(x^2+y^2)(\sqrt{1+x^2+y^2}+1)}=\lim\limits_{\substack{x\rightarrow 0\\ y\rightarrow 0}}\dfrac{1}{\sqrt{1+x^2+y^2}+1}=\frac{1}{2}$$
	
	$$I_{2}=\lim\limits_{\substack{x\rightarrow +\infty\\ y\rightarrow 1}}e^{\frac{y^2}{\sin\frac{2}{x}}}\ln(1+\frac{1}{xy})$$
	$$I_{2}=\lim\limits_{\substack{x\rightarrow +\infty\\ y\rightarrow 1}}e^{\frac{xy^2}{2}\frac{1}{xy}}=e^{\frac{1}{2}}$$
\end{solution}
\myspace{1}

2. $\int_{0}^{\pi}\left( e^{-\cos x}-e^{\cos x}\right)dx $
\myspace{1}
\begin{solution}
	
	区间再现
	
	原积分等价于: 
	$$I=\int_{0}^{\pi}\left( e^{\cos x}-e^{-\cos x}\right)dx$$
	$$2I=0\Rightarrow I=0$$
\end{solution}
\myspace{1}

\hl{\textbf{\textit{May 14}}}

1. $\int_{-\frac{\pi}{2}}^{\frac{\pi}{2}}\left( \arctan e^{x}\right)\sin^{2}xdx $
\myspace{1}
\begin{solution}
	
	区间再现
	
	原积分等价于: 
	$$I=\int_{-\frac{\pi}{2}}^{\frac{\pi}{2}}(\frac{\pi}{2}-\arctan e^{x})\sin^{2}xdx$$
	$$2I=\frac{\pi}{2}\int_{-\frac{\pi}{2}}^{\frac{\pi}{2}}\sin^{2}xdx=\frac{\pi^{2}}{4}$$
	$$I=\int_{-\frac{\pi}{2}}^{\frac{\pi}{2}}\left( \arctan e^{x}\right)\sin^{2}xdx=\frac{\pi^{2}}{8}$$
\end{solution}
\myspace{1}

2. 设$f(x)=\int_{0}^{x}f(x-t)\sin tdt+x$,求$f(x)$
\myspace{1}
\begin{solution}
	
	我们有: $\int_{0}^{x}f(x-t)\sin tdt=\sin x\int_{0}^{x}f(t)\cos tdt-\cos x\int_{0}^{x}f(t)\sin tdt$
	
	原微分方程等价于: 
	$$f(x)=\sin x\int_{0}^{x}f(t)\cos tdt-\cos x\int_{0}^{x}f(t)\sin tdt+x$$
	
	两边对$x$求导得到: 
	$$f'(x)=1+\cos x\int_{0}^{x}f(t)\cos tdt+\sin x\int_{0}^{x}f(t)\sin tdt$$
	
	再次两边对$x$求导得到: 
	$$f''(x)=f(x)+\cos x\int_{0}^{x}f(t)\sin tdt-\sin x\int_{0}^{x}f(t)\cos tdt\Rightarrow f''(x)=f(x)+x-f(x)$$
	
	$$f''(x)=x\Rightarrow f'(x)=\frac{1}{2}x^2+C_{1}\Rightarrow f(x)=\frac{1}{6}x^3+C_{1}x+C_{2}$$
	
	我们有: $f(0)=0,\ f'(0)=1\Rightarrow C_{1}=1,\ C_{2}=0$
	
	$$f(x)=\frac{1}{6}x^3+x$$
\end{solution}
\myspace{1}

\section{Week \Rmnum{3}}
\hl{\textbf{\textit{May 15}}}

1. 下列函数在区间$(0,1)$内无界的是: 
\begin{itemize}
	\item A. $\int_{0}^{x}\dfrac{1}{t^2}e^{-\frac{1}{t}}dt$ 
	\item B. $\int_{0}^{x}\dfrac{\sin t}{t}dt$ 
	\item \hl{C}. $\int_{0}^{x}\dfrac{1}{\sqrt{(1-t)^3}}dt$ 
	\item D. $\int_{0}^{x}\dfrac{1}{(t-1)^2}\sin\dfrac{1}{t-1}dt$ 
\end{itemize}
\myspace{1}
\begin{solution}
	
	$A \quad f(x)=\int_{0}^{x}\frac{1}{t^2}e^{-\frac{1}{t}}dt=e^{-\frac{1}{x}}$
	
	$B \quad \lim\limits_{x\rightarrow 0}\dfrac{\sin x}{x}=1$
	
	$C \quad f(x)=\int_{0}^{x}\dfrac{1}{\sqrt{(1-t)^3}}dt=\dfrac{2}{\sqrt{1-x}}-2, \ f(1)\rightarrow+\infty$
	
	$D \quad f(x)=\int_{0}^{x}\dfrac{1}{(t-1)^2}\sin\dfrac{1}{t-1}dt=\cos \dfrac{1}{x-1}+\dfrac{\pi}{2}\  \text{有界}$
	
	答案:$C$
\end{solution}
\myspace{1}

2. $I=\iint\limits_{D}\dfrac{x^3\sin y\cos y e^{\sqrt{x^2+2}}}{\sqrt{x^2\cos^2y+2}\sqrt{x^2+2}}dxdy,\ \text{其中}D=\{(x,y)|\ 0\leq x\leq 1,\ 0\leq y\leq \dfrac{\pi}{2}\}$
\myspace{1}
\begin{solution}
	
	原二重积分可以化为: 
	\begin{eqnarray*}
		I&=&\iint\limits_{D'}\dfrac{r^2\sin \theta\cos \theta e^{\sqrt{r^2+2}}}{\sqrt{r^2\cos^2\theta+2}\sqrt{r^2+2}}rdrd\theta,\ \text{其中}D'=\{(r,\theta)|\ 0\leq r\leq 1,\ \theta\in(0,\dfrac{\pi}{2})\}\\
		&=&\iint\limits_{D''}\dfrac{xye^{\sqrt{x^2+y^2+2}}}{\sqrt{x^2+2}\sqrt{x^2+y^2+2}}dxdy\\
		&=&\int_{0}^{1}\dfrac{x}{\sqrt{x^2+2}}dx\int_{0}^{\sqrt{1-x^2}}\dfrac{ye^{\sqrt{x^2+y^2+2}}}{\sqrt{x^2+y^2+2}}dy\\
		&=&e^{\sqrt{3}}\int_{0}^{1}\dfrac{x}{\sqrt{x^2+2}}dx-\int_{0}^{1}\dfrac{xe^{\sqrt{x^2+2}}}{\sqrt{x^2+2}}dx\\
		&=&e^{\sqrt{3}}(\sqrt{3}-\sqrt{2})-e^{\sqrt{3}}+e^{\sqrt{2}}
	\end{eqnarray*}
\end{solution}
\myspace{1}

\hl{\textbf{\textit{May 16}}}

1. 已知函数$f(x)=\dfrac{\int_{0}^{x}\ln(1+t^2)dt}{x^{\alpha}}$在$(0,+\infty)$上有界,则$\alpha$的取值范围为: 
\begin{itemize}
	\item A. $(0,+\infty)$ 
	\item B. $(0,3]$ 
	\item C. $(0,2)$ 
	\item \hl{D}. $(1,3]$ 
\end{itemize}
\myspace{1}
\begin{solution}
	
	$f(x)$在$(0,+\infty)$上有界,我们可以得到: 
	$$\lim\limits_{x\rightarrow +\infty}f(x)\text{和}\lim\limits_{x\rightarrow 0}f(x)\text{存在}$$
	$$\lim\limits_{x\rightarrow +\infty}f(x)=\lim\limits_{x\rightarrow +\infty}\dfrac{\ln(1+x^2)}{\alpha x^{\alpha-1}}=\lim\limits_{x\rightarrow +\infty}\dfrac{2x^{3-\alpha}}{\alpha(\alpha-1)(1+x^2)}$$
	$$\lim\limits_{x\rightarrow 0}f(x)=\lim\limits_{x\rightarrow 0}\dfrac{\ln(1+x^2)}{\alpha x^{\alpha-1}}=\lim\limits_{x\rightarrow 0}\dfrac{2x^{3-\alpha}}{\alpha(\alpha-1)}$$
	
	我们得到: 
	$$\left\lbrace 
	\begin{array}{l}
		3-\alpha\geq 0\\
		3-\alpha\leq 2\\
		\alpha\neq 1
	\end{array}
	\right. \Rightarrow 1<\alpha\leq 3$$
	
	综上所述,答案:$D$
\end{solution}
\myspace{1}

2. $\int_{0}^{\frac{\pi}{4}}\dfrac{x}{\cos(\frac{\pi}{4}-x)\cos x}dx$
\myspace{1}
\begin{solution}
	
	区间再现
	
	$$I=\int_{0}^{\frac{\pi}{4}}\dfrac{\frac{\pi}{4}-x}{\cos(\frac{\pi}{4}-x)\cos x}dx$$
	$$2I=\int_{0}^{\frac{\pi}{4}}\dfrac{\frac{\pi}{4}}{\cos(\frac{\pi}{4}-x)\cos x}dx=\frac{\sqrt{2}\pi}{4}\int_{0}^{\frac{\pi}{4}}\frac{dx}{\cos^2x+\sin x\cos x}=\frac{\sqrt{2}\pi}{4}\int_{0}^{\frac{\pi}{4}}\frac{d\tan x}{1+\tan x}=\frac{\ln 2\sqrt{2}\pi}{4}$$
	$$I=\int_{0}^{\frac{\pi}{4}}\dfrac{x}{\cos(\frac{\pi}{4}-x)\cos x}dx=\frac{\ln 2\sqrt{2}\pi}{8}$$
\end{solution}
\myspace{1}

\hl{\textbf{\textit{May 17}}}

1. $\text{计算二重积分}\iint\limits_{D}\dfrac{1}{xy}dxdy,\ D=\{(r,\theta)|\dfrac{\cos \theta}{4}\leq r\leq \dfrac{\cos\theta}{2}, \dfrac{\sin \theta}{4}\leq r\leq \dfrac{\sin\theta}{2}\}$
\myspace{1}
\begin{solution}
	
	二重积分积分区域如下图所示: 

	\begin{figure}[ht]
		\centering
		\begin{tikzpicture}[scale=25]
			\begin{scope}
				\clip (0,0) arc (-90:0:1/8) arc (90:180:1/8);
				\fill[pattern=horizontal lines]
				(0,0) arc (-90:0:1/8) --(1/16,1/16) arc (90:0:1/16);
				\fill[pattern=vertical lines]
				(0,0) arc (180:90:1/8) --(1/16,1/16) arc (0:90:1/16);
			\end{scope}
			\draw [->,thick](-0.05,0)--(0.3,0);\draw [->,thick](0,-0.05)--(0,0.3);
			\node[right]at(0.3,0){$x$};\node[left]at(0,0.3){$y$};
			\draw (0,0) arc (-90:90:1/8) (0,0) arc (-90:90:1/16)
			(0,0) arc (180:0:1/8) (0,0) arc (180:0:1/16) (0,0)--(1/8,1/8);
			\node [below]at(0.069,0.0627){$A$};
			\node [above]at(0.136,0.123){$B$};
			\node [right]at(0.105,0.0541){$C$};
			\node [below]at(0.069,0){$O_{1}$};
			\node [below]at(0.136,0){$O_{2}$};
			\node [left]at(0,0.0627){$O_{3}$};
			\node [left]at(0,0.123){$O_{4}$};
		\end{tikzpicture}
		\caption{二重积分区域示意图}
	\end{figure}

	$$I=2\iint_{D_{ACB}}\frac{1}{xy}dxdy=2\int_{\arctan \frac{1}{2}}^{\frac{\pi}{4}}d\theta\int_{\frac{\cos \theta}{4}}^{\frac{\sin\theta}{2}}\frac{1}{r\sin\theta\cos\theta}dr$$
	$$I=2\int_{\arctan \frac{1}{2}}^{\frac{\pi}{4}}\frac{\ln2\tan \theta}{\sin\theta\cos\theta}d\theta=2\int_{ \frac{1}{2}}^{1}\frac{\ln2t}{t}dt=\ln^2 2$$
\end{solution}
\myspace{1}

2. $\text{当}n\text{充分大时},a-\dfrac{1}{n}<a_{n}<a+\dfrac{1}{n}\text{是数列}a_{n}\text{收敛于}a\text{的什么条件 ?}$
\begin{itemize}
	\item A. $\text{充分必要条件}$
	\item B. $\text{必要条件但非充分条件}$
	\item \hl{C}. $\text{充分条件但非必要条件}$
	\item D. $\text{既非充分也非必要条件}$
\end{itemize}
\myspace{1}
\begin{solution}
	
	(i).充分性: 
	
	$a-\dfrac{1}{n}<a_{n}<a+\dfrac{1}{n},\text{由夹逼定理得到: }$
	$$\lim\limits_{n\rightarrow +\infty}(a-\frac{1}{n})<\lim\limits_{n\rightarrow +\infty}a_{n}<\lim\limits_{n\rightarrow +\infty}(a+\frac{1}{n})$$
	
	我们有: $\lim\limits_{n\rightarrow +\infty}(a-\frac{1}{n})=\lim\limits_{n\rightarrow +\infty}(a+\frac{1}{n})=a$
	因此: 
	$$\lim\limits_{n\rightarrow +\infty}a_{n}=a\Rightarrow\text{充分性成立}$$
	
	(ii).必要性
	
	$\lim\limits_{n\rightarrow +\infty}a_{n}=a\Rightarrow \forall \varepsilon>0,\ \exists N_{0}>0,\text{当}n>N_{0},|a_{n}-a|<\varepsilon$
	
	我们得到: 
	$$n>N_{0},a-\varepsilon<a_{n}<a+\varepsilon,\text{当}\frac{1}{N_{0}+1}<\varepsilon,\text{我们不能得到}a-\frac{1}{n}<a_{n}<a+\frac{1}{n},\text{必要性不成立}$$
	
	答案:$C$
\end{solution}
\myspace{1}

\hl{\textbf{\textit{May 18}}}

1. $\iint\limits_{D}|x^2+y^2-\sqrt{2}(x+y)|dxdy,\ D=\{(x,y)|x^2+y^2\leq 4\}$
\myspace{1}
\begin{solution}
	
	原二重积分等价于: 
	$$I=I_{1}+2I_{2}=\iint\limits_{D}[x^2+y^2-\sqrt{2}(x+y)]dxdy+2\iint\limits_{D_{1}}[\sqrt{2}(x+y)-(x^2+y^2)]dxdy$$
	$$I_{1}=\int_{0}^{2\pi}d\theta\int_{0}^{2}r(r^2-\sqrt{2}r(\sin \theta+\cos \theta))dr=\int_{0}^{2\pi}(4-\frac{8\sqrt{2}}{3}(\sin\theta+\cos\theta))d\theta=8\pi$$
	$$I_{2}=\iint\limits_{D_{1}}[\sqrt{2}(x+y)-(x^2+y^2)]dxdy,D_{1}: (x-\frac{\sqrt{2}}{2})^2+(y-\frac{\sqrt{2}}{2})^2\leq 1$$
	
	做变量替换: $\left\lbrace 
	\begin{array}{l}
		x-\frac{\sqrt{2}}{2}=R\cos \alpha\\
		y-\frac{\sqrt{2}}{2}=R\sin \alpha
	\end{array}
	\right. \Rightarrow dxdy=\left| \dfrac{\partial (x,y)}{\partial (R,\alpha)}\right| dRd\alpha=RdRd\alpha$
	
	$$I_{2}=\int_{0}^{2\pi}d\alpha\int_{0}^{1} (1-R^2)RdR=\frac{\pi}{2}$$
	$$I=I_{1}+2I_{2}=9\pi$$
\end{solution}
\myspace{1}

2. 设$f(x)$可积,则下列结论正确的是: 
\begin{itemize}
	\item A.$\text{如果}\lim\limits_{n\rightarrow +\infty}x_{n}=0,\text{且}\lim\limits_{n\rightarrow +\infty}f(x_{n})=A,\text{则}\lim\limits_{x\rightarrow 0}f(x)=A$
	\item B.$\text{如果}\lim\limits_{n\rightarrow +\infty}x_{n}=0,\text{且}\lim\limits_{x\rightarrow 0}f(x)=A,\text{则}\lim\limits_{n\rightarrow +\infty}f(x_{n})=A$
	\item C.$\text{如果}\lim\limits_{n\rightarrow +\infty}f(n)=A,\text{则}\lim\limits_{x\rightarrow +\infty}f(x)=A$
	\item \hl{D}.$\text{如果}\lim\limits_{n\rightarrow +\infty}\int_{0}^{n}\dfrac{|f(t)|}{1+t^2}dt=A,\text{则}\lim\limits_{x\rightarrow +\infty}\int_{0}^{x}\dfrac{|f(t)|}{1+t^2}dt=A$
\end{itemize}
\myspace{1}
\begin{solution}
	
	针对于此题,我们需要区分几个概念: 
	
	$$\lim\limits_{x\rightarrow x_{0}}f(x)=A, \text{只要求在}x\text{邻域内有定义,不要求}x\text{处有定义}$$
	$$\lim\limits_{n\rightarrow+\infty}f(n)=A,f(x)\text{单调},\Rightarrow \lim\limits_{x\rightarrow+\infty}f(x)=A$$
	
	对于$A$,我们只能得到在$x=0$的邻域内的一组特殊的离散点满足极限的定义,其余点未知,我们举一个反例,$x_{n}=\dfrac{1}{n\pi},f(x)=\sin\dfrac{1}{x}$
	
	对于$B$,我们举出一个反例$f(x)=\left\lbrace 
	\begin{array}{l}
		\dfrac{\sin x}{x},x\neq 0\\
		2
	\end{array}
	\right. $,$x_{n}=\left\lbrace 
	\begin{array}{l}
		\dfrac{1}{n},n\text{为奇数}\\
		0,n\text{为偶数}
	\end{array}
	\right. $
	
	对于$C$,不清楚$f(x)$单调性,举一个反例,$f(x)=\sin \pi x$
	
	对于$D$,利用积分的几何意义知道$g(x)=\int_{0}^{x}\dfrac{|f(t)|}{1+t^2}dt$单调递增
	
	答案:$D$.
\end{solution}
\myspace{1}

\hl{\textbf{\textit{May 19}}}

1. 设$f(u,v)$具有二阶连续偏导数,且满足$\dfrac{\partial^2 f}{\partial u^2}+\dfrac{\partial^2 f}{\partial v^2}=1$,又$g(x,y)=f(xy,\frac{1}{2}(x^2-y^2))$,求$\dfrac{\partial^2 g}{\partial x^2}+\dfrac{\partial^2 g}{\partial y^2}$
\myspace{1}
\begin{solution}
	
	我们令$\left\lbrace 
	\begin{array}{l}
		u=xy\\
		v=\dfrac{1}{2}(x^2-y^2)
	\end{array}
	\right. $
	
	我们有: 
	
	$$\left\lbrace 
	\begin{array}{l}
		\dfrac{\partial g}{\partial x}=\dfrac{\partial f}{\partial u}\dfrac{\partial u}{\partial x}+\dfrac{\partial f}{\partial v}\dfrac{\partial v}{\partial x}=y\dfrac{\partial f}{\partial u}+x\dfrac{\partial f}{\partial v}\\
		
		\\
		\dfrac{\partial g}{\partial y}=\dfrac{\partial f}{\partial u}\dfrac{\partial u}{\partial y}+\dfrac{\partial f}{\partial v}\dfrac{\partial v}{\partial y}=x\dfrac{\partial f}{\partial u}-y\dfrac{\partial f}{\partial v}
	\end{array}
	\right. $$
	$$\left\lbrace 
	\begin{array}{l}
		\dfrac{\partial^2 g}{\partial x^2}=y(y\dfrac{\partial^2 f}{\partial u^2}+x\dfrac{\partial^2 f}{\partial u\partial v})+\dfrac{\partial f}{\partial v}+x(y\dfrac{\partial^2 f}{\partial u\partial v}+x\dfrac{\partial^2 f}{\partial v^2})\\
		
		\\
		\dfrac{\partial^2 g}{\partial y^2}=x(x\dfrac{\partial^2 f}{\partial u^2}-y\dfrac{\partial^2 f}{\partial u\partial v})-\dfrac{\partial f}{\partial v}-y(x\dfrac{\partial^2 f}{\partial u\partial v}-y\dfrac{\partial^2 f}{\partial v^2})
	\end{array}
	\right. $$
	
	我们得到: 
	$$\dfrac{\partial^2 g}{\partial x^2}+\dfrac{\partial^2 g}{\partial y^2}=(x^2+y^2)\dfrac{\partial^2 f}{\partial u^2}+\dfrac{\partial^2 f}{\partial v^2}=x^2+y^2$$
\end{solution}
\myspace{1}

2. $\int_{-\pi}^{\pi}\dfrac{x\sin x\arctan e^{x}}{1+\cos ^2 x}dx$
\myspace{1}
\begin{solution}
	
	区间再现
	
	$$I=\int_{-\pi}^{\pi}\dfrac{x\sin x(\dfrac{\pi}{2}-\arctan e^{x})}{1+\cos ^2 x}dx$$
	$$2I=\dfrac{\pi}{2}\int_{-\pi}^{\pi}\dfrac{x\sin x}{1+\cos ^2 x}dx=\pi\int_{0}^{\pi}\dfrac{x\sin x}{1+\cos ^2 x}dx$$
	
	令$J=\int_{0}^{\pi}\dfrac{x\sin x}{1+\cos ^2 x}dx$
	
	$$J=\int_{0}^{\pi}\dfrac{(\pi-x)\sin x}{1+\cos ^2 x}dx\Rightarrow 2J=\pi\int_{0}^{\pi}\dfrac{\sin x}{1+\cos ^2 x}dx=\dfrac{\pi^2}{2}$$
	
	$$J=\dfrac{\pi^2}{4},I=\dfrac{\pi}{2}J=\dfrac{\pi^3}{8}$$
\end{solution}
\myspace{1}

\hl{\textbf{\textit{May 20}}}

1. 求$\iint\limits_{D}\dfrac{1}{3x^2+y^2}dxdy,D\text{是由}x^2+y^2-xy=1,\ x^2+y^2-xy=2\text{和直线}y=\sqrt{3}x,\ y=0\text{围成}$
\myspace{1}
\begin{solution}
	
	原二重积分等价于: 
	$$I=\int_{0}^{\frac{\pi}{3}}d\theta\int_{r_{1}}^{r_{2}}\dfrac{r}{3r^2\cos ^2\theta+r^2\sin^2\theta}dr,\text{其中}\left\lbrace 
	\begin{array}{l}
		r_{1}=\sqrt{\dfrac{1}{1-\sin \theta\cos\theta}}\\
		r_{2}=\sqrt{\dfrac{2}{1-\sin \theta\cos\theta}}
	\end{array}
	\right. $$
	
	$$I=\frac{\ln 2}{2}\int_{0}^{\frac{\pi}{3}}\dfrac{1}{\sin^2\theta+3\cos^2\theta}d\theta=\frac{\ln 2}{2}\int_{0}^{\frac{\pi}{3}}\dfrac{1}{3+\tan^2\theta}d\tan \theta$$
	$$I=\frac{\ln 2}{2\sqrt{3}}\arctan(\frac{\tan\theta}{\sqrt{3}})|_{0}^{\sqrt{3}}=\frac{\pi \ln 2}{8\sqrt{3}}$$
\end{solution}
\myspace{1}

2. $f(x)\text{在}[0,1]\text{上连续},\forall x,y\in \mathbb{R} ,|f(x)-f(y)|\leq M|x-y|,\text{求证: }\left| \int_{0}^{1}f(x)dx-\dfrac{1}{n}\sum\limits_{k=1}^{n}f(\dfrac{k}{n})\right|\leq \dfrac{M}{2n} $
\myspace{1}
\begin{solution}
	\begin{lemma}[积分和求和]
		
		$$\int_{0}^{1}f(x)dx=\sum\limits_{k=1}^{n}\int_{\frac{k-1}{n}}^{\frac{k}{n}}f(x)dx$$
	\end{lemma}
	我们利用上面的式子,对不等式的左边进行处理: 
	$$\text{左边: }\left| \sum\limits_{k=1}^{n}\int_{\frac{k-1}{n}}^{\frac{k}{n}}f(x)dx-\sum\limits_{k=1}^{n}\int_{\frac{k-1}{n}}^{\frac{k}{n}}f(\frac{k}{n})dx\right|=\left|\sum\limits_{k=1}^{n}\int_{\frac{k-1}{n}}^{\frac{k}{n}}[f(x)-f(\frac{k}{n})]dx\right|$$
	
	由绝对值不等式得到: 
	$$\text{左边}\leq \sum\limits_{k=1}^{n}\int_{\frac{k-1}{n}}^{\frac{k}{n}}\left|  [f(x)-f(\frac{k}{n})]\right|dx\leq  \sum\limits_{k=1}^{n}\int_{\frac{k-1}{n}}^{\frac{k}{n}}M(\frac{k}{n}-x)dx$$
	$$\text{左边}\leq M\sum\limits_{k=1}^{n}\frac{1}{2n^2}=\dfrac{M}{2n}=\text{右边} $$
\end{solution}
\myspace{1}

\hl{\textbf{\textit{May 21}}}

1. 已知数列$\{x_{n}\}$,且$\lim\limits_{n\rightarrow +\infty}\dfrac{x_{n}}{x_{n+1}}=\int_{0}^{1}e^{-x^2}dx$,则: 
\begin{itemize}
	\item A.$\lim\limits_{n\rightarrow +\infty}=0$
	\item \hl{B}.$\lim\limits_{n\rightarrow +\infty}=\infty$
	\item C.$\lim\limits_{n\rightarrow +\infty}=a\ (a\neq 0)$
	\item D.$\lim\limits_{n\rightarrow +\infty}\text{不存在但不是}\infty$
\end{itemize}
\myspace{1}
\begin{solution}
	
	首先我们可以得到: 
	$$\int_{0}^{1}e^{-x^2}dx=a,\ a\in(0,1)$$
	
	我们由$\lim\limits_{n\rightarrow +\infty}\frac{x_{n}}{x_{n+1}}=a\Rightarrow |x_{n}|=a|x_{n+1}|<|x_{n+1}|\Rightarrow x_{n}\text{单调递增}$
	
	我们假设$x_{n}$有上界
	$$\lim\limits_{n\rightarrow +\infty}x_{n}=\lim\limits_{n\rightarrow +\infty}x_{n+1}\Rightarrow \lim\limits_{n\rightarrow +\infty}\frac{x_{n}}{x_{n+1}}=1\neq a\text{矛盾!!!}$$
	
	答案:$B$
\end{solution}
\myspace{1}

2. 设$f(x,y)\text{二阶偏导数连续},f(1,y)=f(x,1)=0,\iint\limits_{D}f(x,y)dxdy=a$,其中$D\in [0,1]\times [0,1]$,计算$\iint\limits_{D}xyf''_{xy}(x,y)d\sigma$
\myspace{1}
\begin{solution}
	
	原二重积分可以化为: 
	\begin{eqnarray*}
		\int_{0}^{1}xdx\int_{0}^{1}ydf'_{x}(x,y)&=&\int_{0}^{1}x\left[yf'_{x}(x,y)|_{y=0}^{y=1}-\int_{0}^{1}f'_{x}(x,y)dy\right]dx\\
		&=&-\int_{0}^{1}xdx\int_{0}^{1}f'_{x}(x,y)dy\\
		&=&-\int_{0}^{1}dy\int_{0}^{1}xdf(x,y)\\
		&=&-\int_{0}^{1}\left[xf(x,y)|_{x=0}^{x=1}-\int_{0}^{1}f(x,y)dx\right]dy\\
		&=&\int_{0}^{1}dx\int_{0}^{1}f(x,y)dy=a
	\end{eqnarray*}
	
	$$\iint\limits_{D}xyf''_{xy}(x,y)d\sigma=\int_{0}^{1}dx\int_{0}^{1}f(x,y)dy=a$$
\end{solution}
\myspace{1}

\section{Week \Rmnum{4}}
\hl{\textbf{\textit{May 22}}}

1. 设$a_{n}=\int_{0}^{1}x^{n}\sqrt{1-x^2}dx\ (n=0,1,2,\cdots)$

(1).$\text{证明: 数列}\{a_{n}\}\text{单调减少,且}a_{n}=\dfrac{n-1}{n+2}a_{n-2},\ (n=2,3,\cdots)$

(2). $\text{求}\lim\limits_{n\rightarrow \infty}\dfrac{a_{n}}{a_{n-1}}$
\myspace{1}
\begin{solution}
	
	(1).
	我们令$x=\sin t,\ t\in[0,\frac{\pi}{2}],\ dx=\cos tdt$,我们有: 
	$$a_{n}=\int_{0}^{\frac{\pi}{2}}\sin^{n}t\cos^{2} tdt=\int_{0}^{\frac{\pi}{2}}\sin^{n}tdt-\int_{0}^{\frac{\pi}{2}}\sin^{n+2}tdt$$
	$$a_{0}=\frac{\pi}{4},\ a_{1}=\frac{1}{3}$$
	
	根据华里士公式,我们有: 
	$$\int_{0}^{\frac{\pi}{2}}\sin^{n}xdx=\left\lbrace 
	\begin{array}{l}
		\dfrac{n-1}{n}\cdot\dfrac{n-3}{n-2}\cdots\dfrac{2}{3}\cdot 1,n\text{为大于1的奇数}\\
		\dfrac{n-1}{n}\cdot\dfrac{n-3}{n-2}\cdots\dfrac{1}{2}\cdot\dfrac{\pi}{2},n\text{为正偶数}
	\end{array}
	\right. $$
	$$a_{n}-a_{n-1}=\int_{0}^{\frac{\pi}{2}}\sin^{n}tdt-\int_{0}^{\frac{\pi}{2}}\sin^{n+2}tdt-\int_{0}^{\frac{\pi}{2}}\sin^{n-1}tdt+\int_{0}^{\frac{\pi}{2}}\sin^{n+1}tdt$$
	$$a_{n}-a_{n-1}=\int_{0}^{\frac{\pi}{2}}-(1-\sin t)^2\sin^{n-1}tdt<0\Rightarrow a_{n}<a_{n-1}\Rightarrow \{a_{n}\}\text{单调递减}$$
	
	(i). 当$n$为奇数时,我们有: 
	$$\left\lbrace 
	\begin{array}{l}
		a_{n}=(1-\dfrac{n+1}{n+2})\dfrac{n-1}{n}\cdot\dfrac{n-3}{n-2}\cdots\dfrac{2}{3}\cdot 1\\
		a_{n-2}=(1-\dfrac{n-1}{n})\dfrac{n-3}{n-2}\cdots\dfrac{2}{3}\cdot 1
	\end{array}
	\right. $$
	$$\left\lbrace 
	\begin{array}{l}
		a_{n}=\dfrac{n-1}{n+2}\cdot\dfrac{1}{n}\cdot\dfrac{n-3}{n-2}\cdots\dfrac{2}{3}\cdot 1\\
		a_{n-2}=\dfrac{1}{n}\cdot\dfrac{n-3}{n-2}\cdots\dfrac{2}{3}\cdot 1
	\end{array}
	\right. 
	$$
	$$a_{n}=\dfrac{n-1}{n+2}a_{n-2}$$
	
	(i). 当$n$为偶数数时,我们有: 
	$$\left\lbrace 
	\begin{array}{l}
		a_{n}=(1-\dfrac{n+1}{n+2})\dfrac{n-1}{n}\cdot\dfrac{n-3}{n-2}\cdots\dfrac{1}{2}\cdot\dfrac{\pi}{2}\\
		a_{n-2}=(1-\dfrac{n-1}{n})\dfrac{n-3}{n-2}\cdots\dfrac{1}{2}\cdot\dfrac{\pi}{2}
	\end{array}
	\right. $$
	$$\left\lbrace 
	\begin{array}{l}
		a_{n}=\dfrac{n-1}{n+2}\cdot\dfrac{1}{n}\cdot\dfrac{n-3}{n-2}\cdots\dfrac{1}{2}\cdot\dfrac{\pi}{2}\\
		a_{n-2}=\dfrac{1}{n}\cdot\dfrac{n-3}{n-2}\cdots\dfrac{1}{2}\cdot\dfrac{\pi}{2}
	\end{array}
	\right. 
	$$
	$$a_{n}=\frac{n-1}{n+2}a_{n-2}$$
	
	综上所述,$\text{数列}\{a_{n}\}\text{单调减少,且}a_{n}=\dfrac{n-1}{n+2}a_{n-2},\ (n=2,3,\cdots)$
	
	(2). 我们不妨设$b_{n}=\dfrac{a_{n}}{a_{n-1}},\ (n=1,2,\cdots)$
	
	$b_{1}=\dfrac{4}{3\pi}$,当$n\geq 2$是,我们有: 
	
	$$b_{n}b_{n-1}=\dfrac{a_{n}}{a_{n-1}}\cdot\dfrac{a_{n-1}}{a_{n-2}}=\dfrac{n-1}{n+2}$$
	
	不妨设$\lim\limits_{n\rightarrow \infty}b_{n}=A$
	$$ A^2=\lim\limits_{n\rightarrow \infty}\dfrac{n-1}{n+2}=1\Rightarrow A=1(A>0)$$
	
	我们来严格证明$\lim\limits_{n\rightarrow \infty}\dfrac{a_{n}}{a_{n-1}}=1$
	
	原命题转换为: 
	
	$$\forall \varepsilon>0,\ \exists N>0,\text{当}n>N\text{时},|\dfrac{a_{n}}{a_{n-1}}-1|<\epsilon$$
	$$\int_{0}^{1}(1-x)x^{n-1}\sqrt{1-x^2}dx<\varepsilon \int_{0}^{1}x^{n-1}\sqrt{1-x^2}dx\Rightarrow \int_{0}^{1}(1-x-\varepsilon)x^{n-1}\sqrt{1-x^2}dx<0$$
	$$\int_{0}^{1-\varepsilon}(1-x-\varepsilon)x^{n-1}\sqrt{1-x^2}dx<\int_{1-\varepsilon}^{1}(x+\varepsilon-1)x^{n-1}\sqrt{1-x^2}dx$$
	
	$$\text{左式}<(1-\varepsilon)^{n-1}\int_{0}^{1}x^{n-1}\sqrt{1-x^2}dx=P(1-\varepsilon)^{n-1}$$
	$$\text{右式}>\int_{1-\frac{\varepsilon}{2}}^{1}(x+\varepsilon-1)x^{n-1}\sqrt{1-x^2}dx>(1-\frac{\varepsilon}{2})^{n-1}\int_{1-\frac{\varepsilon}{2}}^{1}(x+\varepsilon-1)\sqrt{1-x^2}dx>Q(1-\frac{\varepsilon}{2})^{n-1}$$
	
	其中$P,Q$均为与$n$无关的正数,我们只需找到$N$,使得$Q(1-\dfrac{\varepsilon}{2})^{n-1}>P(1-\varepsilon)^{n-1}$
	$$n>[1+\dfrac{\ln P-\ln Q}{\ln(1-\frac{\varepsilon}{2})-\ln(1-\varepsilon)}]$$
	
	综上,$\forall \varepsilon>0,\ \exists N=[1+\dfrac{\ln P-\ln Q}{\ln(1-\frac{\varepsilon}{2})-\ln(1-\varepsilon)}]+1>0,\text{当}n>N\text{时},|\dfrac{a_{n}}{a_{n-1}}-1|<\epsilon$
\end{solution}
\begin{anymark}[注]
	(1). 我们可以得到: 
	$$a_{n+1}-a_{n}=\int_{0}^{1}x^{n+1}\sqrt{1-x^2}dx-\int_{0}^{1}x^{n}\sqrt{1-x^2}dx=\int_{0}^{1}(x-1)x^{n}\sqrt{1-x^2}dx<0$$
	
	我们可以得到数列$\{a_{n}\}$单调递减.
	
	我们还有: 
	\begin{eqnarray*}
		a_{n}&=&\int_{0}^{1}x^{n}\sqrt{1-x^2}dx\\
		&=&-\dfrac{1}{3}\int_{0}^{1}x^{n-1}d(1-x^2)^{\frac{3}{2}}\\
		&=&\dfrac{n-1}{3}\int_{0}^{1}x^{n-2}\sqrt{1-x^2}(1-x^2)dx\\
		&=&\dfrac{n-1}{3}a_{n-2}-\dfrac{n-1}{3}a_{n}\\
		a_{n}&=&\dfrac{n-1}{n+2}a_{n-2}
	\end{eqnarray*}
	
	(2). 我们由(1)知道,$\{a_{n}\}$单调递减且$a_{n}>0$,我们得到: 
	$$\left\lbrace
	\begin{array}{l}
		a_{n}<a_{n-1}\\a_{n-1}<a_{n-2}
	\end{array}
	\right. \Rightarrow \left\lbrace
	\begin{array}{l}
		\dfrac{a_{n}}{a_{n-1}}<1\\\dfrac{a_{n}}{a_{n-2}}<\dfrac{a_{n}}{a_{n-1}}
	\end{array}
	\right. $$
	
	我们由夹逼定理可以得到: 
	$$\left\lbrace
	\begin{array}{l}
		\lim\limits_{n\rightarrow +\infty}\dfrac{a_{n}}{a_{n-2}}=1\\
		\lim\limits_{n\rightarrow +\infty}1=1
	\end{array}
	\right. \Rightarrow \lim\limits_{n\rightarrow +\infty}\dfrac{a_{n}}{a_{n-1}}=1$$
	
	综上所述,我们得到: $\lim\limits_{n\rightarrow +\infty}\dfrac{a_{n}}{a_{n-1}}=1$
\end{anymark}
\myspace{1}

2. $xy'-(2x^2+1)y=x^2(x\geq 1)$,且$y(1)=a$,讨论$\lim\limits_{x\rightarrow +\infty}y(x)$
\myspace{1}
\begin{solution}
	
	我们得到微分方程: 
	$$y'-(2x+\frac{1}{x})y=x\Rightarrow (e^{\int (-2x-\frac{1}{x})dx}y)'=xe^{\int (2x+\frac{1}{x})dx}$$
	
	$$y=xe^{x^2}(\int_{1}^{x}e^{-t^2}dt+C)$$
	
	我们由: $y(1)=a\Rightarrow C=\dfrac{a}{e}$
	
	原问题转化为: 
	$$\lim\limits_{x\rightarrow +\infty}xe^{x^2}(\int_{1}^{x}e^{-t^2}dt+\frac{a}{e})$$
	
	我们有: $$\int_{0}^{+\infty}e^{-x^2}dx=\frac{\sqrt{\pi}}{2}$$
	
	原极限可以写作: 
	$$\lim\limits_{x\rightarrow +\infty}xe^{x^2}(\frac{\sqrt{\pi}}{2}-\int_{0}^{1}e^{-t^2}dt+\frac{a}{e})$$
	
	(i).当$a=e(\int_{0}^{1}e^{-t^2}dt-\dfrac{\sqrt{\pi}}{2})$
	
	洛必达法则: $$\lim\limits_{x\rightarrow +\infty}xe^{x^2}(\frac{\sqrt{\pi}}{2}-\int_{0}^{1}e^{-t^2}dt+\frac{a}{e})=\lim\limits_{x\rightarrow +\infty}\frac{x^2}{-2x^2-1}=-\frac{1}{2}$$
	
	(ii).当$a\neq e(\int_{0}^{1}e^{-t^2}dt-\dfrac{\sqrt{\pi}}{2})$,极限不存在.
\end{solution}
\myspace{1}

\hl{\textbf{\textit{May 23}}}

1. 已知当$n$充分大时,$|a_{n}|\leq |b_{n}|\leq |c_{n}|$,且$\lim\limits_{x\rightarrow \infty}a_{n}=\lim\limits_{x\rightarrow \infty}|c_{n}|$,则: 
\begin{itemize}
	\item A. $\lim\limits_{x\rightarrow \infty}(|a_{n}|-b_{n})$
	\item B. $\lim\limits_{x\rightarrow \infty}(|b_{n}|-c_{n})$
	\item C. $\lim\limits_{x\rightarrow \infty}(|b_{n}|-c_{n})$
	\item \hl{D}. $\lim\limits_{x\rightarrow \infty}(|b_{n}|-a_{n})$
\end{itemize}
\myspace{1}
\begin{solution}
	
	我们有: $\lim\limits_{x\rightarrow \infty}x_{n}=A\Rightarrow \lim\limits_{x\rightarrow \infty}|x_{n}|=A$
	
	我们假设: $\lim\limits_{x\rightarrow \infty}a_{n}=\lim\limits_{x\rightarrow \infty}|c_{n}|=A$
	
	我们得到: $\lim\limits_{x\rightarrow \infty}|a_{n}|=A$
	
	根据夹逼定理: $\lim\limits_{x\rightarrow \infty}|b_{n}|=A$
	
	我们已知的极限有$\lim\limits_{x\rightarrow \infty}a_{n},\lim\limits_{x\rightarrow \infty}|a_{n}|,\lim\limits_{x\rightarrow \infty}|b_{n}|,\lim\limits_{x\rightarrow \infty}|c_{n}|$
	
	答案:$D$.
\end{solution}
\myspace{1}

2. $\lim\limits_{n\rightarrow+\infty}\sin^{2}(\pi\sqrt{n^2+3n})$
\myspace{1}
\begin{solution}
	
	原极限等价于: 
	\begin{eqnarray*}
		\lim\limits_{n\rightarrow+\infty}\sin^{2}(\pi\sqrt{n^2+3n}-\pi n)&=&\lim\limits_{n\rightarrow+\infty}\sin^{2}(\dfrac{3n\pi }{\sqrt{n^2+3n}+n})\\
		&=&\sin^2(\frac{3\pi}{2})=1
	\end{eqnarray*}	
\end{solution}
\myspace{1}

\hl{\textbf{\textit{May 24}}}

1. $\lim\limits_{n\rightarrow +\infty}\dfrac{1}{\sqrt[n]{n!}}\left[\dfrac{1}{n+\ln 1}+\dfrac{2}{n+\ln 2}+\cdots+\dfrac{n}{n+\ln n} \right]$
\myspace{1}
\begin{lemma}[斯特林公式]\label{lem: 斯特林公式}
	$$n!\approx \sqrt{2\pi n}(\dfrac{n}{e})^n$$
\end{lemma}
\begin{anymark}[注]
	\begin{eqnarray*}
		\lim\limits_{n\to +\infty}\dfrac{n}{\sqrt[n]{n!}}
		& = &\lim\limits_{n\to +\infty}e^{-\frac{1}{n}(\ln\frac{1}{n}+\ln\frac{2}{n}+\cdots+\ln\frac{n}{n})}\\
		& = &e^{-\int_{0}^{1}\ln x dx}\\
		& = &e
	\end{eqnarray*}
\end{anymark}
\begin{solution}
	
	我们不妨设$b_{n}=\dfrac{1}{\sqrt[n]{n!}}\left[\dfrac{1}{n+\ln 1}+\dfrac{2}{n+\ln 2}+\cdots+\dfrac{n}{n+\ln n} \right]$
	$$a_{n}=\dfrac{1}{\sqrt[n]{n!}}[\frac{1}{n}+\frac{2}{n}+\cdots+\frac{n}{n}],\ c_{n}=\dfrac{1}{\sqrt[n]{n!}}[\frac{1}{n+\ln n}+\frac{2}{n+\ln n}+\cdots+\frac{n}{n+\ln n}]$$
	
	我们得到: 
	$$c_{n}<b_{n}<a_{n}\Rightarrow \dfrac{n+1}{2(n+\ln n)}\dfrac{n}{\sqrt[n]{n!}}<b_{n}<\dfrac{n+1}{2n}\dfrac{n}{\sqrt[n]{n!}}$$
	
	我们有: $\lim\limits_{n\rightarrow +\infty}\dfrac{n+1}{2(n+\ln n)}=\lim\limits_{n\rightarrow +\infty}\dfrac{n+1}{2n}=\dfrac{1}{2}$
	
	对于$\lim\limits_{n\rightarrow +\infty}\dfrac{n}{\sqrt[n]{n!}}$
	
	我们可以知道: $\lim\limits_{n\rightarrow +\infty}\dfrac{n}{\sqrt[n]{n!}}=e$
	
	我们得到: $\lim\limits_{n\rightarrow +\infty}a_{n}=\lim\limits_{n\rightarrow +\infty}b_{n}=\dfrac{e}{2}$
	
	由夹逼定理,我们得到: $\lim\limits_{n\rightarrow +\infty}b_{n}=\dfrac{e}{2}$
\end{solution}
\myspace{1}

2. $f(x)>0,f(x)f''(x)\geq [f'(x)]^2,f(0)=1$,证明: $f(x)\geq e^{f'(0)x}$
\myspace{1}
\begin{solution}
	
	这个题还有第一问: $f(x_{1})f(x_{2})\geq f^2(\dfrac{x_{1}+x_{2}}{2})$
	
	我们由: $$f(x)f''(x)\geq [f'(x)]^2\Rightarrow \text{可以构造出函数}\frac{f'(x)}{f(x)}\text{单调递增}$$
	
	我们将$f(x_{1})f(x_{2})\geq f^2(\dfrac{x_{1}+x_{2}}{2})$取对数得到: 
	$$\ln f(x_{1})+\ln f(x_{2})\geq 2\ln f(\frac{x_{1}+x_{2}}{2})$$
	
	我们构造函数$g(x)=\ln f(x)$
	
	原命题转化为: $\dfrac{g(x_{1})+g(x_{2})}{2}\geq g(\frac{x_{1}+x_{2}}{2})$
	
	我们只需要证明: $g(x)\text{是凹函数},g''(x)\geq 0$
	
	我们有: $$g'(x)=\dfrac{f'(x)}{f(x)},\ g''(x)=\dfrac{f(x)f''(x)-[f'(x)]^2}{f^2(x)}\geq 0$$
	
	$g(x)\text{为凹函数}\Rightarrow \dfrac{g(x_{1})+g(x_{2})}{2}\geq g(\frac{x_{1}+x_{2}}{2})$
	
	我们证明了: $f(x_{1})f(x_{2})\geq f^2(\frac{x_{1}+x_{2}}{2})$
	
	$$g''(x)\geq 0\Rightarrow g'(x)\geq g'(0)=\dfrac{f'(0)}{f(0)}$$
	
	我们可以理解为: $g(x)\text{始终大于}g(0)\text{处的切线}$
	
	$$g(x)\geq g'(0)x+g(0)\Rightarrow \ln f(x)\geq f'(0)x\Rightarrow f(x)\geq e^{f'(0)x}$$
\end{solution}
\myspace{1}

\hl{\textbf{\textit{May 25}}}

1. 设$f'(x)$在$[a,b]$上连续,且$f(a)=f(b)=0$,证明: $|f(x)|\leq \frac{1}{2}\int_{a}^{b}|f'(x)|dx$
\myspace{1}
\begin{solution}
	
	我们可以得到: $$f(x)=f(x)-f(a)=\int_{a}^{x}f'(t)tdt,\ f(x)=-(f(b)-f(x))=-\int_{x}^{b}f'(t)tdt$$
	
	我们得到: 
	$$\left\lbrace 
	\begin{array}{l}
		|f(x)|=|\int_{a}^{x}f'(t)tdt|\leq\int_{a}^{x}|f'(t)|dt\\
		|f(x)|=|\int_{x}^{b}f'(t)tdt|\leq\int_{x}^{b}|f'(t)|dt
	\end{array}
	\right. \Rightarrow 2|f(x)|\leq\int_{a}^{b}|f'(x)|dx$$
	
	我们证明: $|f(x)|\leq \frac{1}{2}\int_{a}^{b}|f'(x)|dx$
\end{solution}
\myspace{1}

2. $\iint_{D}\ln|sin(x-y)|dxdy,\ D=\{(x,y)|0\leq x\leq y\leq 2\pi\}$
\myspace{1}
\begin{solution}
	
	利用二重积分换元公式: $\left\lbrace 
	\begin{array}{l}
		u=y-x\\
		v=x
	\end{array}
	\right.\Rightarrow dudv=dxdy,\ 0\leq u\leq \pi-v,0\leq v\leq \pi$
	
	我们得到原二重积分等价于: 
	$$I=\int_{0}^{\pi}du\int_{0}^{\pi-u}\ln(sin u)dudv=\int_{0}^{\pi}(\pi-u)\ln(\sin u)du$$
	
	利用区间再现公式: 
	$$2I=\pi\int_{0}^{\pi}\ln(\sin u)du\Rightarrow I=\pi\int_{0}^{\frac{\pi}{2}}\ln(\sin u)du$$
	
	$$I=-\frac{\ln 2\pi^2}{2}$$
	\begin{lemma}[区间再现例子]
		\begin{itemize}
			\item $\int_{0}^{\pi}\ln(\sin x)dx=\int_{0}^{\pi}\ln(\cos x)dx=-\dfrac{\ln 2\pi}{2}$
			\item $\int_{0}^{\frac{\pi}{2}}\ln(\tan x)dx=0$
		\end{itemize}
		\begin{anymark}[注]
			$$\int_{0}^{\frac{\pi}{2}}\ln(\sin x)dx=\int_{0}^{\frac{\pi}{2}}\ln(\cos x)dx$$
			
			我们有: 
			\begin{eqnarray*}
				2I&=&\int_{0}^{\frac{\pi}{2}}\ln(\sin x\cos x)dx\\
				&=&\int_{0}^{\frac{\pi}{2}}\ln(\sin 2x)dx-\int_{0}^{\frac{\pi}{2}}\ln 2dx\\
				&=&I-\frac{\ln 2\pi}{2}\\
				I&=&-\frac{\ln 2\pi}{2}
			\end{eqnarray*}
		\end{anymark}
		$$\int_{0}^{\pi}xf(\sin x)dx=\frac{\pi}{2}\int_{0}^{\pi}f(\sin x)dx$$
	\end{lemma}
\end{solution}
\myspace{1}

\hl{\textbf{\textit{May 26}}}

1. 设$x_{n}=(1+\dfrac{1}{n^2})(1+\dfrac{2}{n^2})\cdots(1+\dfrac{n}{n^2})$,求$\lim\limits_{n\rightarrow\infty}x_{n}$
\myspace{1}
\begin{solution}
	
	原极限等价于: 
	\begin{eqnarray*}
		I&=&\lim\limits_{n\rightarrow\infty}e^{\ln(1+\frac{1}{n^2})}e^{\ln(1+\frac{2}{n^2})}\cdots e^{\ln(1+\frac{n}{n^2})}\\
		&=&\lim\limits_{n\rightarrow\infty}e^{\frac{1}{n^2}}e^{\frac{2}{n^2}}\cdots e^{\frac{n}{n^2}}\\
		&=&\lim\limits_{n\rightarrow\infty}e^{\frac{n(n+1)}{2n^2}}\\
		&=&e^{\frac{1}{2}}
	\end{eqnarray*}
	\begin{lemma}[不等式放缩]
		\begin{itemize}
			\item $\text{琴生不等式: }\dfrac{2}{\pi}x\leq \sin x\leq x,x\in[0,\frac{\pi}{2}]$
			\item $\text{琴生不等式: }x\leq \tan x\leq \dfrac{4}{\pi}x,x\in[0,\frac{\pi}{4}]$
			\item $\text{泰勒不等式: }\dfrac{x}{1+x}\leq \ln(1+x)\leq x,\ x\in[0,+\infty)$
		\end{itemize}
	\end{lemma}
	我们利用不等式放缩: 
	$$\frac{k}{n^2+k}\leq \ln(1+\frac{k}{n^2})\leq \frac{k}{n^2}\Rightarrow \frac{k}{n^2+n}\leq \ln(1+\frac{k}{n^2})\leq \frac{k}{n^2}$$
	
	我们得到: 
	$$\sum\limits_{k=1}^{n}\frac{k}{n^2+n}\leq \sum\limits_{k=1}^{n}\ln(1+\frac{k}{n^2})\leq \sum\limits_{k=1}^{n}\frac{k}{n^2}$$
	
	$$\lim\limits_{n\rightarrow\infty}\sum\limits_{k=1}^{n}\frac{k}{n^2+n}=\lim\limits_{n\rightarrow\infty}\sum\limits_{k=1}^{n}\dfrac{k}{n^2}=\dfrac{1}{2}$$
	
	由夹逼定理,我们得到: $\sum\limits_{k=1}^{n}\ln(1+\dfrac{k}{n^2})=\dfrac{1}{2}\Rightarrow I=e^{\frac{1}{2}}$
\end{solution}
\myspace{1}

2. 若函数$\varphi(x)$具有二阶导数,且满足$\varphi(2)>\varphi(1)$,$\varphi(2)>\int_{2}^{3}\varphi(x)dx$,证明: 至少存在一点$\xi\in(1,3)$,$s.t.\varphi ''(x)<0$
\myspace{1}
\begin{solution}
	
	我们有积分中值定理得到: 
	$$\exists \eta\in(2,3),\ s.t. \int_{2}^{3}\varphi(x)dx=\varphi(\eta)$$
	
	我们由拉格朗日中值定理得到: 
	$$\left\lbrace 
	\begin{array}{l}
		\exists \xi_{1}\in(1,2),\ s.t. \dfrac{\varphi(2)-\varphi(1)}{2-1}=\varphi '(\xi_{1})>0\\
		\exists \xi_{2}\in(2,\eta),\ s.t. \dfrac{\varphi(\eta)-\varphi(2)}{\eta-2}=\varphi '(\xi_{2})<0
	\end{array}
	\right. $$
	
	我们在区间$(\xi_{1},\xi_{2})$内使用拉格朗日中值定理得到: 
	$$\exists \xi_{3}\in(\xi_{1},\xi_{2}),\ s.t. \dfrac{\varphi(\xi_{2})-\varphi(\xi_{1})}{\xi_{2}-\xi_{1}}=\varphi '(\xi_{3})<0$$
	
	综上所述,我们得到: $\exists \xi\in(1,3)$,$s.t.\varphi ''(x)<0$
\end{solution}
\myspace{1}

\hl{\textbf{\textit{May 27}}}

1. 设$x_{0}=0,x_{n}=\dfrac{1+2x_{n-1}}{1+x_{n-1}}$,证明数列$\{x_{n}\}$收敛,并求极限$\lim\limits_{n\rightarrow +\infty}x_{n}$
\myspace{1}
\begin{solution}
	
	我们有: $x_{1}=1,x_{2}=\dfrac{3}{2},x_{n}>0$
	
	我们使用数学归纳法来证明$x_{n}$单调递增: ($x_{n+1}>x_{n}$)
	
	(1).当$n=1$时,$x_{2}>x_{1}$成立
	
	(2).当$n\geq 1$时,假设$n=k$时,$x_{k+1}>x_{k}$,我们有: 
	$$x_{k+1}=\dfrac{1+2x_{k}}{1+x_{k}}>x_{k}\Rightarrow 1+2x_{k}-x_{k}>0$$
	
	当$n=k+1$时: 
	$$x_{k+2}=\dfrac{1+2\dfrac{1+2x_{k}}{1+x_{k}}}{1+\dfrac{1+2x_{k}}{1+x_{k}}}=\dfrac{3+5x_{k}}{2+3x_{k}}$$
	
	$$x_{k+2}-x_{k+1}=\dfrac{3+5x_{k}}{2+3x_{k}}-\dfrac{1+2x_{k}}{1+x_{k}}=\dfrac{1+2x_{k}-x_{k}^2}{(2+3x_{k})(1+x_{k})}>0$$
	
	我们证明: $x_{n}$单调递增,且$x_{n}-2x_{n}-1<0\Rightarrow |x_{n}|<3$
	
	$\{x_{n}\}$单调递增且有上界,我们得到$x_{n}$极限必定存在,我们不妨设: 
	
	$$\lim\limits_{n\rightarrow +\infty}x_{n}=A(A>0)\Rightarrow A=\frac{1+2A}{1+A}\Rightarrow A=\dfrac{1+\sqrt{5}}{2}$$
\end{solution}
\myspace{1}

2. $\lim\limits_{n\rightarrow +\infty}\int_{n}^{n+1}x^2\sin\dfrac{1}{x}dx$
\myspace{1}
\begin{solution}
	
	我们有: 
	$$\lim\limits_{x\rightarrow +\infty}x^2\sin\frac{1}{x}=\lim\limits_{x\rightarrow +\infty}x\dfrac{\sin\frac{1}{x}}{\frac{1}{x}}=+\infty$$
	
	我们有: $\lim\limits_{n\rightarrow +\infty}\int_{n}^{n+1}x^2\sin\dfrac{1}{x}dx=+\infty$
\end{solution}
\myspace{1}

\hl{\textbf{\textit{May 28}}}

1. 判断级数的敛散性: $\sum\limits_{n=1}^{+\infty}\dfrac{n+1}{\sqrt{n^5-n+2}}$和$\sum\limits_{n=1}^{+\infty}\dfrac{1}{\sqrt{n(n+1)}(\sqrt{n+1}+\sqrt{n})}$
\myspace{1}
\begin{solution}
	
	我们运用比较判别法的极限形式: 
	$$\lim\limits_{n\rightarrow +\infty}\dfrac{\dfrac{n+1}{\sqrt{n^5-n+2}}}{n^{-\frac{3}{2}}}=1$$
	
	原级数收敛.
	
	$$\lim\limits_{n\rightarrow +\infty}\dfrac{\frac{1}{\sqrt{n(n+1)}(\sqrt{n+1}+\sqrt{n})}}{2n^{-\frac{3}{2}}}=\frac{1}{2}$$
	
	原级数收敛.
\end{solution}
\myspace{1}

2. $f(x)$在$[0,1]$上连续,且$\int_{0}^{1}f(x)dx=A$,求$\int_{0}^{1}dx\int_{x}^{1}dy\int_{x}^{y}f(x)f(y)f(z)dz$
\myspace{1}
\begin{solution}
	
	我们不妨设: $F(x)=\int_{0}^{x}f(t)dt,\ F(0)=0,F(1)=A$,原积分等价于: 
	\begin{eqnarray*}
		I&=&\int_{0}^{1}dx\int_{x}^{1}f(x)f(y)[F(y)-F(x)]dy\\
		&=&\int_{0}^{1}f(x)dx\int_{x}^{1}f(y)F(y)dy-\int_{0}^{1}f(x)F(x)dx\int_{x}^{1}f(y)dy\\
		&=&\int_{0}^{1}f(x)[\dfrac{F^2(1)}{2}-\dfrac{F^2(x)}{2}]dx-\int_{0}^{1}f(x)F(x)[F(1)-F(x)]dx\\
		&=&\frac{A^3}{2}-\frac{A^3}{6}-\frac{A^3}{2}+\frac{A^3}{3}\\
		&=&\frac{A^3}{6}
	\end{eqnarray*}
\end{solution}
\myspace{1}

\hl{\textbf{\textit{May 29}}}

1. 判断级数$\sum\limits_{n=1}^{+\infty}(-1)^{n-1}\dfrac{2+(-1)^n}{n^2}$敛散性
\myspace{1}
\begin{solution}
	
	我们不妨设: $a_{n}=\sum\limits_{n=1}^{+\infty}(-1)^{n-1}\dfrac{2}{n^2}$,$b_{n}=\sum\limits_{n=1}^{+\infty}-\frac{1}{n^2}$
	
	我们根据莱布尼茨判别法得到: 
	$a_{n}$收敛,$b_{n}$收敛,原级数收敛.
\end{solution}
\myspace{1}

2. $\iint_{D}|\dfrac{x+y}{2}-x^2-y^2|dxdy,D=\{(x,y)|x^2+y^2\leq 1\}$
\myspace{1}
\begin{solution}
	
	原二重积分可化为: 
	$$I=\iint_{D}|\dfrac{1}{8}-(x-\dfrac{1}{4})^2-(y-\dfrac{1}{4})^2|dxdy,D:\{(x,y)|x^2+y^2\leq 1\}$$
	
	$$I=\iint_{D}(x^2+y^2-\dfrac{x+y}{2})dxdy-2\iint_{D'}(x^2+y^2-\dfrac{x+y}{2})dxdy$$
	
	其中: $$D=\{(x,y)|x^2+y^2\leq 1\},\ D'=\{(x,y)|(x-\frac{1}{4})^2+(y-\frac{1}{4})^2\leq \frac{1}{8}\}$$
	
	$$I=\int_{0}^{2\pi}d\theta\int_{0}^{1}[r^3-\frac{r^2}{2}(\sin\theta+\cos\theta)]dr-2\int_{-\frac{\pi}{4}}^{\frac{3\pi}{4}}d\theta\int_{0}^{\frac{\sin\theta+\cos\theta}{2}}[r^3-\frac{r^2}{2}(\sin\theta+\cos\theta)]dr$$
	
	$$I=\frac{\pi}{2}+\frac{1}{24}\int_{0}^{\pi}\sin^4\theta d\theta=\frac{\pi}{2}+\frac{\pi}{64}=\frac{33\pi}{64}$$
\end{solution}
\myspace{1}

\hl{\textbf{\textit{May 30}}}

1. 设$f(x,y)$在区域$0\leq x\leq 1,0\leq y\leq 1$上连续,$f(0,0)=0$,且$f(x)$在点$(0,0)$处可微,$f'_{y}(0,0)=1$,求$\lim\limits_{x\rightarrow 0^{+}}\dfrac{\int_{0}^{x^2}dt\int_{x}^{\sqrt{t}}f(t,u)du}{1-e^{-\frac{x^4}{4}}}$
\myspace{1}
\begin{solution}
	
	我们交换积分次序得到: 
	\begin{eqnarray*}
		I&=&-\lim\limits_{x\rightarrow 0^{+}}\dfrac{\int_{0}^{x}du\int_{0}^{u^2}f(t,u)dt}{\frac{x^4}{4}}\\
		&=&-\lim\limits_{x\rightarrow 0^{+}}\dfrac{\int_{0}^{x^2}f(t,x)dt}{x^3}
	\end{eqnarray*}
	
	我们由积分中值定理得到: 
	$$\exists \xi\in(0,x^2),\ s.t. \int_{0}^{x^2}f(t,x)dt=x^2f(\xi,x)$$
	
	原极限可以化为: 
	$$I=-\lim\limits_{x\rightarrow 0^{+}}\dfrac{x^2f(\xi,x)}{x^3}=-\lim\limits_{x\rightarrow 0^{+}}\dfrac{f(\xi,x)}{x}$$
	
	$f(x,y)$在$(0,0)$处可微,我们得到: 
	$$f(x+\Delta x,y+\Delta y)=f(0,0)+f'_{x}(0,0)\Delta x+f'_{y}(0,0)\Delta y+o(\sqrt{(\Delta x)^2+(\Delta y)^2})$$
	
	我们有: 
	$$f(\xi,x)=f(0,0)+f'_{x}(0,0)\xi+f'_{y}(0,0)x+o(\sqrt{(\xi)^2+x^2})=f'_{x}(0,0)\xi+x+o(\sqrt{\xi^2+x^2})$$
	
	我们知道: 
	$$0<\xi<x^2\rightarrow \lim\limits_{x\rightarrow 0}\dfrac{\xi}{x}=0$$
	$$0<\sqrt{\xi^2+x^2}<\sqrt{(x^4+x^2}\rightarrow \lim\limits_{x\rightarrow 0}\dfrac{\sqrt{\xi^2+x^2}}{x}=0 $$
	
	我们得到原极限$I=-1$
\end{solution}
\myspace{1}

2. 设$f(x)=1-\cos x$,则$\lim\limits_{x\rightarrow 0}\dfrac{(1-\sqrt{\cos x})(1-\sqrt[3]{\cos x})(1-\sqrt[4]{\cos x})(1-\sqrt[5]{\cos x})}{f\{f[f(x)]\}}$
\myspace{1}
\begin{solution}
	
	我们有: 
	$$\left\lbrace 
	\begin{array}{l}
		x\rightarrow 0,\sqrt{\cos x}-1=\sqrt{1+\cos x-1}-1\sim \frac{1}{2}(\cos x-1) \\
		x\rightarrow 0,\sqrt[3]{\cos x}-1=\sqrt[3]{1+\cos x-1}-1\sim \frac{1}{3}(\cos x-1)\\
		x\rightarrow 0,\sqrt[4]{\cos x}-1=\sqrt[4]{1+\cos x-1}-1\sim \frac{1}{4}(\cos x-1)\\
		x\rightarrow 0,\sqrt[5]{\cos x}-1=\sqrt[5]{1+\cos x-1}-1\sim \frac{1}{5}(\cos x-1)
	\end{array}
	\right. $$
	
	原极限为: 
	$$\lim\limits_{x\rightarrow 0}\dfrac{\dfrac{1}{120}(1-\cos x)^4}{\dfrac{1}{8}(1-\cos x)^4}=\frac{1}{15}$$
\end{solution}
\myspace{1}

\hl{\textbf{\textit{May 31}}}

1. $z(x,y)=\int_{0}^{x}dt\int_{t}^{x}f(t+y)g(yu)du$,$f\text{和}g'$连续,求$\dfrac{\partial^2 z}{\partial x\partial y}$
\myspace{1}
\begin{solution}
	
	我们交换积分次序: 
	$$z(x,y)=\int_{0}^{x}du\int_{0}^{u}f(t+y)g(yu)dt$$
	
	我们得到: 
	$$\dfrac{\partial z}{\partial x}=g(xy)\int_{0}^{x}f(t+y)dt$$
	
	我们令$t+y=u,t=u-y,dt=du$,我们得到: 
	$$\dfrac{\partial z}{\partial x}=g(xy)\int_{y}^{x+y}f(u)du$$
	
	我们得到: 
	$$\dfrac{\partial^2 z}{\partial x\partial y}=xg'(xy)\int_{y}^{x+y}f(t)dt+g(xy)[f(x+y)-f(y)]$$
\end{solution}
\myspace{1}

1.$F(x)=\int_{0}^{x}e^{tx-t^2}dt$,求$F'(x)$
\myspace{1}
\begin{solution}
	$$F(x)=\int_{0}^{x}e^{\frac{x^2}{4}-(t-\frac{x}{2})^2}dt=e^{\frac{x^2}{4}}\int_{0}^{x}e^{-(t-\frac{x}{2})^2}dt$$
	
	令$t-\dfrac{x}{2}=u,t=u+\dfrac{x}{2},dt=du$,我们得到: 
	$$F(x)=e^{\frac{x^2}{4}}\int_{-\frac{x}{2}}^{\frac{x}{2}}e^{-u^2}du$$
	$$F'(x)=\frac{x}{2}e^{\frac{x^2}{4}}\int_{-\frac{x}{2}}^{\frac{x}{2}}e^{-u^2}du+e^{\frac{x^2}{4}}e^{-1\frac{x^2}{4}}=\frac{x}{2}e^{\frac{x^2}{4}}\int_{-\frac{x}{2}}^{\frac{x}{2}}e^{-u^2}du+1$$
\end{solution}