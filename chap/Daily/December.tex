\chapterimage{chap34.jpg}
\chapter{December}
\section{Week \Rmnum{1}}
\hl{\textbf{\textit{December 1}}}

1. 求曲线 $y=\dfrac{1}{x}$ 与直线 $y=x$ 及 $y=2$ 所围区域绕 $y=2$ 旋转所得旋转体的体积
\myspace{1}
\begin{solution}
	
\end{solution}
\myspace{1}

\hl{\textbf{\textit{December 2}}}

1. 设曲线 $y=\sin x(0\leq x\leq n\pi,n=1,2,\cdots)$ 和 $x$ 轴围成的区域为 $A$,区域 $A$ 绕 $y$ 轴旋转所得旋转体体积为 $S_{n}$

(1). 求 $S_{n}$

(2). 求极限 $\lim\limits_{n\to\infty}[\dfrac{S_{1}}{n^{3}+1^{3}}+\dfrac{S_{2}}{n^{3}+2^{3}}+\cdots+\dfrac{S_{n}}{n^{3}+n^{3}}]$
\myspace{1}
\begin{solution}
	
\end{solution}
\myspace{1}

\hl{\textbf{\textit{December 3}}}

1. 设 $f(x),g(x)$ 在 $[0,1]$ 上连续,在 $(0,1)$ 内可导,且 $\int_{0}^{1}f(x)dx=3\int_{\frac{2}{3}}^{1}f(x)dx$,试证明:存在两个不同的点 $\xi,\eta\in(0,1),s.t.\ f'(\xi)=g'(\xi)[f(\eta)-f(\xi)]$
\myspace{1}
\begin{solution}
	
\end{solution}
\myspace{1}

\hl{\textbf{\textit{December 4}}}

1. 设函数 $f(x)$ 在闭区间 $[a,b]$ 上有连续导数,证明:
$$\lim\limits_{n\to\infty}n[\int_{a}^{b}f(x)dx-\dfrac{b-a}{n}\sum\limits_{k=1}^{n}f(a+\dfrac{k(b-a)}{n})]=\dfrac{b-a}{2}[f(a)-f(b)]$$
\myspace{1}
\begin{solution}
	
\end{solution}
\myspace{1}

\hl{\textbf{\textit{December 5}}}

1. 设函数 $f(x)$ 在 $[0,\pi]$ 上连续

(1). 证明 $\lim\limits_{n\to +\infty}\int_{0}^{\pi}f(x)|\sin nx|dx=\dfrac{\pi}{2}\int_{0}^{\pi}f(x)dx$

(2). 求极限 $\lim\limits_{n\to +\infty}\int_{0}^{\pi}|\sin nx|\ln(1+x)dx$ 
\myspace{1}
\begin{solution}
	
\end{solution}
\myspace{1}

\hl{\textbf{\textit{December 6}}}

1. 设函数 $f(x,y)=\left\lbrace
	\begin{array}{l}
		\dfrac{(x+y)\sqrt{|xy|}}{\sqrt{x^{2}+y^{2}}},(x,y)\neq (0,0)\\
		0,(x,y)=(0,0)
	\end{array}
	\right.$,则 $f(x)$ 在点 $(0,0)$处
\begin{itemize}
	\item A. 不连续
	\item B. 两个偏导数都不存在
	\item C. 两个偏导数存在但不可微
	\item D. 可微
\end{itemize} 
\myspace{1}
\begin{solution}
	
\end{solution}
\myspace{1}

\hl{\textbf{\textit{December 7}}}

1. $\lim\limits_{x\to 0\\y\to 0}\dfrac{f(x,y)-f(0,0)}{\sqrt{x^{2}+y^{2}}}=0$ 是函数 $f(x,y)$ 在 $(0,0)$ 点可微的:
\begin{itemize}
	\item A. 充分必要条件
	\item B. 必要条件但非充分条件
	\item C. 充分条件但非必要条件
	\item D. 既非充分也非必要条件
\end{itemize} 
\myspace{1}
\begin{solution}
	
\end{solution}
\myspace{1}

\section{Week \Rmnum{2}}
\hl{\textbf{\textit{December 8}}}

1. 设 $f_{x}(x_{0},y_{0})$ 存在,$f_{y}(x_{0},y_{0})$ 在点 $(x_{0},y_{0})$ 处连续,尝试证明:$f(x,y)$ 在点 $(x_{0},y_{0})$ 处可微
\myspace{1}
\begin{solution}
	
\end{solution}
\myspace{1}

\hl{\textbf{\textit{December 9}}}

1. 设 $u=u(\sqrt{x^{2}+y^{2}})$ 具有连续的二阶偏导数,且满足 $\dfrac{\partial^{2} u}{\partial x^{2}}+\dfrac{\partial^{2} u}{\partial x^{2}}-\dfrac{1}{x}\dfrac{\partial u}{\partial x}+u=x^{2}+y^{2}$,试求函数 $u$ 的表达式 
\myspace{1}
\begin{solution}
	
\end{solution}
\myspace{1}

\hl{\textbf{\textit{December 10}}}

1. 设 $f(x,y)$ 有二阶连续导数,$g(x,y)=f(e^{xy},x^{2}+y^{2})$ 且 $\lim\limits_{x\to 1\\y\to 0}\dfrac{f(x,y)+x+y-1}{\sqrt{(x-1)^{2}+y^{2}}}=0$,证明: $g(x,y)$ 在 $(0,0)$ 点取得极值,判断此极值是极大值还是极小值,并求出此极值
\myspace{1}
\begin{solution}
	
\end{solution}
\myspace{1}

\hl{\textbf{\textit{December 11}}}

1. 设区域 $D$ 由 $x^{2}+y^{2}\leq 2x+2y$ 所确定,求 $\iint\limits_{D}[x(1-y^{3})+y(1+x^{3})]d\sigma$
\myspace{1}
\begin{solution}
	
\end{solution}
\myspace{1}

\hl{\textbf{\textit{December 12}}}

1. 设 $D=\{(x,y)|0\leq x\leq 2,0\leq y\leq 2\}$

(1). 计算 $b=\iint\limits_{D}|xy-1|d\sigma$

(2). 设 $f(x,y)$ 在 $D$ 上连续,且 $\iint\limits_{D}f(x,y)d\sigma=0,\iint\limits_{D}xyf(x,y)d\sigma=1$,证明: 存在 $(\xi,\eta)\in D$, 使得 $|f(\xi,\eta)|\geq \dfrac{1}{b}$
\myspace{1}
\begin{solution}
	
\end{solution}
\myspace{1}

\hl{\textbf{\textit{December 13}}}

1. 设 $f(x)$ 有一阶连续导数,$(xy-yf(x))dx+(f(x)+y^{2})dy=du(x,y)$,其中 $f(0)=-1,u(0,0)=0$,则函数 $u(x,y)$ 在条件 $xy=1,x>0$ 下最值情况:
\begin{itemize}
	\item A. 最大值为 $\dfrac{5}{3}$
	\item B. 最大值为 $5$
	\item C. 最小值为 $-3$
	\item D. 最小值为 $\dfrac{1}{3}$
\end{itemize}
\myspace{1}
\begin{solution}
	
\end{solution}
\myspace{1}

\hl{\textbf{\textit{December 14}}}

1. 设函数 $f(x)$ 在 $[0,+\infty)$ 上可导,区域 $D$ 由不等式 $x^{2}+y^{2}\leq t^{2}(t\geq 0),x\geq 0,y\geq 0$ 所确定,且 $f(t)=2\iint\limits_{D}[(x-1)^{2}+(y+1)^{2}]f(\sqrt{x^{2}+y^{2}})dxdy+\dfrac{t^{4}}{4}+t^{2}$,求 $f(x)$
\myspace{1}
\begin{solution}
	
\end{solution}
\myspace{1}

\section{Week \Rmnum{3}}
\hl{\textbf{\textit{December 15}}}

1. 设函数 $f(x)$ 满足 $xf'(x)-3f(x)+6x^{2}=0$,且由曲线 $y=f(x)$,直线 $x=1$ 与 $x$ 轴围成的平面图形 $D$ 绕 $x$ 轴旋转一周所得旋转体体积最小,求 $D$ 的面积
\myspace{1}
\begin{solution}
	
\end{solution}
\myspace{1}

\hl{\textbf{\textit{December 16}}}

1. 下列级数中条件收敛的是:
\begin{itemize}
	\item A. $\sum\limits_{n=1}^{\infty}\ln(1+\dfrac{(-1)^{n}}{\sqrt{n}})$
	\item B. $\sum\limits_{n=1}^{\infty}\dfrac{(-1)^{n}}{\sqrt{n}}\ln(1+\dfrac{1}{n})$
	\item C. $\sum\limits_{n=1}^{\infty}\dfrac{(-1)^{n}[(-1)^{n}+\ln n]}{n}$
	\item D. $\sum\limits_{n=1}^{\infty}\dfrac{(-1)^{n}}{n\ln n}$
\end{itemize}
\myspace{1}
\begin{solution}
	
\end{solution}
\myspace{1}

\hl{\textbf{\textit{December 17}}}

1. 下列结论正确的是:
\begin{itemize}
	\item A. 若$\sum\limits_{n=0}^{\infty}a_{n}x^{n},\sum\limits_{n=0}^{\infty}b_{n}x^{n}$ 的收敛半径分别是 $R_{1},R_{2}$,则 $\sum\limits_{n=0}^{\infty}(a_{n}\pm b_{n})x^{n}$ 的收敛半径为 $R=\min\{R_{1},R_{2}\}$
	\item B. 若幂级数$\sum\limits_{n=0}^{\infty}a_{n}x^{n}$ 的收敛半径为 $R=2$,则 $\lim\limits_{n\to\infty}|\dfrac{a_{n+1}}{a_{n}}|=\dfrac{1}{2}$
	\item C. 若幂级数$\sum\limits_{n=0}^{\infty}a_{n}x^{n}$ 的收敛半径为 $R$,则幂级数 $\sum\limits_{n=0}^{\infty}a_{2n}x^{2n}$ 的收敛半径为 $\sqrt{R}$
	\item D. 若幂级数$\sum\limits_{n=0}^{\infty}a_{n}x^{n}$ 的收敛半径为 $R$,则幂级数 $\sum\limits_{n=0}^{\infty}a_{n}x^{2n}$ 的收敛半径为 $\sqrt{R}$
\end{itemize}
\myspace{1}
\begin{solution}
	
\end{solution}
\myspace{1}

\hl{\textbf{\textit{December 18}}}

1. 设 $a_{0}=0.a_{1}=1.a_{n+1}=3a_{n}+4a_{n-1}(n=1,2,3,\cdots)$

(1). 求极限 $\lim\limits_{n\to\infty}\dfrac{a_{n+1}}{a_{n}}$

(2). 求幂级数 $\sum\limits_{n=1}^{\infty}\dfrac{a_{n}}{n!}x^{n}$ 的收敛域以及和函数
\myspace{1}
\begin{solution}
	
\end{solution}
\myspace{1}

\hl{\textbf{\textit{December 19}}}

1. 设 $f(x)$ 在 $[0,+\infty)$ 上连续,且 $\int_{0}^{+\infty}f^{2}(x)dx$ 收敛,令 $a_{n}=\int_{0}^{1}f(nx)dx$,证明: $\sum\limits_{n=1}^{\infty}\dfrac{a_{n}^{2}}{n^{\alpha}}(\alpha>0)$ 收敛
\myspace{1}
\begin{solution}
	
\end{solution}
\myspace{1}

\hl{\textbf{\textit{December 20}}}

1. 已知函数 $f(x,y)$ 在点 $(1,1)$ 处的梯度 $\mathbf{grad} f(1,1)=2i-j$,求函数 $f(x,y)$ 在该点沿曲线 $e^{x-1}+xy=2$ 在该点处切线方向(与 $y$ 轴正向夹角小于 $\dfrac{\pi}{2}$)的方向导数
\myspace{1}
\begin{solution}
	
\end{solution}
\myspace{1}

\hl{\textbf{\textit{December 21}}}

1. 设 $f(x,y)$ 连续,且 $f(x,y)=x^{2}-y^{2}+\int\limits_{L}\dfrac{yf(x,y)dx+xf(x,y)dy}{x^{2}+y^{2}}$,其中 $L$ 是从点 $A(-1,0)$ 到 $B(1,0)$ 的上半圆周 $y=\sqrt{1-x^{2}}$,求 $f(x,y)$
\myspace{1}
\begin{solution}
	
\end{solution}
\myspace{1}

\section{Week \Rmnum{4}}
\hl{\textbf{\textit{December 22}}}

1. 求 $I=\oint\limits_{L}(y^{2}+z^{2})dx+(z^{2}+x^{2})dy+(x^{2}+y^{2})dz$,其中 $L$ 是球面 $x^{2}+y^{2}+z^{2}=2bx$ 与柱面 $x^{2}+y^{2}=2ax(b>a>0)$ 的交线($z\geq 0$),$L$ 的方向规定为沿 $L$ 的方向运动时,从 $z$ 轴正往下看,曲线 $L$ 所围球面部分总在左边
\myspace{1}
\begin{solution}
	
\end{solution}
\myspace{1}

\hl{\textbf{\textit{December 23}}}

1. 设 $\Sigma$ 为 $x^{2}+y^{2}+z^{2}=1$ 的外侧,求 $\oiint\limits_{\Sigma}\dfrac{dydz}{x}+\dfrac{dzdx}{y}+\dfrac{dxdy}{z}$
\myspace{1}
\begin{solution}
	
\end{solution}
\myspace{1}
