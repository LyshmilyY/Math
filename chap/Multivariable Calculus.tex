\chapterimage{chap10.jpg}
\chapter{多元函数微分}
\section{多元函数微分概念}
\begin{definition}[多元函数极限]
	
	设二元函数 $f(x,y)$ 定义域为 $D$,$P_{0}(x_{0},y_{0})$ 是 $D$ 的聚点,如果存在常数 $A$ 使得 $\forall \varepsilon >0,\ \exists \delta>0,s.t. \ P(x,y)\in D\cap(P_{0},\delta)$, 都有 $|f(x,y)-A|<\varepsilon $,我们将 $A$ 称作函数 $f(x,y)$ 在 $(x,y)\rightarrow (x_{0},y_{0})$ 时的极限,记作: 
	$$\lim\limits_{(x,y)\rightarrow (x_{0},y_{0})}f(x,y)=A$$
\end{definition}
\begin{definition}[连续]\label{def: 多元微分学概念: 连续、偏导、可微}
	$$\lim\limits_{(x,y)\rightarrow (x_{0},y_{0})}f(x,y)=f(x_{0},y_{0})$$
\end{definition}
\begin{definition}[偏导数]
	
	设函数 $f(x,y)$ 在 $(x,y)$ 邻域有定义, 若极限 $\lim\limits_{\Delta x\rightarrow 0}\frac{f(x+\Delta x,y)-f(x,y)}{\Delta x}$ 存在,我们将这记作 $f(x,y)$ 对 $x$ 的偏导数,记作: 
	$$\frac{\partial z}{\partial x}|_{\mbox{\tiny$\begin{array}{c} 
				x=x_{0}\\ 
				y=y_{0}\\ 
			\end{array}$}},\frac{\partial f}{\partial x}|_{\mbox{\tiny$\begin{array}{c} 
				x=x_{0}\\ 
				y=y_{0}\\ 
			\end{array}$}},z_{x}'|_{\mbox{\tiny$\begin{array}{c} 
				x=x_{0}\\ 
				y=y_{0}\\ 
			\end{array}$}}$$
	$$\lim\limits_{\Delta x\rightarrow 0}\frac{f(x+\Delta x,y)-f(x,y)}{\Delta x}=f_{x}'(x,y)$$
	$$\lim\limits_{\Delta y\rightarrow 0}\frac{f(x,y+\Delta y)-f(x,y)}{\Delta y}=f_{y}'(x,y)$$
	对偏导数进一步求偏导数,我们可以得到混合偏导数: $f''_{xx}(x,y),f''_{yy}(x,y),f''_{xy}(x,y)$
\end{definition}
\begin{definition}[可微]
	
	函数 $f(x,y)$ 在点 $(x,y)$ 处的全增量 $\Delta z=f(x+\Delta x,y+\Delta y)-f(x,y)$ 可表示为: 
	$$\Delta z=A\Delta x+B\Delta y+o(\rho),\quad \rho=\sqrt{(\Delta x)^2+(\Delta y)^2}$$
	
	我们称 $f(x,y)$ 在点 $(x,y)$ 处可微,$A\Delta x+B\Delta y$ 是 $f(x,y)$ 在点 $(x,y)$ 处的全微分.
	$$dz=A\Delta x+B\Delta y=Adx+Bdy$$
\end{definition}
\begin{definition}[偏导数连续性]
	
	$$\lim\limits_{\Delta x\rightarrow 0}\frac{f(x+\Delta x,y)-f(x,y)}{\Delta x}\Leftrightarrow\lim\limits_{\mbox{\tiny$\begin{array}{c} 
				x\rightarrow x_{0}\\ 
				y\rightarrow y_{0}\\ 
			\end{array}$}}f_{x}'(x,y)$$
	
	如果这两个极限相等,我们就称偏导数在此点连续.
\end{definition}
\section{链式法则}

\begin{definition}[链式法则]\label{def: 链式法则}
	$$z=f(u,v),u=\varphi(x,y),v=\phi(x,y)$$
	
	我们可以得到偏导数(6项): 
	$$\frac{\partial z}{\partial x}=\frac{\partial z}{\partial u}\frac{\partial u}{\partial x}+\frac{\partial z}{\partial v}\frac{\partial v}{\partial x}$$
	$$\frac{\partial z}{\partial y}=\frac{\partial z}{\partial u}\frac{\partial u}{\partial y}+\frac{\partial z}{\partial v}\frac{\partial v}{\partial y}$$
	
	高阶偏导数: 
	$$\frac{\partial^2 z}{\partial x^2}=(\frac{\partial (\frac{\partial z}{\partial u})}{\partial u}\frac{\partial u}{\partial x}+\frac{\partial (\frac{\partial z}{\partial u})}{\partial v}\frac{\partial v}{\partial x})\frac{\partial u}{\partial x}+\frac{\partial z}{\partial u}\frac{\partial ^2u}{\partial x^2}+(\frac{\partial (\frac{\partial z}{\partial v})}{\partial u}\frac{\partial u}{\partial x}+\frac{\partial (\frac{\partial z}{\partial v})}{\partial v}\frac{\partial v}{\partial x})\frac{\partial v}{\partial x}+\frac{\partial z}{\partial v}\frac{\partial ^2v}{\partial x^2}$$
	
	$$\frac{\partial^2 z}{\partial y^2}=(\frac{\partial (\frac{\partial z}{\partial u})}{\partial u}\frac{\partial u}{\partial y}+\frac{\partial (\frac{\partial z}{\partial u})}{\partial v}\frac{\partial v}{\partial y})\frac{\partial u}{\partial y}+\frac{\partial z}{\partial u}\frac{\partial ^2u}{\partial y^2}+(\frac{\partial (\frac{\partial z}{\partial v})}{\partial u}\frac{\partial u}{\partial y}+\frac{\partial (\frac{\partial z}{\partial v})}{\partial v}\frac{\partial v}{\partial y})\frac{\partial v}{\partial y}+\frac{\partial z}{\partial v}\frac{\partial ^2v}{\partial y^2}$$
	
	$$\frac{\partial^2 z}{\partial x\partial y}=(\frac{\partial (\frac{\partial z}{\partial u})}{\partial u}\frac{\partial u}{\partial x}+\frac{\partial (\frac{\partial z}{\partial u})}{\partial v}\frac{\partial v}{\partial x})\frac{\partial u}{\partial x}+\frac{\partial z}{\partial u}\frac{\partial ^2u}{\partial y\partial x}+(\frac{\partial (\frac{\partial z}{\partial v})}{\partial u}\frac{\partial u}{\partial x}+\frac{\partial (\frac{\partial z}{\partial v})}{\partial v}\frac{\partial v}{\partial x})\frac{\partial v}{\partial y}+\frac{\partial z}{\partial v}\frac{\partial ^2v}{\partial y\partial x}$$
\end{definition}
\section{隐函数求导法则}

\begin{theorem}[隐函数存在定理 1]\label{the: 隐函数存在定理}
	设函数 $F(x,y)$ 在点 $P(x_{0},y_{0})$ 的某一个邻域内有连续偏导数,$F(x_{0},y_{0})=0$,$F_{y}'\neq 0$,则方程 $F(x,y)=0$ 在点 $(x_{0},y_{0})$ 的某一个邻域内能够确定唯一的连续且具有连续导数的函数 $y=f(x)$,满足 $y_{0}=f(x_{0})$,且 $\frac{dy}{dx}=-\frac{F_{x}'}{F_{y}'}$
\end{theorem}
\begin{theorem}[隐函数存在定理 2]
	设函数 $F(x,y,z)$ 在点 $P(x_{0},y_{0},z_{0})$ 的某一个邻域内有连续偏导数,$F(x_{0},y_{0},z_{0})=0$,$F_{z}'(x_{0},y_{0},z_{0})\neq 0$,则方程 $F(x,y,z)=0$ 在点 $(x_{0},y_{0},z_{0})$ 的某一个邻域内能够确定唯一的连续且具有连续偏导数的函数 $z=f(x,y)$,满足 $z_{0}=f(x_{0},y_{0})$,且满足:  
	$$\frac{\partial z}{\partial x}=-\frac{F_{x}'}{F_{z}'}\quad \frac{\partial z}{\partial y}=-\frac{F_{y}'}{F_{z}'}$$
\end{theorem}
\section{多元函数极值和最值}
\textbf{无条件极值}
\begin{definition}[多元函数极值]
	二元函数 $f(x,y)$ 在点 $(x_{0},y_{0})$ 取极值的必要条件: 
	$$f'_{x}(x_{0},y_{0})=f'_{y}(x_{0},y_{0})=0$$
	
	二元函数 $f(x,y)$ 在点 $(x_{0},y_{0})$ 取极值的充分条件: 
	$$\left\lbrace\begin{array}{l}
		f_{xx}'(x_{0},y_{0})=A\\
		f_{xy}'(x_{0},y_{0})=B\\
		f_{yy}'(x_{0},y_{0})=C
	\end{array} \right.  ,\quad \Delta=AC-B^2\Rightarrow \left\lbrace\begin{array}{l}
		\Delta>0,\left\lbrace\begin{array}{l}
			A>0,min\\A<0,max
		\end{array} \right.\\
		\Delta<0,not \ max,not \ min\\ 
		\Delta=0
	\end{array} \right.$$
\end{definition}
\textbf{条件极值}
\begin{definition}[拉格朗日数乘法]
	
	求目标函数 $u=f(x,y,z)$ 在条件 $\left\lbrace \begin{array}{l}
		g(x,y,z)=0\\h(x,y,z)=0
	\end{array}\right.$ 下的最值
	
	构造辅助函数:  $F(x,y,z,\lambda,\mu)=f(x,y,z)+\lambda g(x,y,z)+\mu h(x,y,z)$
	
	令 $$\left\lbrace \begin{array}{c}
		F_{x}'=f_{x}'+\lambda g_{x}'+\mu h_{x}'=0\\
		F_{y}'=f_{y}'+\lambda g_{y}'+\mu h_{y}'=0\\
		F_{z}'=f_{z}'+\lambda g_{z}'+\mu h_{z}'=0\\
		F_{\lambda}'= g(x,y,z)=0\\
		F_{\mu}'= h(x,y,z)=0
	\end{array}\right. $$
	
	得到所有的备选点 $P_{i}$,计算 $f(P_{i})$ 得到最大值和最小值.
\end{definition}	
\chapterimage{chap11.jpg}
\chapter{空间解析几何}
\section{向量代数}
\begin{definition}
	1. 方向角: 非零向量 $a$ 与$x,y,z$ 轴所成夹角 $\alpha,\beta,\gamma$ 称为向量 $a$ 的方向角. 
	
	2. 方向余弦: $\cos \alpha,\cos \beta,\cos\gamma$ 称为向量 $a$ 的方向余弦.
	$$\cos \alpha=\frac{a_{x}}{|a|},\quad \cos \beta=\frac{a_{y}}{|a|},\quad \cos \gamma=\frac{a_{z}}{|a|}$$
	
	3. 投影: 
	$$Prj_{b}a=\frac{\textbf{a}\bullet\textbf{b}}{|b|}=\frac{a_{x}b_{x}+a_{y}b_{y}+a_{z}b_{z}}{\sqrt{b_{x}^2+b_{y}^2+b_{z}^2}}$$
\end{definition}
\section{空间平面和直线}
\subsection{平面}
\begin{definition}[	平面方程]
	
	1. 一般式:  $Ax+By+Cz+D=0$
	\myspace{1}
	
	2. 点法式:  $A(x-x_{0})+B(y-y_{0}+C(z-z_{0})=0$
	\myspace{1}
	
	3. 截距式:  $\dfrac{x}{a}+\dfrac{y}{b}+\dfrac{z}{c}=1$
	\myspace{1}	
	
	4. 三点式 : $\left|\begin{array}{lll}
		x-x_{1}&y-y_{1}&z-z_{1}\\
		x-x_{2}&y-y_{2}&z-z_{2}\\
		x-x_{3}&y-y_{3}&z-z_{3}
	\end{array}\right|=0$,(平面过不共线的三点 $P(x_{i},y_{i},z_{i})$)
\end{definition}
\subsection{直线}
\begin{definition}[直线方程]
	
	1. 一般式:  $\left\lbrace \begin{array}{c}
		A_{1}x+B_{1}y+C_{1}z=0,n_{1}=(A_{1},B_{1},C_{1})\\A_{2}x+B_{2}y+C_{2}z=0,n_{2}=(A_{2},B_{2},C_{2})
	\end{array}\right. $,$n_{1},n_{2}$ 不平行
	\myspace{1}
	
	2. 点向式:  $\dfrac{x-x_{0}}{l}=\dfrac{y-y_{0}}{m}=\dfrac{z-z_{0}}{n}$,$(x_{0},y_{0},z_{0})$ 是直线上的点,$(l,m,n)$ 是直线的方向向量
	\myspace{1}
	
	3. 参数式:  $\left\lbrace\begin{array}{l}
		x=x_{0}+lt\\
		y=y_{0}+mt\\
		z=z_{0}+nt
	\end{array} \right. $,$(x_{0},y_{0},z_{0})$ 是直线上的点,$(l,m,n)$ 是直线的方向向量
	\myspace{1}	
	
	4.两点式: $\dfrac{x-x_{1}}{x_{1}-x_{2}}=\dfrac{y-y_{1}}{y_{1}-y_{2}}=\dfrac{z-z_{0}}{z_{1}-z_{2}}$,$(x_{1},y_{1},z_{1})$和$(x_{2},y_{2},z_{2})$ 是直线上的点
\end{definition}
\section{空间曲面和曲线}
\begin{definition}[空间曲面和曲线]
	1. 空间曲线: 
	
	(i). 一般式: $\left\lbrace\begin{array}{l}
		F(x,y,z)=0\\G(x,y,z)=0
	\end{array} \right. $, 是两个曲面的交线
	
	(ii). 参数方程式: $\left\lbrace\begin{array}{l}
		x=f(t)\\y=g(t)\\z=h(t)
	\end{array} \right. $,$t\in[\alpha,\beta]$ 为参数
	
	(iii). 空间曲线在坐标面的投影: 
	$$\left\lbrace\begin{array}{l}
		F(x,y,z)=0\\G(x,y,z)=0
	\end{array} \right. \Rightarrow \left\lbrace\begin{array}{l}
		H(x,y)=0\\z=0
	\end{array} \right.$$
	
	2. 空间曲面
	$$F(x,y,z)=0$$
\end{definition}
\subsection{空间曲线的切线和法平面}
\begin{definition}[曲线切线和法平面]
	
	(i).参数方程式: $\left\lbrace\begin{array}{l}
		x=f(t)\\y=g(t)\\z=h(t)
	\end{array} \right. $ 在 $t=t_{0}$ 时,点 $P_{0}(x_{0},y_{0},z_{0})$ 处
	\myspace{1}
	
	切线的方向向量: $\textbf{n}=(f'(t_{0}),g'(t_{0}),h'(t_{0}))$
	\myspace{1}
	
	切线方程为: $ \dfrac{x-x_{0}}{f'(t_{0})}=\dfrac{y-y_{0}}{g'(t_{0})}=\dfrac{z-z_{0}}{h'(t_{0})}$
	\myspace{1}
	
	曲线法平面: $f'(t_{0})(x-x_{0})+g'(t_{0})(y-y_{0})+h'(t_{0})(z-z_{0})=0 $
	\myspace{1}
	
	(ii). 一般式: $\left\lbrace\begin{array}{l}
		F(x,y,z)=0\\G(x,y,z)=0
	\end{array} \right. $
	\myspace{1}
	
	切线的方向向量: $(\left| \begin{array}{ll}
		F_{y}'&F_{z}'\\G_{y}'&G_{z}'
	\end{array}\right| ,\left|\begin{array}{ll}
		F_{z}'&F_{x}'\\G_{z}'&G_{x}'
	\end{array} \right|,\left|\begin{array}{ll}
		F_{x}'&F_{y}'\\G_{x}'&G_{y}'
	\end{array} \right|)$
	\myspace{1}
	
	切线方程为: $\dfrac{x-x_{0}}{\left| \begin{array}{ll}
			F_{y}'&F_{z}'\\G_{y}'&G_{z}'
		\end{array}\right|}=\dfrac{y-y_{0}}{\left|\begin{array}{ll}
			F_{z}'&F_{x}'\\G_{z}'&G_{x}'
		\end{array} \right|}=\dfrac{z-z_{0}}{\left|\begin{array}{ll}
			F_{x}'&F_{y}'\\G_{x}'&G_{y}'
		\end{array} \right|}$
	\myspace{1}
	
	曲线法平面: $\left| \begin{array}{ll}
		F_{y}'&F_{z}'\\G_{y}'&G_{z}'
	\end{array}\right|(x-x_{0})+\left|\begin{array}{ll}
		F_{z}'&F_{x}'\\G_{z}'&G_{x}'
	\end{array} \right|(y-y_{0})+\left|\begin{array}{ll}
		F_{x}'&F_{y}'\\G_{x}'&G_{y}'
	\end{array} \right|(z-z_{0})=0$
	
	
\end{definition}
\subsection{空间曲面的切平面和法线}
\begin{definition}
	曲面的切平面和法线
	
	1. 曲面方程:  $F(x,y,z)=0$
	\myspace{1}
	
	法向量: $\textbf{n}=(F_{x}'(x,y,z),F_{y}'(x,y,z),F_{z}'(x,y,z))$
	\myspace{1}
	
	切平面方程: $F_{x}'(x,y,z)(x-x_{0})+F_{y}'(x,y,z)(y-y_{0})+F_{z}'(x,y,z)(z-z_{0})=0$
	\myspace{1}
	
	法线方程: $\dfrac{x-x_{0}}{F_{x}'(x,y,z)}=\dfrac{y-y_{0}}{F_{y}'(x,y,z)}=\dfrac{z-z_{0}}{F_{z}'(x,y,z)}$
	\myspace{1}
	
	2. 曲面方程:  $z=f(x,y)$
	\myspace{1}
	
	法向量: $\textbf{n}=(f_{x}'(x,y),f_{y}'(x,y),-1)$
	\myspace{1}
	
	切平面方程: $f_{x}'(x,y)(x-x_{0})+f_{y}'(x,y)(y-y_{0})-(z-z_{0})=0$
	\myspace{1}
	
	法线方程: $\dfrac{x-x_{0}}{f_{x}'(x,y)}=\dfrac{y-y_{0}}{f_{y}'(x,y)}=\dfrac{z-z_{0}}{-1}$
\end{definition}
\section{场论初步}
\subsection{方向导数}
\begin{definition}[方向导数]
	
	设三元函数 $u=u(x,y,z)$ 在点 $P(x_{0},y_{0},z_{0})$ 的某空间邻域内 $U\subset R^3$ 有定义,$l$ 是从 $P_{0}$ 出发的一条射线,$P(x,y,z)$ 为 $l$ 上且在 $U$ 中的任意一点,我们有: 
	$$\left\lbrace \begin{array}{l}
		x-x_{0}=\Delta x=t\cos \alpha\\
		y-y_{0}=\Delta y=t\cos \beta\\
		z-z_{0}=\Delta z=t\cos \gamma
	\end{array}\right. $$ 
	
	$t=\sqrt{(\Delta x)^2+(\Delta y)^2+(\Delta z)^2}$ 表示$|PP_{0}|$,如果下面极限存在: 
	$$\lim\lim\limits_{t\rightarrow 0}\frac{u(P)-u(P_{0})}{t}=\lim\lim\limits_{t\rightarrow 0}\frac{u(x_{0}+t\cos \alpha,y_{0}+t\cos \beta,z_{0}+t\cos \gamma)-u(x_{0},y_{0},z_{0})}{t}$$
	
	我们将此极限称为 $u=f(x,y,z)$ 在$P_{0}$ 处沿着 $l$ 的方向导数,记作 $\frac{\partial u}{\partial l}|_{P_{0}}$
\end{definition}
\begin{theorem}[方向导数计算公式]
	$$\frac{\partial u}{\partial l}|_{P_{0}}=u_{x}'\cos \alpha+u_{y}'\cos \beta+u_{z}'\cos \gamma$$
	
	其中$\cos \alpha,\cos \beta,\cos \gamma$ 为方向 $l$ 的方向余弦.
\end{theorem}
\subsection{梯度}
\begin{definition}[梯度]
	
	设三元函数$u=u(x,y,z)$ 在点 $P(x_{0},y_{0},z_{0})$ 处具有一阶偏导数,定义下面为$u=u(x,y,z)$ 在 $ P_{0}(x_{0},y_{0},z_{0})$ 处的梯度: 
	$$\textbf{guad}\ u|_{P_{0}}=(u_{x}'(P_{0}),u_{y}'(P_{0}),u_{z}'(P_{0}))$$
	
	
	梯度和方向导数之间的关系: 
	$$\frac{\partial u}{\partial l}|_{P_{0}}=\textbf{guad}\ u|_{P_{0}}\bullet \textbf{l}=|\textbf{guad}\ u|_{P_{0}}|l|\cos \theta$$
\end{definition}
\subsection{散度和旋度}
\begin{definition}[散度和旋度]
	设向量场 $A(x,y,z)=(P(x,y,z),Q(x,y,z),R(x,y,z))$
	
	散度: 
	$$div\ A=\frac{\partial P}{\partial x}+\frac{\partial Q}{\partial y}+\frac{\partial R}{\partial z}$$
	
	旋度: 
	$$rot \ A=\left| \begin{array}{lll}
		i&j&k\\\dfrac{\partial}{\partial x}&\dfrac{\partial}{\partial y}&\dfrac{\partial}{\partial z}\\P&Q&R
	\end{array}\right| $$
\end{definition}
\chapterimage{chap12.jpg}
\chapter{三重积分}
\begin{definition}[三重积分]
	$$\iiint\limits_{\Omega}f(x,y,z)d\nu$$
	
	我们将 $f(x,y,z)$ 看作空间区域 $d\nu$ 内的密度,积分表示的就是空间区域的质量,$M=\iiint\limits_{\Omega}f(x,y,z)d\nu$,特别的,当 $f(x,y,z)=1$ 时,三重积分表示的积分区域 $\Omega$ 的体积.
\end{definition}
\section{三重积分对称性}

\subsection{普通对称性}

\begin{definition}
	设 $\Omega$ 关于平面 $xoz$ 对称,我们有: 
	$$\iiint\limits_{\Omega}f(x,y,z)d\nu=\left\lbrace \begin{array}{l}
		2\iiint\limits_{\Omega_{1}}f(x,y,z)d\nu,f(x,y,z)=f(x,-y,z)\\
		0,f(x,y,z)=-f(x,-y,z)
	\end{array}\right. $$
\end{definition}
\subsection{轮换对称性}
\begin{definition}
	若将 $x,y,z$ 任意两个交换位置后 积分区域 $\Omega$ 保持不变,我们有: 
	$$\iiint\limits_{\Omega}f(x)d\nu=\iiint\limits_{\Omega}f(y)d\nu=\iiint\limits_{\Omega}f(z)d\nu$$
\end{definition}

\section{三重积分计算方法}

\subsection{直角坐标系}
\begin{definition}
	1. 先一后二: 
	$$\iiint\limits_{\Omega}f(x,y,z)d\nu=\iint\limits_{D}d\sigma \int_{z_{1}(x,y)}^{z_{2}(x,y)}f(x,y,z)dz$$
	
	适用于空间区域无侧面,能"压扁"到一个坐标平面内.
	
	2. 先二后一法: 
	$$\iiint\limits_{\Omega}f(x,y,z)d\nu=\int_{a}^{b}dz\iint\limits_{D_{z}}f(x,y,z)d\sigma$$
	
	适用于旋转体,不能“压扁”到一个坐标平面
\end{definition}
\subsection{柱面坐标系}
\begin{definition}[柱坐标替换]
	$$\iiint\limits_{\Omega}f(x,y,z)d\nu=\int_{a}^{b}dz\iint\limits_{D_{z}}f(x,y,z)d\sigma$$
	
	利用极坐标和直角坐标公式转换: 
	$$\iiint\limits_{\Omega}f(x,y,z)d\nu=\iint\limits_{D_{r\theta}}drd\theta \int_{z_{1}(r,\theta)}^{z_{2}(r,\theta)}rf(r\cos \theta,r\sin\theta,z)dz$$
\end{definition}
\subsection{球面坐标系}
\begin{definition}[球面坐标替换]
	令 $\left\lbrace \begin{array}{l}
		x=r\sin\varphi\cos\theta\\
		y=r\sin\varphi\sin\theta \\\
		z=r\cos\varphi
	\end{array}\right. $,我们有: $d\nu=r^2\sin\varphi drd\theta$
	$$\iiint\limits_{\Omega}f(x,y,z)d\nu=\iiint\limits_{\Omega}r^2\sin\varphi f(r\sin\varphi\cos\theta,r\sin\varphi\sin\theta,r\cos\varphi) drd\theta d\varphi$$
	
	其中: $\varphi\in[0,\pi],\quad \theta\in[0,2\pi]$
\end{definition}
\chapterimage{chap13.jpg}
\chapter{第一型曲线和曲面积分}
\section{第一型曲线积分}
\begin{definition}[第一型曲线积分]
	$$\int_{\Gamma}f(x,y)ds\quad \int_{\Gamma}f(x,y,z)ds$$
	我们将 $f(x,y,z)$ 称为曲线的线密度,第一型曲线积分的意义是求曲线的质量,类比定积分,定积分是在直线上积分,曲线积分则是在曲线上积分.
	
	特别的,我们有: $\int_{\Gamma}ds=L_{\Gamma}$
\end{definition}
\begin{theorem}[曲线积分的求解]
	
	1. 空间曲线
	
	$$\left\lbrace \begin{array}{l}
		x=x(t)\\
		y=y(t)\\
		z=z(t)
	\end{array}\right. ,t\in[\alpha,\beta]$$
	
	我们有: $ds=\sqrt{[x'(t)]^{2}+[y'(t)]^{2}+[z'(t)]^{2}}dt$
	
	$$\int_{L}f(x,y,z)ds=\int_{\alpha}^{\beta}f(x(t),y(t),z(t))\sqrt{[x'(t)]^{2}+[y'(t)]^{2}+[z'(t)]^{2}}dt$$
	
	2. 平面曲线
	
	(i). $L:\ y=f(x),\quad x\in[a,b]$
	
	$$\int_{L}f(x,y)ds=\int_{a}^{b}f(x,y)\sqrt{1+[f'(x)]^2}dx$$
	
	(ii). $L:\ \left\lbrace \begin{array}{l}
		x=x(t)\\
		y=y(t)
	\end{array}\right.\quad t\in[\alpha,\beta]$
	
	$$\int_{L}f(x,y)ds=\int_{\alpha}^{\beta}f(x(t),y(t))\sqrt{[x'(t)]^2+[y'(t)]^2}dt$$
	
	(iii). $L:\ r=r(\theta),\quad \theta\in[\theta_{1},\theta_{2}]$
	
	$$\int_{L}f(x,y)ds=\int_{\theta_{1}}^{\theta_{2}}f(r\cos \theta,r\sin\theta)\sqrt{[r(\theta)]^2+[r'(\theta)]^2}d\theta$$
\end{theorem}

\section{第一型曲面积分}
\begin{definition}[第一型曲面积分]
	$$\iint_{\Sigma}f(x,y,z)dS$$
	我们将 $f(x,y,z)$ 称为曲面的面密度,第一型曲面积分的意义是求曲面的质量,类比二重积分,二重积分是在平面上积分,曲面积分则是在曲面上积分.
	
	特别的,我们有: $\iint_{\Sigma}dS=S_{\Sigma}$
\end{definition}
\begin{theorem}
	$$z=f(x,y)\quad F(x,y,z)=0$$
	
	我们将曲面 $\Sigma$ 投影到任意一个平面,这里以 $xoy$ 为例,$dS=\sqrt{1+(z_{x}')^2+(z_{y}')^2}d\sigma$
	
	$$\iint_{\Sigma}f(x,y,z)dS=\iint_{D_{xy}}f(x,y,z)\sqrt{1+(z_{x}')^2+(z_{y}')^2}d\sigma$$
\end{theorem}
\subsection{应用}
1. 重心、形心

2. 转动惯量
\begin{definition}[转动惯量: $I=mr^2$]
	
	(i). 平面物体: 
	
	对 $x$ 轴: $I_{x}=\iint\limits_{D}y^2\rho(x,y)d\sigma$
	
	对 $y$ 轴: $I_{y}=\iint\limits_{D}x^2\rho(x,y)d\sigma$
	
	对 坐标原点$O$: $I_{O}=\iint\limits_{D}(x^2+y^2)\rho(x,y)d\sigma$
	
	(ii). 空间物体: 
	
	对 $x$ 轴: $I_{x}=\iiint\limits_{\Omega}(y^2+z^2)\rho(x,y,z)d\nu$
	
	对 $y$ 轴: $I_{y}=\iiint\limits_{\Omega}(x^2+z^2)\rho(x,y,z)d\nu$
	
	对 $z$ 轴: $I_{z}=\iiint\limits_{\Omega}(x^2+y^2)\rho(x,y,z)d\nu$
	
	对坐标原点 $O$: $I_{O}=\iiint\limits_{\Omega}(x^2+y^2+z^2)\rho(x,y,z)d\nu$
	
	(iii). 光滑曲线
	
	对 $x$ 轴: $I_{x}=\int\limits_{L}(y^2+z^2)\rho(x,y,z)ds$
	
	对 $y$ 轴: $I_{y}=\int\limits_{L}(x^2+z^2)\rho(x,y,z)ds$
	
	对 $z$ 轴: $I_{z}=\int\limits_{L}(x^2+y^2)\rho(x,y,z)ds$
	
	对坐标原点 $O$: $I_{O}=\int\limits_{L}(x^2+y^2+z^2)\rho(x,y,z)ds$
	
	(iiii). 曲面
	
	对 $x$ 轴: $I_{x}=\iint\limits_{\Sigma}(y^2+z^2)\rho(x,y,z)dS$
	
	对 $y$ 轴: $I_{y}=\iint\limits_{\Sigma}(x^2+z^2)\rho(x,y,z)dS$
	
	对 $z$ 轴: $I_{z}=\iint\limits_{\Sigma}(x^2+y^2)\rho(x,y,z)dS$
	
	对坐标原点 $O$: $I_{O}=\iint\limits_{\Sigma}(x^2+y^2+z^2)\rho(x,y,z)dS$
	
\end{definition}

3. 引力
\begin{definition}[引力公式: $F=\frac{GMm}{r^2}$]
	
	(i). $xoy$ 平面
	$$F_{x}=GM\iint\limits_{D}\frac{\rho(x,y)(x-x_{0})}{[(x-x_{0})^2+(y-y_{0})^2+(z-z_{0})^2]^{\frac{3}{2}}}d\sigma$$
	$$F_{y}=GM\iint\limits_{D}\frac{\rho(x,y)(y-y_{0})}{[(x-x_{0})^2+(y-y_{0})^2+(z-z_{0})^2]^{\frac{3}{2}}}d\sigma$$
	$$F_{z}=GM\iint\limits_{D}\frac{\rho(x,y)(z-z_{0})}{[(x-x_{0})^2+(y-y_{0})^2+(z-z_{0})^2]^{\frac{3}{2}}}d\sigma,z=0$$
	
	(ii). 空间物体
	$$F_{x}=GM\iiint\limits_{\Omega}\frac{\rho(x,y)(x-x_{0})}{[(x-x_{0})^2+(y-y_{0})^2+(z-z_{0})^2]^{\frac{3}{2}}}d\nu$$
	$$F_{y}=GM\iiint\limits_{\Omega}\frac{\rho(x,y)(y-y_{0})}{[(x-x_{0})^2+(y-y_{0})^2+(z-z_{0})^2]^{\frac{3}{2}}}d\nu$$
	$$F_{z}=GM\iiint\limits_{\Omega}\frac{\rho(x,y)(z-z_{0})}{[(x-x_{0})^2+(y-y_{0})^2+(z-z_{0})^2]^{\frac{3}{2}}}d\nu$$
	
	(iii). 曲线
	$$F_{x}=GM\int\limits_{L}\frac{\rho(x,y)(x-x_{0})}{[(x-x_{0})^2+(y-y_{0})^2+(z-z_{0})^2]^{\frac{3}{2}}}ds$$
	$$F_{y}=GM\int\limits_{L}\frac{\rho(x,y)(y-y_{0})}{[(x-x_{0})^2+(y-y_{0})^2+(z-z_{0})^2]^{\frac{3}{2}}}ds$$
	$$F_{z}=GM\int\limits_{L}\frac{\rho(x,y)(z-z_{0})}{[(x-x_{0})^2+(y-y_{0})^2+(z-z_{0})^2]^{\frac{3}{2}}}ds$$
	
	(iiii). 曲面
	$$F_{x}=GM\iint\limits_{\Sigma}\frac{\rho(x,y)(x-x_{0})}{[(x-x_{0})^2+(y-y_{0})^2+(z-z_{0})^2]^{\frac{3}{2}}}dS$$
	$$F_{y}=GM\iint\limits_{\Sigma}\frac{\rho(x,y)(y-y_{0})}{[(x-x_{0})^2+(y-y_{0})^2+(z-z_{0})^2]^{\frac{3}{2}}}dS$$
	$$F_{z}=GM\iint\limits_{\Sigma}\frac{\rho(x,y)(z-z_{0})}{[(x-x_{0})^2+(y-y_{0})^2+(z-z_{0})^2]^{\frac{3}{2}}}dS$$
\end{definition}

\chapterimage{chap14.jpg}
\chapter{第二型曲线和曲面积分}
\section{第二型曲线积分}
\begin{definition}[第二型曲线积分]
	物理意义: 变力沿曲线做功
	$$\int_{L}P(x,y)dx+Q(x,y)dy \quad \int_{\Gamma}P(x,y,z)dx+Q(x,y,z)dy+R(x,y,z)dz$$
\end{definition}
\subsection{格林公式}
\begin{theorem}
	格林公式: (第二型曲线积分 $\rightarrow$ 二重积分 )
	$$\oint_{L}P(x,y)dx+Q(x,y)dy=\oiint\limits_{D}(\frac{\partial Q}{\partial x}-\frac{\partial P}{\partial y})d\sigma$$
	
	前提条件: $L$取正向,左手在内侧,$L$闭合.一般适用于平面曲线
\end{theorem}
\subsection{斯托克斯公式}
\begin{theorem}[斯托克斯公式]
	斯托克斯公式: (第二型曲线积分 $\rightarrow$ 第一型曲面积分 )
	$$\oint_{\Gamma}Pdx+Qdy+Rdz=\iint\limits_{\Sigma}\left| \begin{array}{lll}
		\cos\alpha&\cos\beta&\cos\gamma\\
		\dfrac{\partial}{\partial x}&\dfrac{\partial }{\partial y}&\dfrac{\partial}{\partial z}\\
		P&Q&R
	\end{array}\right|dS $$
\end{theorem}
\section{第二型曲面积分}
\begin{definition}[第二型曲面积分]
	物理意义: 向量场通过一个曲面的通量
	$$\iint\limits_{\Omega}P(x,y,z)dydz+Q(x,y,z)dxdz+R(x,y,z)dxdy$$
\end{definition}
\subsection{高斯公式}
\begin{theorem}[高斯公式]
	高斯公式: (第二型曲面积分 $\rightarrow$ 三重积分)
	$$\iint\limits_{\Omega}P(x,y,z)dydz+Q(x,y,z)dxdz+R(x,y,z)dxdy=\iiint\limits_{\Omega}(\frac{\partial P}{\partial x}+\frac{\partial Q}{\partial y}+\frac{\partial R}{\partial z})d\nu$$
\end{theorem}